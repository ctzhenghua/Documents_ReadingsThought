\documentclass[UTF8,a4paper,12pt]{ctexbook} 

\usepackage{graphicx}%学习插入图
\usepackage{verbatim}%学习注释多行
\usepackage{booktabs}%表格
\usepackage{geometry}%图片
\usepackage{amsmath}
\usepackage{amssymb}
\usepackage{listings}%代码
\usepackage{xcolor}  %颜色
\usepackage{enumitem}%列表格式
\setenumerate[1]{itemsep=0pt,partopsep=0pt,parsep=\parskip,topsep=5pt}
\setitemize[1]{itemsep=0pt,partopsep=0pt,parsep=\parskip,topsep=5pt}
\setdescription{itemsep=0pt,partopsep=0pt,parsep=\parskip,topsep=5pt}
\usepackage{tcolorbox}
\usepackage{algorithm}  %format of the algorithm
\usepackage{algorithmic}%format of the algorithm
\usepackage{multirow}   %multirow for format of table
\usepackage{tabularx} 	%表格排版格式控制
\usepackage{array}	%表格排版格式控制
\usepackage{hyperref} %超链接 \url{URL}
\usepackage{tikz}
\usepackage{dirtree}

\CTEXsetup[format+={\flushleft}]{section}

%%%% 设置图片目录
\graphicspath{{figure/}}

%%%% 段落首行缩进两个字 %%%%
\makeatletter
\let\@afterindentfalse\@afterindenttrue
\@afterindenttrue
\makeatother
\setlength{\parindent}{2em}  %中文缩进两个汉字位

%%%% 下面的命令重定义页面边距,使其符合中文刊物习惯 %%%%
\addtolength{\topmargin}{-54pt}
\setlength{\oddsidemargin}{0.63cm}  % 3.17cm - 1 inch
\setlength{\evensidemargin}{\oddsidemargin}
\setlength{\textwidth}{14.66cm}
\setlength{\textheight}{24.00cm}    % 24.62

%%%% 下面的命令设置行间距与段落间距 %%%%
\linespread{1.0}
\setlength{\parskip}{0.5\baselineskip}
\geometry{left=1.6cm,right=1.8cm,top=2cm,bottom=1.7cm} %设置文章宽度
\pagestyle{plain} 		  %设置页面布局

%代码效果定义
\definecolor{mygreen}{rgb}{0,0.6,0}
\definecolor{mygray}{rgb}{0.5,0.5,0.5}
\definecolor{mymauve}{rgb}{0.58,0,0.82}
\lstset{ %
	backgroundcolor=\color{white},   % choose the background color
	basicstyle=\footnotesize\ttfamily,      % size of fonts used for the code
	%stringstyle=\color{codepurple},
	%basicstyle=\footnotesize,
	%breakatwhitespace=false,         
	%breaklines=true,                 
	%captionpos=b,                    
	%keepspaces=true,                 
	%numbers=left,                    
	%numbersep=5pt,                  
	%showspaces=false,                
	%showstringspaces=false,
	%showtabs=false,        
	columns=fullflexible,
	breaklines=true,                 % automatic line breaking only at whitespace
	captionpos=b,                    % sets the caption-position to bottom
	tabsize=4,
	commentstyle=\color{mygreen},    % comment style
	escapeinside={\%*}{*)},          % if you want to add LaTeX within your code
	keywordstyle=\color{blue},       % keyword style
	stringstyle=\color{mymauve}\ttfamily,     % string literal style
	frame=L,
	xleftmargin = .04\textwidth,
	rulesepcolor=\color{red!20!green!20!blue!20},
	% identifierstyle=\color{red},
	language=c++,
}
 \author{\kaishu 郑华}
 \title{\heiti 2019读书笔记}
 
\begin{document}          %正文排版开始
 	\maketitle

\chapter{文学类}
	\section{红楼梦}
		遁入空门
		
		人世浮沉,只是世上走一遭。
	\section{1984}
		作者营造了当时或者心中 社会主义政权的社会环境,社会结构。有老大哥,核心党、外围党、无产阶级。
		
		生活的环境随时受着电幕的监控,和思想警察的监视,可以说是极度的一种不自由,而在这种不自由的情况下,人们尝试着一些内心的反抗,或者是温斯顿和裘莉亚 的反抗,因而在这段不易得来的自由时光里使得两人的肉体之情发展到情爱。
		
		而为了生存,或者躲避严酷的酷刑,或者在被折磨的无意识下、在相互承诺的前提下、出卖对方求取生存、或者是求取最后的解脱。
		
		里面有一个角色,虽然作者用了很少的笔墨,但是也对于我来说是一个比较不寻常的存在,赛麦,因为太聪明,话太多,不知道隐藏自己,或者不知道装傻,早早被思想警察带走。
		
		是一个故事,一段历史。
		
	\section{漫长的告别}
		这个破烂的开头让我还犹豫了好久,它到底在讲什么,但是读到他的朋友,或者他的点头之交,杀害文头提到的漂亮的女人的时候,或者因杀掉她然后自杀,才恍然知道这个开头也是精心安排的。
		
		是的,前10节,我看到的是这期案子的起因、结果、和作者本人为朋友(特里·伦纳克斯)而死活不招供的描述。
		
		然后再提到了作者朋友的几个朋友关于特里·伦纳克斯的几点讨论。
		
		接下来的几章描述了一个看似不相关的案子,作家韦德,在帮助韦德太太找韦德的过程中,时不时提醒几句关于特里·伦纳克斯的星星点点信息。
		
		与特里·伦纳克斯的妻子的朋友认识,与特里·伦纳克斯妻子的姐姐认识,与特里·伦纳克斯妻子的父亲认识。
		
		韦德死了,在作者当前的描述中,也好像不是自杀的,被谋杀。
		
	\section{爱你就像爱生命}
		
	\section{局外人}
	
	\section{白夜行}
	
	\section{棋王}
	
	\section{源泉}
	
	\section{变形记}
	
	\section{霍乱时期的爱情}
	
	\section{围城}
	
	\section{百年孤独}
		一个家族7代人的故事。兴衰史?平淡生活的点点记录?曲折离奇日子的描述?一段注定百年孤独的家族注定只会在这个世上出现一次!是珍惜现在,尊重宝贵的历史遗产?
		
		描写马贡多被殖民者文明侵入后,然后下了4年多少的雨,留下的是什么?带走的又是什么?
		
		起头的猪尾巴的可怕担忧,竟成了本家族最后的结尾。
		
	\section{麦田里的守望者}
		老实话,没看懂
		
	\section{我与地坛}
	
	\section{基督山伯爵}
	
	\section{3个火枪手}	
	
	\section{穆斯林的葬礼}
	
	\section{金锁记}
	
	\section{动物农场}
	
	\section{月亮与六便士}
	
	\section{无声告白}
	
	\section{鼠疫}
	
	\section{牧羊少年奇幻之旅}
	
	\section{人间失格}
	
	\section{九故事}
	
	\section{追风筝的人}
	
	\section{黄金时代}
	
	\section{平凡的世界}
		
	\section{呐喊}
	
	\section{通往奴役之路}
	
	\section{卡拉马佐夫兄弟}
	
	\section{呼啸山庄}
	
	\section{悲惨世界}
	
	\section{金瓶梅}
	
\chapter{哲理性}
	\section{论语}

	\section{大学}
	
	\section{中庸}
	
	\section{曾国藩家书}
	
	\section{纯粹理性批判-康德}
	
	\section{实践理性批判-康德}
	
	\section{判断力批判-康德}
	
	\section{少有人走的路}
\chapter{经济类}
	\section{国富论}
	
	\section{财富自由之路}

	\section{经济学通识}
	
	
\chapter{人文社科传记类}
	\section{青春恋爱心理学}		
	
	\section{富兰克林传}

	\section{格局逆袭}
	
	
\chapter{历史类}
	\section{资治通鉴故事}
	
	\section{明朝那些事}

\chapter{诗词类}
	\section{唐诗}

	\section{宋词}


\chapter{技术类} 
	\section{Linux 环境编程-从应用到内核}
	
	\section{Unix 环境高级编程}

	\section{Redis 设计与实现}
	
	\section{Shell 高级编程}
	
	\section{后台开发:核心技术与应用实践}
	
	\section{深入理解计算机操作系统}
	
	\section{Linux 内核设计与实现}
	
	\section{深入理解Linux 内核}
	
	\section{LevelDB}
	
	\section{RocksDB}
	
	\section{深入探讨MySQL}
	
	\section{深入探索C++ 对象模型}
	
	\section{LeetCode500题}
	
	\section{设计模式}
	
	\section{代码大全2}
	
\end{document} 
 		    
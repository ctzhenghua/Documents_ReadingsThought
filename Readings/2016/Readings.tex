\documentclass[UTF8,a4paper,8pt]{ctexart} 

 \usepackage{graphicx}%学习插入图
 \usepackage{verbatim}%学习注释多行
 \usepackage{booktabs}%表格
 \usepackage{geometry}%图片
 \usepackage{amsmath} 
 \usepackage{amssymb}
 \usepackage{listings}%代码
 \usepackage{xcolor}  %颜色
 \usepackage{enumitem}%列表格式
 \usepackage{hyperref}
 \CTEXsetup[format+={\flushleft}]{section}


\geometry{left=1.6cm,right=1.8cm,top=2cm,bottom=1.7cm} %设置文章宽度

\pagestyle{plain} 		  %设置页面布局
\author{郑华}
\title{读书笔记}
 %代码效果定义
 \definecolor{codegreen}{rgb}{0,0.6,0}
 \definecolor{codegray}{rgb}{0.5,0.5,0.5}
 \definecolor{codepurple}{rgb}{0.58,0,0.82}
 \definecolor{backcolour}{rgb}{0.95,0.95,0.92}
 
 \lstdefinestyle{mystyle}{
 	backgroundcolor=\color{backcolour},   
 	commentstyle=\color{codegreen},
 	keywordstyle=\color{magenta},
 	numberstyle=\tiny\color{codegray},
 	stringstyle=\color{codepurple},
 	basicstyle=\footnotesize,
 	breakatwhitespace=false,         
 	breaklines=true,                 
 	captionpos=b,                    
 	keepspaces=true,                 
 	%numbers=left,                    
 	%numbersep=5pt,                  
 	showspaces=false,                
 	showstringspaces=false,
 	showtabs=false,                  
 	tabsize=2
 }
\lstset{style=mystyle, escapeinside=``}

\begin{document}          %正文排版开始
 	\maketitle
 	\tableofcontents
 	
 \section{《走错路,也能到对岸》 }
 
 \newpage
 \section{《穷爸爸富爸爸》-罗伯特 }
 \begin{itemize}
 	\item 我还是震惊于那些成年人连会计和投资方面的基本知识都不曾具备,他们对收支平衡表和资产负债表间的因果关系知之甚少,当他们买卖资产时,总是难以记住每笔交易都会对他们的每月现金流量产生影响
 	\item 当宣布缩编时上市公司股价通常会上升,市场喜欢这样的消息,因为当公司减人时成本就下降了,这意味着公司通过自动化提高了平均劳动生产率。”
 	\item 经济技能和交流技能也十分重要,甚至可以说更为重要。
 	\item 记住,经济头脑是在解决我们经济问题的过程中锻炼出来的。
 	\item 认识到一个人的观念对其一生的巨大影响力
 	\item 一个爸爸爱说“我可付不起”这样的话,而另一个爸爸则禁止用这类话,他会说:“我怎样才能付得起呢?”
 	\item 这并不意味着人们必须去买每一件你想要的东西,这里只是强调要不停地锻炼你的思维――实际上人的大脑是世界上最棒的“计算机”
 	\item 在他看来,轻易就说“我负担不起”这类话是一种精神上的懒
 	\item 选择不同,命运也是不同的。
 	\item 钱来了又去,但如果你了解钱是如何运转的,你就有了驾驭它的力量,并开始积累财富。光想不干的原因是绝大部分人接受学校教育后却没有掌握钱真正的运转规律,所以他们终生都在为钱而工作
 	\item “如果你们放弃了你们才真的只能当穷人了。一件事情的成败并不重要,重要的是你们曾经尝试过。要知道大多数人只是谈论和梦想发财,而你们已经付出了行动。我再说一遍,我为你们骄傲,孩子们,别灰心,别放弃。”
 	\item 但生活可不是这样的教法。你知道吗,生活才是最好的老师,大多数时候,生活并不对你说些什么,它只是推着你转,每一次推,它都像是在说‘喂,醒一醒,有些东西我想让你学学”
 	\item “假如你弄懂了生活这门大课,做任何事情你都会游刃有余。但就算你学不会,生活照样会推着你转。所以生活中,人们通常会做两件事。一些人在生活推着他转的同时,抓住生活赐予的每个机会;而另一些人则听任生活的摆布,不去与生活抗争。他们埋怨生活的不公平,因此就去讨厌老板,讨厌工作,讨厌家人,他们不知道生活也赐予了他们机会。”
 	\item 如果你是那种没有毅力的人,你将放弃生活对你的每一次推动。这样的话,你的一生会过得稳稳当当,不做错事、随时准备着当永远不会发生的事情发生时解救自己,然后,在无聊中老死。你会有许多像你一样的朋友,希望生活稳定、处世无误。
 	\item 但事实是,你对生活屈服了,不敢承担风险。你的确想赢,但失去的恐惧超过了成功的兴奋,事实是从内心深处,你就始终认为你不可能*,所以你选择了稳定。
 	\item 大多数人认为世界上除了自己外,其他人都应该改变。让我告诉你吧,改变自己比改变他人更容易
 	\item 可我知道我会一辈子去研究钱这东西,因为我研究得越深,知道的东西也就越多。大多数人从不研究这个题目,他们去上班,挣工资,然后去开销,总也不明白为何老被钱所困扰,于是以为多点钱就能解决问题,却几乎没有人意识到缺乏财务知识才是他们真正的问题所在。
 	\item 工作只是试图用暂时的办法来解决长期的问题
 	\item 驴子在拼命拉车,因为车夫在它鼻子前面放了个胡萝卜。车夫知道该把车驶到哪里,而驴却只是在追逐一个幻觉。但第二天驴依旧会去拉车,因为又有胡萝卜放在了驴子的面前
 	\item 人生实际上是在无知和幻觉之间的一场斗争
 	\item 你要花时间去想这样的问题,更努力地工作是解决问题的最好方法吗?
 	\item 从一开始我们自己的小人书阅览室起,我们开始自己赚钱,而不是依赖雇主。尤其是我们的生意给我们带来了钱,甚至于当我们不在那儿时,它也在生钱,我们的钱为我们工作了。 没有付给我们工钱,富爸爸却给了我们更多的东西。
 	\item 他曾一遍又一遍地对我们说:“聪明人总是雇比他更聪明的人。
 	\item 如果一对年轻夫妇早点在他们的资产项中多投些钱,以后几年他们就会过得轻松些,尤其是他们准备把孩子送人大学的话。因为资产项中的投资会使他们的资产不断增加,自动弥补支出。而先投资买下一所大房子的做法只不过是取得抵押贷款以支付不断攀升的开支,其结果不过是拆了东墙补西墙
 	\item 决定拥有很昂贵的房子,而不是早早地开始证券投资,将对一个人的财务生活在以下三个方面形成冲击:
 	\begin{enumerate}
 		\item 失去了用其他资产增值的时机
 		\item 本可以用来投资的资本将用于支付房子的各种高额、长期开支
 		\item 失去受教育机会。人们经常把他们的房子、储蓄和退休金计划列入他们的资产项目
 	\end{enumerate}
 	\item 如果我想增加支出,我首先必须增加资产项产生的现金流来维持我的财富水平
 	\item 这时我不再依赖工资,如果我辞职了,我每月还能用资产项产生的现金流维持支出,也就是说我仍能够生存
 	\item 我的下个目标是从资产中得到多余现金再进行投资。流入资产项的钱越多,资产就增加得越快;资产增加得越快,现金流入得就越多。只要我把支出控制在资产所能够产生的现金流之下,我就会变富,就会有越来越多除我自身劳动力收入之外的其他收入来源
 	
 	\item 记住下面这些话:富人买入资产;穷人只有支出;中产阶级买他们以为是资产的负债
 	
 	\item 真正的资产可以分为下列几类:
 	\begin{enumerate}
 		\item 不需我到场就可以正常运作的业务。我拥有它们,但由别人经营和管理。如果我必须在那儿工作,那它就不是我的事业而是我的职业了
 		\item 股票
 		\item 债券
 		\item 共同基金
 		\item 产生收入的房地产
 		\item 票据(借据)
 		\item 专利权如音乐、手稿、专利
 		\item 任何其他有价值、可产生收入或可能增值并且有很好的流通市场的东西
 	\end{enumerate}
 	\item 喜欢小公司的股票,尤其是刚成立的公司,原因是我是一个企业家而不是一个雇员
 	\item 对于小公司,我的投资策略是:1年内脱手;另一方面我的房地产投资策略则是从小买卖开始并一点点做大,条件允许的话尽量晚一些出手,这样做的好处是可以推迟缴纳所得税,从而使资产可能戏剧般地增加。我通常持有房地产在7年以上。
 	\item 一个重要的区别是富人最后才买奢侈品,而穷人和中产阶级会先买下诸如大房子、珠宝、皮衣、宝石、游艇等奢侈品,因为他们想看上去很富有
 	\item 财务知识是非常重要的技能
 	\item 第二是投资,我称为钱生钱的科学。投资涉及到策略和方案,这是右脑要做的事,或者说是创造
 	\item 第三是了解市场,它是供给与需求的科学。这要求了解受感情驱动的市场的“技术面”
 	\item 我们都拥有巨大的潜能――这一上天赏赐的礼物。然而,问题是我们都或多或少地存在着某种自我怀疑,从而阻碍了自己的前进。阻碍我们前进的障碍很少是由于缺乏技术性信息,更多的是由于缺乏自信
 	\item 我的班上,我极力劝说学生们学着去冒险,去勇敢地发挥才能,把畏难情绪转化成动力和智慧
 	\item 富人则创造金钱
 	\item 不错,每月拿出一笔钱存起来听上去确实是一个好的主意
 	\item 我有一大笔钱投资于股票和房地产,手头缺少现金。这是因为每个人都在卖出,而我却在买入。我不是在储蓄金钱,而是在投资。我太太和我有一百多万美元的现金投在了将要迅速上升的市场上,我们相信这是最好的投资机会。
 	\item 原先价值10万美元的房屋现在只值7.5万。但我没有去找本地房地产公司买进这些房地产,而是去找破产事务律师办公室,或者通过法院开始洽谈业务。在这些地方,一幢7.5万美元的房屋有时可以按2万美元或更低的价格买下。首先,我以现金支票的形式支付给律师2000元定金,这是我向朋友借的,为期90天。 利息20O 元。当购买程序刚一启动,我就在报纸上刊登售房广告,以6万美元、首期付款为零的条件,卖出这幢价值7.5万美元的房屋。我的电话铃很快就响个不停,我对有希望成交的买主―一进行了调查筛选。然后,当房屋在法律上归我所有后,所有有望成交的买主都被允许去实地察看这幢房屋。交易非常火爆,房子在几分钟之内就售出了。我要求得到2500美元的手续费,买主很高兴地支付了。这笔钱我用于支付提供了中介服务的公司、偿还我的朋友2000美元和额外的200美元利息。在这笔交易中,我的朋友高兴、房屋的买主高兴、律师高兴,而我,当然更高兴。我支付2万美元的成本买入一幢房子,又以6万美元的价格卖出去,净赚的4万美元以买主开出的承兑汇票的形式流入我的资产项目。 所有的工作时间累计起来只有5个小时
 	
 	\item 你会看到,当我第一次卖出房屋时,我归还了2000美元。从技术上讲,我在交易中没有投人任何资金,可我的投资回报是无穷大。这就是无钱变有钱的一个很好的例子
 	\item 我主要使用两种工具来实现资金的增值:房地产和小型公司股票。房地产是基础,通过每月买进卖出,我的财产不断地提供现金流人,偶尔也会有价值上的飘升。再有就是等待小型公司股票的快速增值
 	\item 在任何情况下,成功的办法就是运用你的技术知识、智慧以及对于游戏的喜爱来减少意外情况的发生并降低风险
 	\item 好机会是用你的脑子而不是用你的眼睛看到的。大部分人没办法致富仅仅是因为他们没有在财务上受到训练,因而不能认识到机会其实就在他们面前
 	\item 我喜欢房地产是因为它很稳定,变化比较缓慢
 	\item 胜利者是不怕失去的,但失败者都害怕失去。失败是成功之母,如果避开失败,也就避开了成功
 	\item 如果你看看人类学习的方法,就会明白人类其实就是在犯错误的过程中进行学习的。我们从跌倒中学会了走路,如果我们从不跌倒,我们就永远也学不会走路。学骑自行车也是同样的道理,尽管我的膝盖上仍有伤疤,但我今天骑自行车时已不费吹灰之力了。富裕起来更是同样的道理,不幸的是,大部分人不富有的主要原因就在于他们太担心失去
 	\item 如何寻找到其他人都忽视的机会
 	\item 如何增加资金
 	\item 大多数人眼睁睁地让缺少资金阻止了他们去做成一笔交易,如果你能越过这些障碍,你就能比那些没有掌握这些技能的人早一步成为百万富翁
 	\item 因为从事房地产投资,我学会了如何不找银行就能买下房子的技巧。房子本身并不太重要,而学到的融通资金的技巧却是无价之宝
 	\item 在银行没有一分钱存款的情况下,买下了房子、股票或公寓楼。有一次我买了一幢价值120万美元的公寓楼,我的办法就是“成为联系的桥梁”,即通过在卖方和买方之间订立一纸合同来实现目的
 	\item 怎样把精明的人们组织起来
 	\item 风险总是无处不在,要学会驾驭风险,而不是一味回避风险
 	\item “他们只有一项技能,所以他们挣不到大钱。”这句话的意思是说,大部分人需要学习并掌握不止一项技能,只有这样他们的收人才能获得显著增长
 	\item “对许多知识你只需要知道一点就足够了”,这是他的建议
 	\item 对于受过良好教育的爸爸来说,稳定的工作就是一切。而对于富爸爸来说,不断学习才是一切
 	\item \textbf{销售技能是个人成功的基本技能},它涉及到与其他人的交往,包括与顾客、雇员、老板、配偶和孩子的交往。
 	
 	\item \textbf{而交际能力},如书面表达、口头表达及谈判能力等对于一个人的成功更是至关重要。我就是通过学习各种课程、买来教学磁带等来增长知识并木断提高自己的这一技能而最终获得成功的。
 	\item 他们在财务上不能获胜的原因是因为对他们而言损失金钱所造成的痛苦远远大于致富所带来的乐趣。得克萨斯人的另一句谚语讲道:“人人都想上天堂,却没有人想死。”可是不死怎么能进入天堂呢,这就如同大部分人梦想发财,但却害怕在投资过程中损失金钱,所以他们永远进不了“天堂”
 	\item 他们接受失败的现实并把它转变成通向成功道路上的一个个插曲
 	\item 得克萨斯人并不掩饰他们的失败,他们愈挫愈奋,他们接受自己失败的现实并将失败转化为动力。失败激发得克萨斯人成为成功者,而这个公式并不仅只适用于得克萨斯人,它适用于所有的成功者
 	\item 因为我们想守着那些安全的东西,而机会却从身边溜掉了
 	\item 埋怨使人头脑受蒙蔽,而分析使人心明眼亮。进行分析能使成功者看到那些愤世嫉俗者无法看到的东西,也能发现被其他人都忽视了的机会,而发现人们忽视了的机会的能力正是取得成功的关键
 	\item 他们总是批评而不是去分析,总是看到细节上的麻烦而看不到解决麻烦之后总体上的巨大收益
 	\item 如果大多数人懂得股票市场上“横盘”(预定低点抛售)意味着投资机会的话,就会有更多的人去投资以赢利而不是投资以避免损失
 	\item “我怎样才能够支付这个?”
 	\item 你怎样\textbf{克服懒惰心理}呢?答案是多一点点“贪婪”,要勇于去追求并得到自己所想要的生活。
 	\item 没有想拥有更好东西的渴望,就不会取得进步。
 	\item 如果你知道自己在某一问题上欠缺知识,不要试图掩饰,因为那是在欺骗你自己,你应该做的是去找一位这一领域的专家或者找一本有关这一问题的书,马上开始教育自己
 	\item 如果你不够坚强,那么前行道路上的严酷现实就会迫使你退缩
 	\item 给自己一个强有力的理由或目标。若非如此,你在生活中会感到步履维艰
 	\item 在积累财富的过程中,最困难的事情莫过于坚持自己的选择而不盲目从众
 	\item 穷人有不好的习惯,一个普遍的坏习惯是随便“动用储蓄”
 	
 	\item 我们上学去学习某种技能专长,这样我们可以为金钱而工作,但我的观点是:学会让金钱为你工作更加重要
 	\item 每当我觉得自己需要点什么,或者缺钱,或者缺少帮助时,我就去想一想,自己心里到底需要什么,然后首先为此而付出
 	\item 我发现,越是真诚地教那些想学习的人,我从中学到的就更多
 	\item 找一个做过你想做的事情的人,请他和你一块共进午餐,向他请教一些诀窍和一些做生意的技巧
 	\item 至于股票,我喜欢彼得。林奇的(称雄华尔街)一书中介绍的选择价值有上升潜力股票的方法
 	\item 你先要知道你在寻找什么,然后再去寻找它!
 	\item 买下馅饼并把它切成小块。大部分人寻找的是自己能够支付的东西,这样他们看到的都是较小的东西。他们只购买一块馅饼,却因此付出更多。只盯着小生意的人是不会有大的突破的。如果你想致富,就要首先考虑较大的生意
 	\item 行动者总会击败不行动者
 \end{itemize}
 
 \newpage
 \section{《成就你一生的100条哲理》 }
 
 \newpage
 \section{《因为痛,所以叫青春》-(韩)金兰都 }
 
 人生从不会嫌太年轻或者太老,一切都刚刚好。”
 
 
 包括沟通力、领导力、责任感、诚信感、业务处理能力、组织能力等等数不清的条件,
 
 
 你需要奋斗的目标并非是暂时获得一个新人奖,而是成为人生这个大舞台的最佳主角。
 
 
 凡事还是应该给自己留有余地,放下焦虑,为未来设计之门留下一条选择的缝隙。
 
 
 在攀爬很长的阶梯时,眼睛不要盯着最顶端的那一阶,而是要专注于眼前最近的这一阶。一个阶段一个阶段走下去,将目标的视觉距离拉近一些,尽可能稳扎稳打,这比好高骛远容易多了。
 
 
 书基本上每天读一本,在卫生间时,在坐车时,只要一有时间我就会用来读书。
 
 
 最后发现,一个人完全可以在生活中担当多个角色。我最讨厌的话就是‘我没有时间’。”
 
 
 我们为什么会没有时间呢?如朴京哲先生所说,喝酒、打高尔夫、被各种诱惑吸引(对年轻人来说,可能是将大量时间花在游戏和网上冲浪上)……我们就是这样,沉溺于各种暂时的快感而无法自拔,将宝贵的时间一点点浪费殆尽。
 
 
 冷静地确定工作优先级
 
 
 现在最让学生上瘾的应该是网上冲浪和网络游戏。为了在日后回顾走过的青春岁月时不流下悔恨的眼泪,请减少或戒掉这些活动,特别是游戏吧,它们对你的未来毫无益处。
 
 
 从某种程度上说,在这样高速运转的社会里,管理时间主要是管理鸡肋时间,不要等待某一大块时间的出现,我们要学会搜集鸡肋时间去完成大事情。
 
 
 一刻钟之内能做的事情要立即着手解决,而不要用过一会儿再做的想法来安慰自己,那样基本上无法像样地完成任何一件事情。因为你现在不想做,即便再过一会儿也还是不想做,不如赶紧把它解决掉,然后忘记,才会更加轻松。
 
 
 小睡一会儿是我最喜欢的选择。短短十几分钟,往往会解除很大疲劳,如果实在睡不着,就会读报纸。
 
 
 一天中至少拿出一点时间来反省自己,哪怕只是十分钟也好。如果没有认真自我反省,就会不知道自己未来将成为什么样的人,甚至不知道自己此刻究竟在忙什么
 
 
 忙碌才是万事开头容易的好时机。
 
 
 忙碌才是万事开头容易的好时机。在忙碌时,请将手头的时间细分再细分,而后利用其中一小段时间去开始“那个事情”。
 
 
 如果不趁现在,就意味着永远都不能做了。
 
 
 “明日复明日,明日何其多,我生待明日,万事成蹉跎。”你的明天取决于今天的24小时做了什么,因此,从某种程度上说,你的时间比你本身更要珍贵。
 
 
 请回顾一下之前提到的那个谐星的故事。对于他来讲,他所处的时期并非是积累财富的时期,而是积累人生竞争力的时期。与其从每月的牙缝里挤出一点钱来进行可怜的投资,不如等到大红大紫后利用增加的演出费用进行理财,不是更具效率,回报率也更高吗?
 
 
 要想真正将自己提升至一定的高度,你就要暂时大胆放弃一些短期利益,要做好这种“先苦后甜”的准备,才是合理的青春“储蓄”。
 
 
 一个人如果长时间懒惰,会腐朽的,
 
 
 希望你能狠狠下定决心从低迷,不,从懒惰中真正走出来。
 
 
 每当目标不够明确时懒惰就会找来。
 
 
 想走出低迷吗?如果你真想如此,其实比想象的要简单。关键在于将时间周期弄得短一些,没错,就是解决好“今天”的事情就行。
 
 
 你只需踩好“今天”的第一脚就可以满足了。无数个“今天”会以惊人的速度集合到一起,让勤奋集合成惯性。
 
 
 罗马之所以是罗马,全凭风雨
 
 
 但请记住:历练终将使你变得更加强大,这是对年轻最好的祝福。
 
 
 “罗马之所以是罗马,全凭风雨……国家的未来并非战争的胜负决定的,而是战争过后做了什么,更重要的是如何去做。”[
 
 
 因为,那一个人,才是值得你去畅游的无限海洋。
 
 
 爱情是相互成就的关系。爱人就像是自己的一面镜子,通过这面镜子改进自己的不足,将最美的一面呈现给对方,才是爱情。如果
 
 
 人总喜欢隔一段时间犯同样错误
 
 
 这样的笔记本的确很实用,将答错的问题重新回顾一遍既能帮助学生找到自己错误的原因,又能让学生加深印象,在未来遇到同类型的问题时不再重蹈覆辙。
 
 
 而最令人恼火的是,付出了巨大代价之后,却总是继续犯同类型的错误。这种时候自己都不能饶恕自己,为什么我会如此不思悔改呢
 
 
 时间有治愈伤口的功效。
 
 成功人士与平常人的不同之处在于他们拥有客观审视自己的能力。(
 
 
 但我们不能被这种话迷惑。无论对谁来说,眼下才是你人生的暮年,是你苦恼最多,最辛苦的时节
 
 请热爱你眼前的那些苦恼吧,20岁就是用来迷茫不安的,这是必然的经历,请投入全部精力去思考,并谋划自己的将来!
 
 
 总是跟更幸运、拥有更多的人比较,拿别人的长处来比自己的短处,以至于如今能够不假思索地说出“我很幸福”的人并不太多。
 
 
 总是跟更幸运、拥有更多的人比较,拿别人的长处来比自己的短处,以至于如今能够不假思索地说出“我很幸福”的人并不太多。那些已经拥有很多的人也很难感到满足,因为他们感觉自己需要拥有更多,要比周围所有的人拥有的都多。所以才会有那句“只有从不幸中回首时,才能感受到刻骨铭心的幸福”,是的,经历了不幸和失去之后,才会发现自己原来已经足够幸运,拥有的足够多了。
 
 
 也许糟糕的一天,让你想要放弃整个人生, 但就是这样的一天,对于某些人来说也可能是一辈子的奢望。
 
 
 失去发烫的梦想并非只是由于年龄增长,更主要在于安定的生活以及生于忧患死于安乐这一亘古不变的道理。犹如悲伤是诗人的精神食粮,不安是保持梦想发烫的防腐剂。藏于失败之内的革命性让人怀揣梦想,而对梦想不言放弃的自恋[1]情结才能让梦想始终保持发烫的热度。
 
 
 我害怕自己会陶醉于一点点成绩从而忘记最初的梦想,甚至忘记人生所应有的热度,于是今天写下这段文字。希望这段文字能够保持住此刻不安的热度,保持住不安背后那点滴培植的梦想,让梦想不会日后腐败。
 
 
 也不必太过自责;最初下定的决心就算没有坚持几天,也不要失望。及时调整心情,重振旗鼓,继续前行,通过持续的管理,一步步向目标迈近。
 
 
 不抛弃,不放弃, 不要担心, 你的脚步比别人缓慢, 只要记住, 永远不要停下脚步! 边坚持边调整,会完成得十分出色
 
 
 但是这时更需要站在对方的立场上理解和沟通。好朋友和融洽的人际关系并不是从天而降的,而是要通过自己的努力去建立的。
 
 
 我们总是把目光放在自己的前进速度有多快上,盯着时间不放。其实比时间更为重要的是指南针,相比你人生跑得有多快,更重要的是你是否寻找到了正确的方向。
 
 
 从现在做起,行动力更强。
 
 
 不要期待自己突然在某一瞬间醍醐灌顶大彻大悟,通过小小的实践去积累资本吧。当面临一个问题却一时找不到解决方法时,不要在原地苦恼该怎么做,而是应该去思考为了解决这个问题,我今天可以做些什么。
 
 
 当年轻时,可以为崇高的理想而光荣的死。当年长时,可以为崇高的理想而选择卑贱的活。”
 
 
 不要后悔,让对过去的遗憾填满现在的人生。也不要奢望过高,因为不可能完成的事情让现在的生活变得悲惨。不要纠结,因为对未来的不安就会让现在迷茫。坚定自己的目标,向着目的地一步步脚踏实地,这才是真正意义上的及时行乐。
 
 
 事实上,不仅歌词需要押韵,我们的人生何尝不需要押韵呢?给自己设计一些规则去遵守,去制约自己,如果能够同生活里的规则保持步调一致,那么你我都有可能成为人生的诗人。
 
 
 1986年,一天一个小时的投资,一直养活我至今。
 
 
 马尔科姆·格拉德威尔在《异类:不一样的成功启示录》一书中提出了“1万小时法则”。
 
 
 一天3小时,一周20小时,坚持10年才能攒够1万小时。这是普通人很难达到的练习量。当然,光有练习量也是不够的,还要具备非凡的才能。
 
 
 懂得练习和积累的人是不会失败的。练习和积累是为了迎接未来的甜蜜而暂时忍受短暂的痛苦。忍受的时间即使达不到1万小时也可以让你幸福,因为那时的你一定有所变化。
 
 
 那时,我明白了一个道理,决定前途的关键并不是要看别人的眼色行事,而是选择去做能够让自己快乐的事情。
 
 
 所以,请你不要安于现有的成绩,要不断挖掘自身的潜力,要勇敢地超越“现在的我”。陶醉在自我的世界会让人变得不思进取。如果不能完成自我更新,自我升级,那么早晚有一天,社会将把你当做古董遗弃在角落。我们的人生比想象的要漫长,如果只谋求眼前的利益,毫无原则地随波逐流,重复着同样的节目,是不是有些乏味呢? 请不要忘记,有句俗话叫做“破壳而出是生命,砸壳而出是料理”。请遵从那些内心的渴望,向着未来前进吧!
 
 
 你不仅要熟练地驾驭英语,更要驾驭人际关系,不仅要积累知识,更要积累丰富的人生智慧。
 
 
 营销的核心并非自夸,而是要传递给消费者一条值得购买的理由,当这种购买需求真实地体现在产品质量上时,就会形成独特的品牌价值。
 
 
 你无需将你的整个人生和你所具备的所有能力一五一十讲给对方听,只需向招聘单位传达出一条值得聘用你的理由就可以了。
 
 
 负责招聘的人事部门最为重视的是业务执行能力而不是光鲜的简历。
 
 
 众所周知,品牌是一切成功的核心,想要在当今社会站稳脚跟,也要有自己独特的品牌价值。
 
 
 在做事时,三思而行,要想想,这件事会把我塑造成什么样的人,对我的人生有没有帮助,这才是比钱更值得考量的评价标准。
 
 
 我们就像一条条只为奔向大海的小溪,经历过千山万壑的阻隔,却在最终看到浩瀚的海洋时,茫然而不知所措。
 
 
 体验新事物虽然难免失误或碰壁,但通过‘漂亮’的失误去不断学习和提高,反而是宝贵的人生财富。”
 
 请尝试“漂亮”的失误,失误也可以是一种资产。正如丹尼尔·品克的建议一样,应从“漂亮”的失误当中不断吸取经验从而让自己发展。重要的并非是不让失误出现,而是不重复同样的失误,不在同一个地方跌倒。从失误当中学习经验,让失误一点点造就成熟的自己。这样,你就可以坦然期待偶然的发生了。
 
 
 相比你的潜力而言,一个组织更看重的是你的经历。他们并不会在意你简历上的各类能力指数有多高,而是看你过往工作经验中所积累下来的实际业绩。
 
 
 社会是有规矩的,但这种规矩在学校里学不到。”
 \newpage
 \section{《哈佛凌晨四点半》-韦秀英 }
 \begin{itemize}
 	\item 哈佛励志箴言:只有比别人更早、更勤奋地努力,才能尝到成功的滋味。
 	\item 伟大的科学家爱因斯坦也曾说过:“人的差异在于业余时间。”
 	\item 至少你应该明白,成功永远不会敲响懒汉之门!
 	\item “尽力而为”与“竭尽全力”是存在差别的,前者发挥了自己的能力,后者却让自己的潜能得到了充分地开发。 所以说,不管做什么事情,“尽力而为”是远远不够的,这样只能说明你比一般人付出得更多,却无法让自己超越平庸的界限。 只有“竭尽全力”,让自己的潜能得到充分的发挥,你才能取得更突出的成功!
 	\item 如果我的大脑不经常使用,相应的脑神经分支就会出现萎缩死亡的状况。
 	\item 哈佛励志箴言:那些能够做成大事的人,从来不会拒绝从小事做起。
 	\item 我们更应该关注那些微不足道的小事,不要让它们积累成山,变成你成功路上的绊脚石。
 	\item 哈佛励志箴言:懒惰是万恶的源头,它可以很轻易地毁掉一个人,甚至一个民族。
 	\item 懒惰不仅是生活的大敌,也是学习的大敌。一个人一旦养成了懒惰的品性,那么他想要获得成功,就会变得比登天还难。因为懒惰的人总会在风险面前退缩,总是贪图享乐。
 	\item 将“明天开始”变成“现在就动手”
 	\item 想要彻底地除掉杂草,最好的办法就是在土地里种上有用的庄稼。
 	\item 当你为自己想要的东西而忙碌的时候,就没有时间去为不想要的东西而担忧
 	\item 世界上根本就不存在天才,而他之所以可以取得那样的成就,只是因为他将别人喝咖啡的时间用在了工作上
 	\item 年轻人更应该反省自己,曾经犯过的错误就要尽量去避免,这样才能让日后的路更加顺畅!
 	\item 年轻人不仅要反省自己,同时也要学会从他人的经验与教训中汲取营养,让自己不犯同样的错误
 	\item 拿破仑·希尔曾经说过:“拥有自信心的人,可以将一座山移开;相信自己能够成功的时候,你就离成功不远了
 	\item 也正如哈佛成功学导师爱默生所说:“相信自己能,便会攻无不克……不能超越一个恐惧,便从未学会生命的第一课
 	\item 事物本身并不会对人产生影响,能够影响人的是对事物的看法
 	\item 一个人被自卑所束缚,那么在学习和生活中,总会表现得没精打采、自我封闭
 	\item 在日常生活中,我也经常看到一些自卑的青少年朋友,他们表现得意志消沉、颓丧,甚至是堕落。他们往往过于关注外界的评价,时时处处都显得小心谨慎,于是很容易就否定了自己
 	\item 要时刻谨记一句话:自卑的泥沼里永远无法长出成功的花朵!
 	\item 当你不再拖延、不再等待的时候,就会发现恐惧早已离你而去。
 	\item 如何才能克服心中的恐惧感?”我都会告诉他:“现在就开始学习吧!这就是战胜恐惧的第一步!”(我觉得,克服恐惧最好的方法就是经历它)
 	\item 成功者并不是坐在那里,等待成功自己来敲门。只有失败者会才会抱着侥幸心理,期盼好运砸到自己的头上,尽管这种好运只在想象之中
 	\item 一个人获得成功,并不像自己想象中得那样困难,其实只要你有追求目标的自信心,并且不断地尝试与付出,也许将来因为一个想法目标就实现了
 	\item 世界上并没有什么真正的困难,所谓的困难不过是缺乏面对困难的信心!
 	\item 如果一个人在遇到困难的时候,只想着怨天尤人、自暴自弃,那么最终只能是一事无成,抱憾终生了
 	\item 如果你的目标是大海的彼岸,那么就不要在险风恶浪前止步;如果你的目标在远方,那么就不要让自己的信念在苦闷中越发消沉;如果你的目标是高高的蓝天,那么就要展开自己的双翼,努力地逆风飞行!这便是哈佛学子成功的秘诀之一。
 	\item 在我看来,失败者经常会犯一个同样的错误,那就是竭尽全力地去羡慕、模仿别人,可是对自身的宝藏却视而不见。其实,获得成功并不是什么难事,那就是自信地走你自己的道路!
 	\item 一个人不管做什么事情,热忱都是必不可少的品质,因为热忱可以让你全身心地投入,将事情做得更快、更好。这也是每一位成功人士所必须具有的品质。如果我们能够将所有的热忱,都用在每天的学习与生活中,不断学习新的知识,不断掌握新的技能。那么总有一天,我们能够散发出耀眼的光和热
 	\item 你应该拥有足够的热忱去学习、去生活,同时也要有自信去获取成功,去改变这个世界。这才是年轻人应该有的心态。
 	\item 这一刻的热忱要永远比下一刻多!
 	\item 
 	要知道,人的精力和时间都是有限的,我们只能将自己的热忱集中在一个点上,才能做出一番成就出来。(朱熹-理学)
 	\item 当你真心想做某件事情的时候,千万不要觉得为时已晚!
 	\item 
 	当你觉得为时已晚的时候,恰恰是最早的时候!
 	\item 很多事情都是如此,只要你愿意,现在就可以开始,而且永远不会太迟
 	\item 不要让现在的懒惰成为将来的遗憾
 	\item 我们应该明白,无论什么样的成功,都始于心动,成于行动
 	\item 每日反问:
 	\begin{enumerate}
 		\item 我现在多少岁?决定用多少年的时间去实现自己的理想?
 		\item 每天都坚持学习多长时间?现在开始学习了吗?
 		\item 30岁以后你想拥有怎样的生活?怎样的工作?
 		\item 你离自己的梦想还有多远?现在努力吗?
 	\end{enumerate} 
 	\item 上一刻的成功已经变成回忆,等待你的只会是此时此刻的超越,以及下一刻的成功!
 	\item 无论你的理想是什么,想要实现它都不是一朝一夕的事情,而需要长时间的坚持与不断超越自我的决心。
 	
 	\item \textbf{让每一天的自己都是全新的,不要被昨天的成就与失意所困扰。}
 	\item 当你想做什么事情的时候,现在马上就去做!
 	\item 无论你想做任何事情,永远不要等所有的条件都成熟了才开始着手行动,\textbf{否则你将永远处于等待之中!}
 	\item 自然应该想方设法合理利用自己的时间,让学习更有效率一些。因为当今竞争如此激烈的社会中,效率就是一切!
 	\item 只知道等待明天的人,永远也无法将今天握在手里。因为你所等待的明天能够给予你的只有死亡和坟墓。
 	\item 成功的秘诀就是要养成迅速行动的好习惯!
 	\item \textbf{如果今日不走,明天就只能奔跑!}
 	\item 学习是一种循序渐进的过程,连贯性的学习方法有助于巩固我们已经学到的知识,同时也在刺激我们的大脑不断掌握并吸收新知识,如果学习不能做到持之以恒,那么我们不可能享受到成功的喜悦。半途而废,不仅是意志薄弱的表现,同时也证明你是一个行动上的矮子,不能坚持长久的实际行动,最终会被成功拒之千里之外。
 	\item 地不耕种,再肥沃也长不出果实;人不学习,再聪明也目不识丁。
 	\item 只有反复应用所学,才能掌握真才实学
 	\item 首先,积极思考。无论是学习新知识,还是复习旧知识,思考能够让我们收获更多
 	\item 解决问题。要想提高自己应用知识的能力,就要无时无刻不去解决问题,解决的问题可以是新遇到的,也可以是之前解决过的
 	\item 要养成做计划的习惯,有时要做的事太多,我们就会茫然无绪,这时就极易产生浪费时间的现象,正在做某一件事时,突然发现还有一件更重要的事要做,经常顾此失彼。如果我们能够事先制定一个计划,那么一切都会显得有条不紊,自然也会节省不少时间。
 	\item “归零心态是一种空杯效应,指的是一种谦虚的心态。能放下身段、凡事从头来过。”
 	\item 记住:如果你想要喝咖啡,就必须把先把杯子里的茶倒掉。如果你舍不得倒掉茶,那么,把咖啡加进去之后,就不再是茶,也不再是咖啡,成了一种“怪味”
 	\item 遇到问题时,我们要转变观念,要先从自己身上来找原因,不要总认为错的是别人,自己是不可能犯错的。这种想法不叫自信,而叫没有自知之明。
 	\item 
 	在奋斗之前,一定要先弄清楚自己想要的是什么,这样才不会在他人的期许中迷失了自己。
 	\item 勇气是什么?就是指你尽管害怕、尽管痛苦,但还会硬着头皮继续往前走
 	\item 每天花时间区分事情的重要性可以让时间帮助你,而不是阻碍你。
 \end{itemize}
 
 \newpage
 \section{《35岁前要做的33件事》-蔡虹 }
 \begin{itemize}
 	\item 开阔眼界的确是非常重要的。因为了解得越多,才会更加宽容,人也会变得Open(开放)。
 	\item 你应该在自己的时间里干一点特别的事情,以区别于你的众多同事。
 	\item 黄金般的周末,多半也在不愿意起床、懒得梳洗、不想出门中胡乱度过。
 	\item 有一门业余爱好,有的人甚至发展到了相当高的水平,有可能改变你的人生。如同开始提到的爱略特,后来的人知道他不是因为他的职业,他因为写作而著名。
 	\item 当你真正地决定了自己想要什么以后,要做的就是努力把理想变成现实
 	\item "偷得浮生半日闲"。碌碌人生当中,总要偶尔抽出一点时间来,在自己的角落里,稍稍地享受一下。
 	\item 其实最重要的是集中精力和有限的少数几个人保持和谐关系
 	\item 如果你渴望按照自己的想法生活,面对次要关系和可有可无的交际,你真的得学会温和而坚定地说“不”
 	\item 一定要记得家人的生日、圣诞节、过年、情人节、母亲节、儿童节等等特别的日子,常常拥抱他们,以及用语言表达自己的情感。
 	\item 除了亲人和工作伙伴,我们当然还有一些至关重要的社会关系人物。他也许不是你的老同学、不是旧同事或者有什么远亲关系,却是你的知己。下班以后也许你不想回家、也不想回父母家,却愿意和朋友喝咖啡。好像《欲望都市》里四个关系紧密的女人一样。
 	\item 实施以前想想自己是否真的还有精力去做,不要马上答应下来。不会没有付出就有收获,无论得到什么都是有代价的。交朋友这件事情上也是这样。要付出时间、精力去维系。
 	\item 要想获得自在的人生,就要放弃次要的关系;理解长期的关系中必然有“阴晴圆缺”,例如婚姻、例如老朋友;为对你来说真正重要的人腾出时间
 	\item 一个有原则的人,可能让初识的人感觉怪僻,但是长期来说总比模棱两可的人好,自己舒服,让别人也舒服
 	\item 拒绝的时候完全可以理直气壮。我们要努力做一个心胸开阔、进退有礼的人,同时要给与人交往设立一些粗线条的原则
 	\item 朋友就是不分尊卑贵贱、职业高低的,朋友就是朋友。朋友就是你在天寒地冻的时候,想起来心中含有一丝丝暖意的人。”我们每个人都对“朋友”充满了向往
 \end{itemize}
 
 
 \newpage
 \section{《每天学点时间管理术》 }
 
 \newpage
 \section{《不曾走过,怎会懂得》 }
 
 \newpage
 \section{《春天沉醉的晚上》 }
 
 \newpage
 \section{《跟任何人都聊得来》 }
 \begin{itemize}
 	\item 自己少装, 多给对方装的机会。 话糙理不糙, 少一份自我, 多关注对方兴趣和需求,对方会觉得你是真正在意他,信任感就会日益增加,所有的人际关系问题也就迎刃而解了。
 	
 	\item 充分准备是让交谈更为顺利的最简单的方式,预先思考的越多,你就越自信。不管你处在何种水平,都有进步的空间。坚持下去,时间长了,你就会 养成事先充分思考和准备的习惯。
 	
 	\item 当我们对他人感兴趣时,他人才会对我们感兴趣。
 	
 	\item 不管任何学习过程,跟踪自己的每一步学习都很有用。将每一步记录下来,可以让你切实感受到自己的进步。这样你就可以关注自己走过的路,而不是忧虑的看着面前需要走的路了。这样自信心自然就会建立。
 	
 	\item 问问自己:“如果接近对方,最糟糕的结果是什么?”然后尝试一下,看看事实是否和你的假设相吻合。多数情况下,结果都会让你非常意外,非常惊喜。
 \end{itemize}
 
 
	 \subsection{坚持自己的个性,交谈才会更有趣}
	 
	 \subsection{你真的了解自己么}
	 
	 \subsection{为什么听不懂,为什么说不清}
	 
	 \subsection{挑战我们心中的假设}
	 
	 \subsection{找到共同的兴趣点}
	 
	 \subsection{千方百计让自己变得有趣}
	 
	 \subsection{学会从他人的角度出发}
	 
	 \subsection{学会倾听,别人才能聊的开}
	 
	 \subsection{学会提问,别人才能聊的透}
		 对方所说的话,我们都会加上自己的注解,用自己的经历去理解,然后提出自己的问题。这一点是很致命的,很多人际关系的矛盾都源于此。而那些经验丰富的沟通达人会避开这一点,他们往往会从倾听者的角度提出问题,而非从自己的经历和经验出发。
		 
		 如果没有得到想要的得到的回应,那么我就应该更多的谈论对方,少说自己。
		 
		 注意对方说了些什么,而非时刻关注自己的感受。
	 
	 \subsection{压力,一种积极的沟通力量}
	 
		 人际关系中没什么操控术。一切的一切,其关键还是要真诚。
		 
		 你的目的是让对方注意到他们的言行,而不是给予纠正。如果你所说的话让对方觉得是批评,就不会起任何作用了。
	 \subsection{如何应对棘手的交谈}
	 
	 \subsection{战胜交谈拖延症}
	 
	 \subsection{学会记笔记,为人际关系加分}
		 既然知道在哪里能够找到,我为什么还要记住呢? 我把精力节省下来,用以处理更为重要的事情。
		 
		 我们要做的只是在交谈前后各花几分钟的时间来整理信息,这样,交谈的影响就可以最大化。花时间思考整个过程 才是让交谈成为令参与各方都满意的关键所在。
		 
		 通过录音回放你就可以判断自己的说话方式、语音、语调以及词语使用了。
		 
		 每次交谈后要做记录,以便提醒你谈到的话题以及下一次可以探索的新话题。
	 \subsection{移动互联时代的沟通技巧}
	 
	 \begin{itemize}
	 	\item 不要忽视其他人的感受
	 	\item 如果对方感到沮丧,不要咄咄逼人去劝说,对方需要的是感同身受,而不是建议。
	 \end{itemize}
	 
	 
 \newpage
 \section{《微时间管理术》 }
 \begin{itemize}
 	\item 
 	第一,确定阻止你行动的恐惧是什么,并面对它。第二,确定你将在何时、如何完成这项任务。第三,去做!
 	\item 当你列出当天高优先级的工作的时候,试着列出自己的“个人原则”。我最喜欢的一些原则是积极、赞美他人、微笑、认真倾听,以及自律。 一天中都要按照自己列出的个人原则行事,然后在这一天结束的时候像检查其他工作情况一样检查自己是否遵从了个人原则。 由此,你的生活原则会得到逐步改善,你会以积极的方式影响他人,并且会对自己感觉愈发良好。
 	\item 早睡早起”。这么做的人可以获得黄金一小时。
 	\item 当你放松的时候,你的潜意识就空闲下来继续为你工作了。答案会在你意想不到的时候出现,有时候会是凌晨3点。当答案出现的时候,赶紧把它写下来,免得它再次溜走。	
 \end{itemize}
 
 \newpage
 \section{《乔布斯的魔力演讲》- 卡迈恩·加洛}
 乔布斯的演讲遵照亚里士多德经典的五要素原则,树立了一个有说服力的论点
 \begin{itemize}
 	\item 讲述一个故事或提出一个观点,激发听众的兴趣
 	\item 抛出一个问题,必须得到解决或回答
 	\item 对你提出的问题给出一种答案
 	\item 描述采纳你的解决方案能带来的具体利益
 	\item 号召听众行动起来
 \end{itemize}

\newpage
 \section{《 岛上书店 》-加布瑞埃拉•泽文  }
 \begin{itemize}
 	\item 大多数人如果能给更多事情一个机会的话,他们的问题都能解决。
 	\item 独自生活的难处,在于不管弄出什么样的烂摊子,都不得不自己清理。
 	\item 不过我觉得我后来的反应也说明了读小说需要在适合它的人生阶段去读。
 	\item 我们在二十岁有共鸣的东西,不过强迫孩子们读那种书,就好像让他们觉得自己讨厌阅读。
 	\item “好的婚姻,至少有一部分是阴谋。”
 	\item 因为从心底害怕自己不值得被爱,我们独来独往
 	\item “然而就是因为独来独往,才让我们以为自己不值得被爱。有一天,你不知道是什么时候,你会驱车上路。有一天,你不知道是什么时候,你会遇到他(她)。你会被爱,因为你今生第一次真正不再孤单。你会选择不再孤单下去。”
 	\item 要是有谁觉得你在一屋子人中是独一无二的,就选那个人吧
 	\item 混蛋。我喜欢你,我习惯了你,你是唯一,你这个混蛋。我不想再去认识新的人。
 \end{itemize}
 \newpage
 \section{《 浪潮之巅 》 }
 
 \newpage
 \section{《 一分钟能做什么 》 }    
 \paragraph{动起来}“我们培养了某些习惯,这些习惯也会开始培养我们。” -- 爱默生
 \\
 
 当你为了得到一个理想的结果,必须要采取一些行动的时候,得到结果之前的过程总是不会那么顺心的。 不管为了减小腰围参加慢跑,还是为了省钱减少开支,任何时候做这些事情都不会让你感觉好受。
 不过,等到验收成果的那一天,囊你的腰围终于纤细了,而整个人也变得更加苗条了,或者你终于节省出了足够多的钱平衡财政,你肯定会明白为什么“阳光总在风雨后”。
 
 这段话起初看确实能让我recall some excited memory,可是问题在于我现在被这个拖延症搞得不想动...
 
 
 想象你干完这件事的情景
 
 把你的想法列成提纲
 
 先干上几分钟..
 
 \paragraph{心态} "逆境最有利于激发人的天分,而安逸的环境只会让它冬眠... --- 贺拉斯"
 
 别在日程表上塞太多东西..
 
 在追逐远大的目标的时候,哪怕一点小的行动也比什么都不做要好。
 
 解决拖拉的最好办法就是找到阻碍你前进的根源,并解决他。
 
 令人不快的任务并不会随着时间的流逝变得令人愉快。 许多被不断推迟的任务,比如清理马厩,拖拉越久,就会变得越糟糕。所以,如果你不得不做一些事,不如现在就开始动手,一而再再而三的推迟,绝对不会给你带来任何好处..
 
 \newpage			
 \section{《 乖,摸摸头 》- 大冰 }
 \subsection{故事}
 \paragraph{1.杂草野}
 \paragraph{2.大兵}
 \paragraph{3.那个流浪狗}
 \subsection{句}
 \begin{itemize}
 	\item 你身边是否有这么几个人? 不是路人,不是亲人,也不是恋人、情人、爱人。 是友人,却又不仅仅是友人,更像是家人。 —这一世自己为自己选择的家人。
 	\item 有些话,年轻的时候羞于启齿,等到张得开嘴时,已是人近中年,且远隔万重山水。
 	\item 不是不爱喝,但分与谁醉。
 	\item 它会把你欠下的对不起,变成还不起。又会把很多对不起,变成来不及。
 	\item 时间无情第一,它才不在乎你是否还是一个孩子,你只要稍一耽搁、稍一犹豫,它立马帮你决定故事的结局。 它会把你欠下的对不起,变成还不起。 又会把很多对不起,变成来不及。 我不确定她最后是否跑赢了时间,那句“对不起”,是否来得及。
 	\item 情义这东西,一见如故容易,难的是来日方长的陪伴。
 	\item 能当上一辈子彼此陪伴的普通朋友,已是莫大的缘分了。
 	\item 没有什么过不去,只是再也回不去。
 	\item 雨林里,阿明挖着鸡,唱着歌,想念着外公外婆,身上和心里都是湿漉漉的。 有时候他会停下来哭一会儿。 然后接着挖。
 	\item 其实每个人都会遇到想要的人,可惜大多数人在遇到对方时,己身却并未做好准备,故而,往往遗憾地擦肩。
 	\item 第一堂课老师问了一个问题:正确地做事与做正确的事,你愿意选择哪个?她举手问:只要正确地做事,做的不就是正确的事吗? 
 	\item 公司发我薪水,那我就应该对得起这份薪水,这是一种必然的责任。但我在工作时间内履行这份责任就好,没必要搭上我的私人时间,否则就是对自己的不负责任。我觉得最负责任的做法就是,上班认真工作,下班认真生活,二者谁都不要侵占对方的时间,这样才能保证质量。所以,姑娘我不加班。
 	\item 热情和责任,哪个更持久?靠热情去维持的工作不见得能长久,靠契约精神去履行自己的责任才是王道。
 	\item 十三年的长跑后,当下他们遇到的对方,都是最好的自己。 她和他懂得彼此等待、彼此栽种、彼此付出,她和他爱的都不仅仅是自己。越是美好的东西,越需要安静的力量去守护。
 	\item 其实这个世界上的大部分传奇,不过是普普通通的人们将心意化作了行动而已。不论驻守还是漂流,不论是多项选择还是单项选择。 心若诚一点儿,自然会成为传奇。
 	\item 男人哦,不论年龄多大、经历过什么,总会保留几分孩子气的,听说这种孩子气只会在他们爱的人面前时隐时现。
 	\item 爱一个人,若能有条不紊地说出一二三四个理由来,那还叫爱吗?
 	\item 好日子不是别人单方面给的,我既然真爱他,就不能单方面地指望他、倚靠他、向他索取。他照顾我,我也要照顾他,两个人都认真地付出,才有好日子
 	\item 我不需要靠鲜花钻戒宾朋满座来营造存在感,也不需要像开发布会一样向全世界去宣布和证明,朋友们的祝福一句话一条信息即可,就不必走那些个形式了。我的生活是过给我自己的,编剧是我、导演是我、主演是我、观众还是我,不是过给别人看的。 我知道,于成子而言,也是一样的。 其实对于每一个人而言,这不都应该是事情本来该是的样子吗?
 	\item 世上没有什么命中注定,所谓命中注定,都基于你过去和当下有意无意的选择。
 	\item
 	\item 
 	\item 
 	\item 
 	\item 
 	\item 
 	\item 
 	\item
 	\item 
 	\item 
 	\item 
 	\item 
 	\item 
 	\item 
 	\item
 	\item 
 	\item 
 	\item 
 	\item 
 \end{itemize}
 
 \newpage			
 \section{《 围城 》 }  			
 \paragraph{1.方鸿渐}
 \paragraph{2.苏小姐}
 \paragraph{3.唐晓芙}
 
 \newpage			
 \section{《 一个人的朝圣 》-蕾秋·乔伊斯 }
 \subsection{故事}  			
 \paragraph{1.哈罗德}一个人的旅行给的是对陌生环境还是对自己的发现,到头来是-在那个陌生的地点对自己的反思与尘寓.
 
 宠辱不惊,闲看庭前花开花落。
 
 去留无意,漫观天上云卷云舒。
 
 \subsection*{句}	
 \begin{itemize}
 	\item 他望向窗外的花园,看到一个塑料袋挂在月桂篱上,在风中上下翻飞,却无法挣脱,获得自由
 	\item 我并不是说要……信教什么的。我的意思是,去接受一些你不了解的东西,去争取,去相信自己可以改变一些事情
 	\item 他思量着现在的情景:奎妮远在英格兰的那一头小睡,而他站在这一头的小电话亭里,两人之间隔着他毫不了解、只能想象的千山万水:道路、农田、森林、河流、旷野、荒原、高峰、深谷,还有数不清的人。他要去认识它们,穿过它们——没有深思熟虑,也无须理智思考,这个念头一出现,他就决定了。哈罗德不禁因为这种简单笑了
 	\item 哈罗德自己也承认有些地方计划得不够周详。他没有走远路的鞋子,没有指南针,更没有地图和换洗的衣服,整件事考虑得最少的就是旅途本身。本来他就是走起来之后才意识到自己要做什么,别说细枝末节了,就连大致的计划都没有。德文郡的路他还知道一点,但出去之后呢?反正一直往北走就是了。
 	\item 他们都相信他。他们都看见了他的帆船鞋,听过了他说的话,却用心说服了理性,选择忽略一切证据,去期待一种比不言自明的现实更大、更疯狂,也更美好的可能性
 	\item 他所要做的只是不停地把一只脚迈到另一只脚前面。这种简单令人高兴。只要一直往前,当然一定能抵达的。周围静止了,只有呼啸而过的车子轧过地上落叶的沙沙声不时打破这片宁静。这声音几乎让他以为自己又回到了海边。哈罗德突然发现自己已经深深陷入了变戏法一般纷纷浮现出来的回忆。
 	\item 只不过是把一只脚放到另一只脚前面。但我一直很惊讶这些原本是本能的事情实际上做起来有多困难。
 	\item 他弓起双肩,更加用力地迈步,仿佛不仅仅是为了赶到奎妮身边,更是为了逃避自己
 	\item 或许人就是这样,越害怕什么,就越容易被什么吸引
 	\item 这种自由的感觉太珍贵了
 	
 \end{itemize}
 
 \newpage			
 \section{《 从你的全世界路过 》 -张嘉佳   }  	
 \subsection{图书篇}		
 \paragraph{1.猪头}
 \paragraph{2.许多的爱情}
 \paragraph{3.胡言与悦悦 与他的母亲}
 \paragraph{4.摆渡人(备胎)的爱情}
 \paragraph{5.那个叫大黑的野狗}
 
 \subsection{电影篇-戳不中女人G点,再爱也只能做个路人}
 \paragraph{1.陈末-小容}
 很多男生都问过我一个问题:“为什么我倾尽全力也撩不到的妹子,别的男生一招手她就投怀送抱了呢?”
 
 这其实是很多人心中的隐痛。
 
 青春期的时候,爱一个姑娘,拼命讨好她,悄悄往她课桌里塞牛奶、巧克力,写甜得掉牙的情书,你连自己都感动得稀里哗啦,而她却无动于衷,她把牛奶丢进垃圾桶,情书贴到了学校的公告栏……\textbf{你以为她是不想谈恋爱},可你一回头,却发现她在校外的梧桐树下跟一个混混头子忘情地接吻……
 
 \textbf{很多时候,男人都很笨,以为掏心掏肺对一个姑娘,就能俘获她的芳心。}
 
 \textit{但大多时候,你却发现,你的全世界她都不想要,而别人一个眼神,都让她为之神魂颠倒。}
 你以为给得不够多,自己不够好,其实你错了,你只是用尽了全身力量,却没有用对地方,没戳中她的G点而已。
 
 武学上说,制敌要打命门,不然打遍全身也是白费力气。
 俗语说,打蛇要打七寸。
 \textbf{撩妹也是一样,方法比行动重要。}
 
 
 《从你的全世界路过》里,杜鹃饰演的小容姐说:\textbf{“你们根本不知道女人需要什么,你觉得相爱很重要,而我更看重合不合适。”}
 这就是她和陈末分手的理由。
 
 此时陈末内心的OS大概是这样的\textbf{:谁特么知道你想要什么啊,你想要你就说啊,你不说,谁知道你想要呢?}
 
 \textbf{但对于女生来说,往往是这样:我想要,你明白,你给了,我会加倍开心,因为心有灵犀。我想要的,我说出来,你给了,那跟没给没什么两样。}
 
 很矛盾对吧?每个女生都是哲学家,矛盾统一!
 
 其实只要陈末走点心,就不难发现小容要什么。
 
 小容恐怕是《从你的全世界路过》这部影片里最容易搞懂的姑娘了。
 
 她反复跟陈末强调,\textbf{你要有点上进心,因为你如果没有上进心,我要的东西,你不光现在给不了,以后也给不了。她要的,是名、利,以及“更高更远的天空”}。
 
 而分手后,陈末给的,是散漫、矫情、时不时制造的小难堪小尴尬。
 
 陈末是很爱小容,当小容出事儿时,他拿出了自己的全部家当给了她。但是有什么用呢?这样的女人,不是一张卡就能摆平。
 
 所以,在小容的世界里,他只能当一个路人。
 
 \paragraph{2.猪头}
 相比于陈末,他的兄弟猪头就更惨了。
 
 如果说陈末是浪荡子的代表,那猪头就是死心眼星人代表。\textbf{他是真正的情种,他把自己所有的钱都给了自己喜欢的燕子}。还向朋友们借钱付了房子的首付,打算给燕子一个家。\textbf{可是燕子千里迢迢从国外回来,在求婚现场对他说的话却是:我们分手吧。我觉得分手这种事儿在电话里说不太好,要当面说}。
 
 说你妹啊,早知道要分手。买什么房,办什么同学趴,求什么婚啊?想说就在电话里说啊!
 
 可是猪头却一句抱怨也没有,哭着和燕子送别。燕子说:“这些年你给我的那些钱,等我赚到钱就还给你。”标准的绿茶婊啊!
 
 猪头仿佛也瞬间彻悟了:我对她好,其实是我不好,她对我不好,其实是对我好。
 
 真想冲进去一巴掌把这个憨包给扇醒过来:\textbf{她要是真对你好,从一开始,就不该给你机会。你信任她没有偷钱,信任她可以和你白头偕老,但她却把你当做一块向上攀爬的垫脚石而已。}
 
 这段感情里,无法得知柳岩饰演的燕子究竟想要什么。但可以知道的是,她想要的,猪头一定给不了。万一,是颜值呢?
 
 \paragraph{3.荔枝和茅十八}
 第三段感情,是荔枝和茅十八,荔枝对茅十八,上演了一段倒追戏码。荔枝追茅十八,是真的很用心在追哦,他们俩感情的前半段,就是荔枝各种追茅十八的桥段。“如果你追到我,我就让你嘿嘿嘿”,幸运的是,荔枝追上了茅十八。而后半段基本就是撒狗粮,茅十八各种给荔枝制造导航,各种秀恩爱,还来了一出浪漫求婚。	
 但是正验证了那句话——秀恩爱死得快。在一次流氓报复的意外中,本来可以跟荔枝过着幸福快乐的日子的茅十八,为荔枝英勇献身了。
 
 这是一对能互相戳中G点的男女,但是,却以遗憾收场
 
 其实,这部影片里最完满的感情,应该是幺鸡和陈末。
 
 \textbf{几年前,一个雨夜,幺鸡生日,孤独地站在大雨中,给陈末所在的电台打了个电话,向陈末倾诉孤独的心境。陈末的一席话,就戳中了她的G点。于是她拼命考进了陈末的电台,成为了他的学生。}
 
 \textbf{幺鸡对陈末也是像猪头一样无怨无悔的付出,陪他去稻城散心,两人还一起躺然后还为陈末策划了一场“给小容”的告白……然后,就默默地走开了,也许以为,只要你幸福就好。}
 
 她的一片赤诚,终究打动了陈末,几年后,在稻城,陈末一回头,发现身后站着消失了好久的幺鸡,这似乎预示着他们美好的结局
 
 \paragraph{总结}
 《从你的全世界路过》这部电影里爱情,是那种太过完美、毫无瑕疵的爱情,轰轰烈烈,撕心裂肺,很具有代表性,就像童话故事一般。看了这部电影,你会感叹:这TM的才叫谈恋爱啊!
 
 可是你绝对不想要这样的爱情,太作,太累,太折腾……
 
 虽然影片遭遇了很多吐槽,但是如果你带入自己的思考,就会发现,在几段感情中,获得圆满结局的,都是彼此都用恰到好处的力量,戳到了对方的G点,所以得到了爱情的高潮。
 
 而只要那些用尽全力也戳不中G点的,最终成了路人甲乙丙丁。所以,爱一个人,花了多少力气不重要,戳中G点才重要。
 \subsection{句子}
 \begin{itemize}
 	\item 我打算在毕业前,偷满她五百二十个水瓶,她就知道这是520(我爱你)的意思了
 	\item 那时候,所有人不相信她,只有我相信她。所以,她也相信我
 	\item 十年醉了太多次,身边换了很多人,桌上换过很多菜,杯里洒过很多酒
 	\item 他不懂星座血型,但是他说,通过人的长相和姓名,基本就可以判断他的一生。   比如,人的相貌,会决定你从小周边的人对你是什么态度。   重眉的面相凶,少人亲近;方脸的面相正,易得信任;嘴大的大家喜欢觉得有趣可爱,常跟你开玩笑,于是活泼奔放;眼细的大家觉得你心机重,不会跟你聊太深,于是表里不一。你的长相决定了他人对你的态度,他人对你的态度决定了你的性格,你的性格决定了一生的路。
 	\item 世界上,总有一个人和你刚见面,两人就互相吸引,莫名觉得是一个整体。   这就是你的反向人。   世界上,总有一个人和你刚见面,两人就互相吸引,莫名觉得是一个整体
 	\item 其实有满腹话要说,可对面已经不是该说的人
 	\item 我喜欢牵着父母的手一起走路,不管是在哪里
 	\item 我希望有个如你一般的人。这世界有人的爱情如山间清爽的风,有人的爱情如古城温暖的阳光。但没关系,最后是你就好。
 	\item 因为自己挣来的,更可贵的是你获得它的能力。而从他人处攫来的,你会恐惧失去,一心想要牢牢把握在手中。
 	\item 所以,虾子要吃活着烧的,痛出来的鲜美,才足够颠倒众生
 	\item 接着发现,描绘只能靠经历来解决。\textbf{很多情况的表达方式是一样的,只有细微的差别,没有经历过,就无法陈述出不同}
 	\item 每个人有自己的表达方式,如果你不喜欢,只能说明不是为你准备的
 	\item 雨过天晴,终要好天气。世间予我千万种满心欢喜,沿途逐枝怒放,全部遗漏都不要紧,得你一枝配我胸襟就好。
 	\item 美食和风景的意义,不是逃避,不是躲藏,不是获取,不是记录,而是在想象之外的环境里,去改变自己的世界观,从此慢慢改变心中真正觉得重要的东西
 	\item 就算过几天就得回去,依旧上班,依旧吵闹,依旧心烦,可是我对世界有了新的看法。   就算什么改变都没有发生,至少,人生就像一本书,我的这本也比别人多了几张彩页。   这就是旅行的意义。
 	\item 不需要倾诉,不需要安慰,不需要批判,不需要声讨,独自做回顾。   朋友不能陪你看完,但会在门口等你散场,然后傻笑着去新的地方。   再难过,有好基友陪在身边,就可以顺利逃亡。
 	\item 钱花完可以再赚,吃亏了可以再来,年轻没了怎么办?
 	\item 我年纪大了,本来想你结婚后,每天包粽子给你们小两口吃。吃到你们腻了,我也可以走了。你是我儿子,走错路不怕,走错就回家,你妈我一时半会儿死不了,回来的时候我在家。
 	\item 亲爱的刘雪同志,我很喜欢你,我已经跟领导申请过了,我要调到南京来。他们没同意,所以我辞职了。现在档案怎么移交我还没想好,所以,请你做好在南京接待我的准备。   亲爱的刘雪同志,我不会说话,但我有句心里话要告诉你。   我想和你生活在一起,永远。
 	\item 我说的拼命,是真的今天就算死了,我也愿意。
 	\item 小玉慢慢抬起手,地面上她的影子也抬起手。她微笑着,让自己的影子抱住了马力的影子
 	\item 她要走了,只能抱抱他的影子。可能这是他们唯一一次隆重的拥抱。白天你的影子都在自己身旁,晚上你的影子就变成夜,包裹我的睡眠。
 	\item 世事如书,我偏爱你这一句,愿做个逗号,待在你脚边。
 	\item 现在我特别后悔小时候没学点儿乐器。一个人坐在海边,如果你会弹吉他,或者会吹口琴,那就能独自坐一天。因为可以在最美的地方,创造一个完全属于自己的世界。”   
 	
 	她停顿一下,说:“不过我发现即使自己什么都不会,也能在海边,听着浪潮,看着篝火,创造一个完全属于自己的世界。那,我有回忆
 	\item “女人发发牢骚,其实不用你来装牛逼分析,只是要你的安慰。”
 	\item “女人是情绪的,感性的,别用逻辑来框死我们。”
 	\item “一句话,女人不在乎对错,在乎你的态度。”
 	\item 他想找你。狗一辈子就认一个主人,要是方便,姑娘,你就带着他吧
 	\item 一趟旅途最深刻的,反而是这些哭笑不得的片段,他们也许就是人生旅途中那些辉煌的山寨景点
 	\item 既然我们相爱,就一定要在一起。   什么都可以放弃,一定却一定不能放弃。
 	\item 人没有留住,至少能留住那味道
 	\item 他是带着思念去的,一个人的旅途,两个人的温度,无论到哪里,都是在等她。那么,也许并不需要其他人打扰
 	\item 我觉得这个世界美好无比。晴时满树花开,雨天一湖涟漪,阳光席卷城市,微风穿越指间,入夜每个电台播放的情歌,沿途每条山路铺开的影子,全部是你不经意写的一字一句,留我年复一年朗读。这世界是你的遗嘱,而我是你唯一的遗物。
 	\item 而在人生中,因为我一定会喜欢你,所以真的有些道路是要跪着走完的,就为了坚持说,我喜欢你。
 	\item 看着孤独的日,守着暗淡的夜,并且要以岁月为马,奔腾到彼岸,找到和你周长、角度、裂口都相互衔接的故事。然后捧着书籍,晒着月光,心想:做怎样的跋山涉水,等怎样的蹉跎时光,都不重要,重要的是对面有谁在等你。
 	\item 分母那么浩瀚,分子那么微弱。唯一就等于没有
 	\item 原本你是想去找一个人的影子,在歌曲的间奏里,在无限的广阔里,在四季的缝隙里,在城市的黄昏里。结果脚印越来越远,河岸越来越近,然后看到,那些时刻在记忆中闪烁的影子,其实是自己的。
 	\item 照顾好自己,爱自己才能爱好别人。如果你压抑,痛苦,忧伤,不自由,又怎么可能在心里腾出温暖的房间,让重要的人住在里面。如果一颗心千疮百孔,住在里面的人就会被雨水打湿。
 \end{itemize}			
 \newpage			
 \section{《 学会提问 》 }  
 
 \newpage			
 \section{《 追风筝的人 》 } 
 \begin{itemize}
 	\item 许多年过去了,人们说陈年旧事可以被埋葬,然而我终于明白这是错的,因为往事会自行爬上来。
 	\item 阿里转过身,看到我正学着他。他什么也没说。当时没说,以后也一直没说,他只是继续走。
 	\item 伤口很痛,几个星期都好不了,但我毫不在意。我们的冬天总是那样匆匆来了又走,伤疤提醒我们怀念那个最令人喜爱的季节
 	\item 一年某个邻居的小孩爬上松树,去捡风筝,结果树枝不堪重负,他从三十英尺高的地方跌下来,摔得再也无法行走,但他跌下来时手里还抓着那只风筝。如果追风筝的人手里拿着风筝,没有人能将它拿走。这不是规则,而是风俗。
 	\item “为你,千千万万遍!”
 	\item 我仍有最后的机会可以作决定,一个决定我将成为何等人物的最后机会。我可以冲进小巷,为哈桑挺身而出——就像他过去无数次为我挺身而出那样——接受一切可能发生在我身上的后果。或者我可以跑开。
 	\item 我最怕看到的:真诚的奉献。所有这些里,那是我最不愿看到的
 	\item 我看着爸爸的轿车驶离路边,带走那个人,那个平生说出的第一个字是我名字的人。我最后一次模糊地瞥见哈桑,他瘫坐在后座,接着爸爸转过街角,那个我们曾无数次玩弹珠的地方。
 \end{itemize}
 \newpage			
 \section{《 活着 》 -  余华 } 
 
 \newpage
 \section{<爱> - 张爱玲}
 这是真的。
 
 有个村庄的小康之家的女孩子,生得美,有许多人来做媒,但都没有说成。那年她不过十五六岁吧,是春天的晚上,她立在后门口,手扶着桃树。她记得她穿的是一件月白的衫子。对门住的年轻人,同她见过面,可是从来没有打过招呼的,他走了过来。离得不远,站定了,轻轻的说了一声:“噢,你也在这里吗?”她没有说什么,他也没有再说什么,站了一会,各自走开了。
 
 就这样就完了。
 
 后来这女人被亲眷拐了,卖到他乡外县去作妾,又几次三番地被转卖,经过无数的惊险的风波,老了的时候她还记得从前那一回事,常常说起,在那春天的晚上,在后门口的桃树下,那年青人。
 
 \textbf{于千万人之中遇见你所要遇见的人,于千万年之中,时间的无涯的荒野里,没有早一步,也没有晚一步,刚巧赶上了,那也没有别的话可说,惟有轻轻地问一声:“噢,你也在这里吗?”}
 
 \newpage
 \section{《 我们仨 》 - 杨绛}
 
 \newpage
 \section{<一只特立独行的猪> - 王小波}
 是人变了,还是猪变了,或者是那些势利小人改变了一切..
 
 \newpage
 \section{<一碗清汤荞麦面> - 栗良平}
 满满的感动, 这个北海平的面馆给人的鼓励,这两懂事的孩子,还有那不一样的母亲..
 
 我反正是听哭了..
 
 \paragraph{原文}
 对于面馆来说,最忙的时候,要算是大年夜了。北海亭面馆的这一天,也是从早就忙得不亦乐乎。
 
 平时直到深夜12点还很热闹的大街,大年夜晚上一过10点,就很宁静了。北海亭面馆的顾客,此时也像是突然都失踪了似的。
 
 就在最后一位顾客出了门,店主要说关门打烊的时候,店门被咯吱咯吱地拉开了。一个女人带着两个孩子走了进来。6岁和10岁左右的两个男孩子,一身崭新的运动服。女人却穿着不合时令的斜格子短大衣。
 
 “欢迎光临!”老板娘上前去招呼。
 
 “啊……清汤荞麦面……一碗……可以吗?”女人怯生生地问。那两个小男孩躲在妈妈的身后,也怯生生地望着老板娘。
 
 “行啊,请,请这边坐。”老板娘说着,领他们母子三人坐到靠近暖气的二号桌,一边向柜台里面喊着,“清汤荞麦面一碗!”
 
 听到喊声的老板,抬头瞥了他们三人一眼,应声回答道:“好咧!清汤荞麦面一碗——”
 
 \textbf{案板上早就准备好了面条,一堆堆像小山,一堆是一人份。老板抓起一堆面,继而又加了半堆,一起放进锅里。老板娘立刻领悟到,这是丈夫特意多给这母子三人的。}
 
 热腾腾香喷喷的清汤荞麦面一上桌,母子三人立即围着这碗面,头碰头地吃了起来。
 
 “真好吃啊!”哥哥说。
 
 “妈妈也吃呀!”弟弟夹了一筷子面,送到妈妈口中。
 
 不一会,面吃完了,付了150元钱。
 
 “承蒙款待。”母子三人一起点头谢过,出了店门。
 
 “谢谢,祝你们过个好年!”老板和老板娘应声答道。
 
 过了新年的北海亭面馆,每天照样忙忙碌碌。一年很快过去了,转眼又是大年夜。
 
 和以前的大年夜一样,忙得不亦乐乎的这一天就要结束了。过了晚上10点,正想打烊,店门又被拉开了,一个女人带着两个男孩走了进来。
 
 老板娘看那女人身上那件不合时令的斜格子短大衣,就想起去年大年夜最后那三位顾客。
 
 \textbf{  “……这个……清汤荞麦面一碗……可以吗?”}
 
 \textbf{ “请,请到里边坐,”老板娘又将他们带到去年的那张二号桌,“清汤荞麦面一碗——”“好咧,清汤荞麦面一碗——”老板应声回答着,并将已经熄灭的炉火重新点燃起来。}
 
 \textbf{“喂,孩子他爹,给他们下三碗,好吗?”}
 
 老板娘在老板耳边轻声说道。
 
 \textbf{  “不行,如果这样的话,他们也许会尴尬的。”}
 
 老板说着,抓了一份半的面下了锅。
 
 桌上放着一碗清汤荞麦面,母子三人边吃边谈着,柜台里的老板和老板娘也能听到他们的声音。
 
 “真好吃……”
 
 “今年又能吃到北海亭的清汤荞麦面了。”
 
 “明年还能来吃就好了……”
 
 吃完后,付了150元钱。老板娘对着他们的背影说道:“谢谢,祝你们过个好年!”
 
 这一天,被这句说过几十遍乃至几百遍的祝福送走了。
 
 生意日渐兴隆的北海亭面馆,又迎来了第三个大年夜。
 
 从九点半开始,老板和老板娘虽然谁都没说什么,但都显得有点心神不定。十点刚过,雇工们下班走了,老板和老板娘立刻把墙上挂着的各种面的价格牌一一翻了过来,赶紧写好“清汤荞麦面150元”。其实,从当年夏天起,随着物价的上涨,清汤荞麦面的价格已经是200元一碗了。
 
 二号桌上,在30分钟以前,老板娘就已经摆好了“预约”的牌子。
 
 到十点半,店里已经没有客人了,但老板和老板娘还在等候着那母子三人的到来。他们来了。哥哥穿着中学生的制服,弟弟穿着去年哥哥穿的那件略有些大的旧衣服,兄弟二人都长大了,有点认不出来了。母亲还是穿着那件不合时令的有些退色的短大衣。
 
 “欢迎光临。”老板娘笑着迎上前去。
 
 “……啊……清汤荞麦面两碗……可以吗?”母亲怯生生地问。
 
 “行,请,请里边坐!”
 
 老板娘把他们领到二号桌,顺手将桌上那块预约牌藏了起来,对柜台喊道:
 
 “清汤荞麦面两碗!”
 
 “好咧,清汤荞麦面两碗——”
 
 老板应声答道,把三碗面的分量放进锅里。
 
 母子三人吃着两碗清汤荞麦面,说着,笑着。
 
 “大儿,淳儿,今天,妈妈我想要向你们道谢。”
 
 “道谢?向我们?……为什么?”
 
 “你们也知道,你们的父亲死于交通事故,生前欠下了八个人的钱。我把抚恤金全部还了债,还不够的部分,就每月五万元分期偿还。”
 
 “是呀,这些我们都知道。”
 
 老板和老板娘在柜台里,一动不动地凝神听着。
 
 “剩下的债,本来约定到明年三月还清,可实际上,今天就可以全部还清了。”
 
 “啊,这是真的吗,妈妈?”
 
 “是真的。大儿每天送报支持我,淳儿每天买菜烧饭帮我忙,所以我能够安心工作。因为我努力工作,得到了公司的特别津贴,所以现在能够全部还清债款。”
 
 “好啊!妈妈,哥哥,从现在起,每天烧饭的事还是包给我了!”
 
 “我也继续送报。弟弟,我们一起努力吧!”
 
 “谢谢,真是谢……谢……”
 
 “我和弟弟也有一件事瞒着妈妈,今天可以说了。那是在十一月的一个星期天,我到弟弟学校去参加家长会。那时,弟弟已经藏了一封老师给他*的信……弟弟写的作文如果被选为北海道的代表,就能参加全国的作文比赛。正因为这样,家长会的那天,老师要弟弟自己朗读这篇作文。老师的信如果给妈妈看了,妈妈一定会向公司请假,去听弟弟朗读作文,于是,弟弟就没有把这封信交给妈妈。这事,我还是从弟弟的朋友那里听来的。所以,家长会那天,是我去了。”
 
 “哦,是这样……那后来呢?”
 
 \textit{ “老师出的作文题目是,‘你将来想成为怎样的人’。全体学生都写了,弟弟的题目是《一碗清汤荞麦面》,一听这题目,我就知道写的是北海亭面馆的事。当时我就想,弟弟这家伙,怎么把这种难为情的事都写出来了。}
 
 \textbf{“作文写的是,父亲死于交通事故,留下一大笔债。妈妈每天从早到晚拼命工作,我去送早报和晚报……弟弟全写了出来。接着又写,十二月三十一日的晚上,母子三人吃一碗清汤荞麦面,非常好吃……三个人只买一碗清汤荞麦面,面馆的叔叔阿姨还是很热情地接待我们,谢谢我们,还祝福我们过个好年。在弟弟听来,那祝福的声音分明是在对他说:不要低头!加油啊!要好好活着!因此,弟弟长大成人后,想开一家日本第一的面馆,也要对顾客说:‘加油啊!’‘祝你幸福!’ ‘谢谢!’弟弟大声地朗读着作文……”}
 
 此刻,柜台里竖着耳朵,全神贯注听母子三人说话的老板和老板娘不见了。在柜台后面,只见他们两人面对面地蹲着,一条毛巾,各执一端,正在擦着夺眶而出的眼泪。
 
 “作文朗读完后,老师说:‘今天淳君的哥哥代替他母亲来参加我们的家长会,现在我们请他来说几句话……’”
 
 “这时哥哥都说了些什么?”
 
 “因为突然被叫上去发言,一开始,我什么也说不出……‘大家一直和我弟弟很要好,在此,我谢谢大家。弟弟每天要做晚饭,只能放弃兴趣小组的活动,中途回家,我做哥哥的,感到很难为情。刚才,弟弟刚开始朗读《一碗清汤荞麦面》的时候,我感到很丢脸,但是,当我看到弟弟激动地大声朗读的样子,我心里更感到羞愧。这时我想,决不能忘记妈妈买一碗清汤荞麦面的勇气。我们兄弟二人一定要齐心协力,照顾好我们的妈妈!希望大家以后也能够和我弟弟做好朋友。’我就说了这些……”
 
 母子三人,静静地,互相握着手,良久。继而又欢快地笑了起来。和去年相比,像是完全变了个模样。
 
 作为年夜饭的清汤荞麦面吃完了,付了300元。
 
 “承蒙款待。”母子三人深深地低头道谢,走出了店门。
 
 “谢谢,祝你们过个好年!”
 
 老板和老板娘大声向他们祝福,目送他们远去……
 
 又是一年的大年夜降临了。北海亭面馆里,晚上九点一过,二号桌上又摆上了’预约席“的牌子,等待着母子三人的到来。可是,没看到那三人的身影。
 
 一年,又是一年,二号桌始终默默地等待着。可母子三人还是没有出现。  北海亭面馆因为生意越来越兴隆,店内重又进行了装修。桌子、椅子都有换了机关报的。可二号桌却仍然好故。老板夫妇不但没感到不协调,反而把二
 
 号桌安放在店堂中央。
 
 “为什么把这张旧桌子放在店堂中央?”有的顾客感到奇怪。
 
 于是,老板夫妇就把“一碗阳春面”的事告诉他们。并说,看到这张桌子,就是对自己的激励。而且说不定哪天那母子三人还会来,这个时候,想用这张桌子来迎接他们。
 
 就这样,关于二号桌的故事,使二号桌成了“幸福的桌子”。顾客们到处传诵着。有人特意从远方赶来。有女学生,也有年轻的情侣,都要到二号桌上吃一碗阳春面。二号桌也因此而名声大振。
 
 
 
 时光流逝,年复一年。这一年的大年夜又来到了。
 
 这时,北海亭面馆已经是同一条街的商店会的主要成员。大年夜这天,亲如家人的朋友、近邻、同行,结束了一天的工作后,都来到了北海亭。在北海亭吃了过年面,听着除夕夜的钟声,然后亲朋好友聚集起来,一起到附近的神社去烧香磕头,以求神明保佑在新的一年里万事如意,厄除运开。这种情形,已经有五六年的历史了。
 
 今年的大年夜当然也不例外。九点半一过,以鱼店老板夫妇双手捧着装满生鱼片的大盆子进来为信号,平时亲如家人的朋友们大约三十多人,也都带着酒菜,陆陆续续地会集到北海亭,店里的气氛,一下子热闹起来。
 
 知道二号桌由来的朋友们,嘴里虽然没说什么,可心里都有在想着,今年二号桌也许又要空等了吧。那块“预约席”的牌子,早已悄悄地站在二号桌上。
 
 狭窄的座席之间,客人们一点一点地移动着身子坐下,有人还招呼着迟到的朋友。吃着面,喝着酒,互相夹着菜。有人到柜台里去帮忙,有人随意拉开冰箱拿来东西。什么廉价出售的生意啦,海水浴的艳闻轶事啦,什么添了孙子的事啦。十点半时,北海亭里的热闹气氛到达了顶点。就在这时,店门被咯吱咯吱地拉开了。人们都向门口望去,屋子里突然静了下来。
 
 两位西装笔挺,手臂上搭着大衣的青年走了进来。这时,大伙都松了口气,随着轻轻的叹息声,店里又恢复了刚才的热闹。
 
 “真不凑巧,店里已经坐满了。”老板娘面带着歉意说。
 
 就在她拒绝两位青年的时候,一位身穿和服的妇人,深深低着头走了进来,站在两位青年的中间。
 
 店里的人们,一下子都屏住了呼吸,耳朵也竖起来了。
 
 “唔……三碗阳春面,可以吗?”穿和服的妇人平静地说。
 
 听了这话,老板娘的脸色一下子变了。十几年前留在脑海中的母子三人的印象,和眼前这三人的形象重叠起来了。
 
 老板娘指着三位来客,目光和正在柜台里找韭菜的丈夫的目光撞到一处。
 
 “啊!啊……孩子他爹!”
 
 面对不知所措的老板娘,青年中的一位开口了。
 
 “我们就是14年前的大年夜,母子三人共吃一碗阳春面的的顾客。那时,就是这一碗阳春面的鼓励,使我们三人同心合力,度过了艰难的岁月。这以后,我们搬到母亲的亲家滋贺县去了。”
 
 “我今年通过了医生的国家考试,现在京都的大学医院里当实习医生。明年四月,我将到札幌的综合医院工作。还没有开面馆的弟弟,现在京都银行里工作。我和弟弟商谈,计划了这生平第一次的奢侈的行动。就这样,今天我们
 
 母子三人,特意来拜访,想要麻烦你们烧三碗阳春面。”
 
 边听边点头的老板夫妇,泪珠一串串地掉下来。
 
 坐在靠近门口桌上的蔬菜店老板,嘴里含着一口面听着,直到这时,才把面咽下去,站起身来。
 
 “喂喂!老板娘,你呆站着干什么!这十年的每一个大年夜,你都为等待他们的到来而准备着,这十年后的预约席,不是吗?快!请他们上座,快!”
 
 被蔬菜店老板用肩一撞,老板娘这才清醒过来。
 
 “欢……欢迎,请,请坐……孩子他爹,二号桌阳春面三碗——”
 
 \textbf{ “好咧——阳春面三碗——”可泪流满面的丈夫却应不出声来。}
 
 店里,突然爆发出一阵欢呼声和鼓掌声。
 
 店外,刚才还在纷纷扬扬的飘着的雪,此刻也停了。皑皑白雪映着明净的窗子,那写着“北海亭”的布帘子,在正月的清风中,摇曳着,飘着……
 
 \newpage
 \section{《 天才在左疯子在右 》 - 高铭}
 \paragraph{<永远永远>}爱情最美的可能就是 最后的那句,永远永远,其实这可能就是-陪伴是最长的告白。
 
 指间的戒指不再闪亮,婚纱在衣柜早就尘封,我们的容颜都已慢慢的苍老,但那份心情却依旧没有改变,感谢你带给我的每一天,正是因为你,我才有勇气说-“永远永远”
 \newpage
 \section{《 知行合一,王阳明 》- 度阴山}
 \subsection{心学的问世}
 贵州的龙场(驿站)悟道
 
 \subsection{为什么是王阳明}
 \paragraph{1.虽小技、必有可观者焉}
 \paragraph{2.圣人}
 为天地立心、为生民立命、为往圣继绝学、为万世开太平
 \paragraph{3.教育为人}
 最好的教育是引导,不是你这种强制管束,你应该顺着孩子的习性去教育
 \paragraph{4.静}
 养生之诀,无过一静。老子清静,庄子逍遥。唯清静而后能逍遥也。
 \paragraph{5.智}
 在苦恼了一段时间后,他适时的转向、王阳明就是有这样一种本事:此路不通,掉头再寻找另外的路,绝不会在一条路上走到黑
 
 能勇敢向前是勇气、能转身是智慧、智勇兼备,才可成大事。	
 \paragraph{*6.朱熹理学-道出天机}
 虔诚的坚持唯一的志向,是读书之本,循序渐进,是读书的方法(\textbf{"居敬持志,为读书之本;循序致敬,为读书之法"})	
 
 王阳明像是被雷劈了一样,这句话恰好戳中了他多年来的毛病:\textbf{始终不能坚持唯一志向,而是在各个领域间跳来跳去,也没有循序渐进的去研究一个领域,所以什么成果都没有获得}。
 
 还有:他无法确定是朱熹错了,到底还是自己智慧不够..
 
 \paragraph{7.外力-机遇}
 每个大人物的成功都有一个外部环境,这个外部环境像运气一样,绝不可少。有的人在外部环境特别好的时候不需要过人的自身素质就能成功,比如官二代、富二代。而从来没有听说过拥有超级素质的人在没有外部环境的帮助下可以成功的。人类历史上怀才不遇的人多如过江之鲫。
 
 \subparagraph{站的角度不同,看到的情景就完全不一样}
 \subparagraph{人生一切所谓的苦难,都是比较而言}
 
 \subparagraph{竹子}
 竹子具备君子的四个特征,中空而静,通而有间,这是\textbf{君子之德};外节而实,一年四季枝叶颜色不改,这是\textbf{君子之操}; 随着天气而出而隐而明,适应性强,这是\textbf{君子“适应时势”的变通},挺然而立,不屈不挠,这是\textbf{君子之容}。
 
 \subparagraph{经历}
 任何一位伟大的圣贤都要经历过一番非比寻常的困苦环境。摩西被放逐渺无人迹的沙漠,才有了《摩西十戒》;耶稣在颠沛流离的传道中悟出大道;穆罕默德在放逐地创建了伊斯兰教; 释迦摩尼放弃了王子养尊处优的生活,到深山老林中读过艰苦的岁月,创建佛教。
 
 这几个人的成功似乎告诉了我们一个人生哲理:不经风雨,就不能见彩虹,逆境使人成长,让人成熟。
 
 \subparagraph{知行合一}
 “知行合一”实际上也是他心学“心即理”和“事上练”的延伸:天理既然都在我心中,那我唯一也必须要做的就是去实践来验证我心中的天理,而不是去外面寻找真理。
 
 \subparagraph{经验}
 任何人都看得出,他是真的以百姓心为己心。但我们应该知道,王阳明在此之前从未有过在基层工作的经验。按朱熹的说法,你没有工作经验,就不可能知道这份工作的道理,那你就无从下手。你必须先通过书本或者是前任的工作总结“格”出你工作的道理,才能胜任这份工作。 王阳明用事实反驳了朱熹,按王阳明心学的说法,天理就在我心中,我之前所以没有显露在基层工作的那些道理,是因为我没有碰到这个机会,现在我碰到了这个机会,那些道理就显现出来了,所以我不需要向外求取任何关于基层工作的道理。
 
 \textbf{这个道理是什么呢?其实就是用心,只要你用心为百姓好,就能想到为百姓做任何好事的道理,然后去做就是了..}
 
 \textbf{于事上的心念端正后},知识自然就能丰富;知识得以丰富,意念也就变得真诚;意念能够真诚,心情就会保持平正;心情能够平正,\textit{本身的行为就会合乎规范}。
 
 “我早已说过,年轻时涉世未深,内心浮躁,心不定就难成事。人非要经历一番不同平时的劫难才能脱胎换骨,成为真正能解决问题的人。”
 
 \subsection{王阳明如何做到知行合一之南赣(gan)剿匪}
 \newpage
 \section{<木兰花·拟古决绝词柬友,纳兰性德>}	
 事物的结果并不像人们最初想象的那样美好,在发展的过程中往往会变化得超出人们最初的理解,没有了刚刚认识的时候的美好、淡然。那么一切停留在初次的感觉多么美妙,当时的无所挂碍,无所牵绊,一切又是那么自然。初见时的美好,结局的超乎想象,勾绘的人生,总有那么几许淡淡的遗憾和哀伤
 
 人生若只如初见,
 
 何事秋风悲画扇。
 
 等闲变却故人心,
 
 却道故人心易变。
 
 骊山语罢清宵半,
 
 泪雨霖铃终不怨。
 
 何如薄幸锦衣郎,
 
 比翼连枝当日愿。
 
 \newpage
 \section{《 格局逆袭 》 -  万能的大熊 } 
 \paragraph{层次,格局}
 所谓层级就是看问题的高度吧,所谓格局就是眼界吧
 \paragraph{有才华的穷人-你是吗}
 你身边永远都不缺跟你讲道理的人,也永远不乏各种质疑你的人。 我发现失败者最喜欢做的事就是通过质疑别人的成绩是虚假的,来换取自己心灵上的安慰。 似乎这个世界上大家都没他明白,都没他能干,所以如果他没有赚钱,那么别人赚钱也是假的。这种人我就称之为-有才华的穷人。
 
 \newpage
 \section{<地毯的那一端,张晓风>}
 德:
 
 从疾风中走回来,觉得自己像是被浮起来了。山上的草香得那样浓,让我想到,要不是有这样猛烈的风,恐怕空气都会给香得凝冻起来!
 
 我昂首而行,黑暗中没有人能看见我的笑容。白色的芦荻在夜色中点染着凉意-这是深秋了,我们的日子在不知不觉中临近了。我遂觉得,我的心像一张新帆,其中每一个角落都被大风吹得那样饱满。
 
 \textbf{星斗清而亮,每一颗都低低地俯下头来。溪水流着,把灯影和星光都流乱了。我忽然感到一种幸福,那样浑沌而又陶然的幸福。我从来没有这样亲切地感受到造物的宠爱──真的,我们这样平庸,我总觉得幸福应该给予比我们更好的人。}
 
 但这是真实的,第一张贺卡已经放在我的案上了。洒满了细碎精致的透明亮片,灯光下展示着一个闪烁而又真实的梦境。画上的金钟摇荡,遥遥的传来美丽的回响。我彷佛能听见那悠扬的音韵,我彷佛能嗅到那沁人的玫瑰花香!而尤其让我神往的,是那几行可爱的祝词:"愿婚礼的记忆存至永远,愿你们的情爱与日俱增。"
 
 \textbf{是的,德,永远在增进,永远在更新,永远没有一个边儿和底儿──六年了,我们护守着这份情谊,使它依然焕发,依然鲜洁,正如别人所说的,我们是何等幸运。每次回顾我们的交往,我就彷佛走进博物馆的长廊。其间每一处景物都意味看一段美丽的回忆。每一件东西都牵扯着一个动人的故事。}
 
 那样久远的事了。刚认识你的那年才十七岁,一个多么容易错误的年纪!但是,我知道,我没有错。我生命中再没有一件决定比这项更正确了。前天,大伙儿一起吃饭,你笑着说:"我这个笨人,我这辈子只做了一件聪明的事。"你没有再说下去,妹妹却拍手起来,说:"我知道了!"啊,德,我能够快乐的说,我也知道。因为你做的那件聪明事,我也做了。
 
 那时候,大学生活刚刚展开在我面前。台北的寒风让我每日思念南部的家。在那小小的阁楼里,我呵着手,写蜡纸。在草木摇落的道路上,我独自骑车去上学。生活是那样的黯淡,心情是那样的沉重。\textit{在我的日记上有这样一句话:"我担心,我会冻死在这小楼上。"而这时候,你来了。你那种毫无企冀的友谊四面环护着我,让我的心触及最温柔的阳光。}
 
 我没有兄长,从小我也没有和男孩子同学过。但和你交往却是那样自然,和你谈话又是那样舒服。有时候,我想,如果我是男孩子多么好呢!我们可以一起去爬山,去泛舟。让小船在湖里任意飘荡,任意停泊,没有人会感到惊奇。好几年以后,我将这些想法告诉你,你微笑地注视着我:"那,我可不愿意,如果你真想做男孩子,我就做女孩。"而今,德,我没有变成男孩子,但我们可以去遨游,去做山和湖的梦。因为,我们将有更亲密的关系了。啊,想象中终生相爱相随该是多么美好!
 
 那时候,我们穿着学校规定的卡其服。我新烫的头发又总是被风刮得乱蓬蓬的。想起来,我总不明白你为什么那样喜欢接近我。那年大考的时候,我蜷曲在沙发里念书。你跑来,热心地为我讲解英文文法。好心的房东为我们送来一盘春卷,我慌乱极了,竟吃得洒了一裙子。你瞅着我说:"你真像我妹妹,他和你一样大。"我窘得不知如何是好,只是一径低着头,假作抖那长长的裙幅。
 
 那些日子真是冷极了。每逢没有课的下午我总是留在小楼上,弹弹风琴,把一本拜尔琴谱都快翻烂了。有一天你对我说:"我常在楼下听你弹琴。你好像常弹那首甜蜜的家庭。怎么?在想家吗?"我很感激你的窃听,唯有你了解、关切我凄楚的心情。德,那个时候,当你独自听着的时候,你想些什么呢?你想到有一天我们会组织一个家庭吗?你想到我们要用一生的时间以心灵的手指合奏这首歌吗?
 
 寒假过后,你把那迭泰戈尔诗集还给我。\textbf{你指着其中一行请我看:"如果你不能爱我,就请原谅我的痛苦吧!"我于是知道发生什么事了。我不希望这件事发生,我真的不希望。并非由于我厌恶你,乃是因为我太珍重这份素净的友谊,反而不希望有爱情去加深它的色彩。}
 
 但我却乐于和你继续交往。你总是给我一种安全稳妥的感觉。从起头,我就付给你我全部的信任。但是,当时我心中总响往着那种传奇式的、惊心动魄的恋爱。并且喜欢那么一点点的悲剧气氛。为着这些可笑的理由,我耽延着没有接受你的奉献。我奇怪你为什么仍作那样固执的等待。
 
 \textit{你那些小小的关怀常令我感动。那年圣诞节你把得来不易的几颗巧克力糖,全部拿来给我了。我爱吃笋豆里的笋子,唯有你注意到,并且耐心地为我挑出来。我常常不晓得照料自己,唯有你想到用自己的外衣披在我身上。(我至今不能忘记那衣服的温暖,它在我心中象征了许多意义。)是你,敦促我读书。是你,容忍我偶发的气性。是你,仔细纠正我写作的错误,是你,教导我为人的道理。如果说,我像你的妹妹,那是因为你太像我大哥的缘故。}
 
 后来,我们一起得到学校的工读金。分配给我们的是打扫教室的工作。每次你总强迫我放下扫帚,我便只好遥遥地站在教室的末端,看你奋力工作。在炎热的夏季里,你的汗水滴落在地上。我无言地站着,等你扫好了,我就去掸掸桌椅,并且帮你把它们排齐。\textbf{每次,当我们目光偶然相遇的时候,总感到那样兴奋。我们是这样地彼此了解,我们合作的时候总是那样完美。我注意到你手上的硬茧,它们把那虚幻的字眼十分具体地说明了。我们就在那飞扬的尘影中完成了大学课程──我们的经济从来没有富裕过;我们的日子却从来没有贫乏过。我们活在梦里,活在诗里,活在无穷无尽的彩色希望里。}记得有一次我提到玛格丽特公主在她婚礼中说的一句话:"世界上从来没有两个人像我们这样快乐过。"你毫不在意地说:"那是因为他们不认识我们的缘故。"我喜欢你的自豪,因为我也如此自豪着。
 
 我们终于毕业了,你在掌声中走上到台上,代表全系领取毕业证书。我的掌声也夹在众人之中,但我知道你听到了。在那美好的六月清晨,我的眼中噙着欣喜的泪。我感到那样骄傲,我第一次分沾你的成功,你的光荣。
 
 "我在台上偷眼看你,"你把系着彩带的文凭交给我,"要不是中国风俗如此,我一走下台来就要把它送到你面前去的。"
 
 我接过它,心里垂着沉甸甸的喜悦。你站在我面前,高昂而谦和、刚毅而温柔。我忽然发现,我关心你的成功,远远超过我自己的。
 
 那一年,你在军中。在那样忙碌的生活中,在那样辛苦的演习里,你却那样努力地准备研究所的考试。我知道,你是为谁而作的。在凄长的分别岁月里,我开始了解,存在于我们中间的是怎样一种感情。你来看我,把南部的冬阳全带来了。那厚呢的陆战队军服重新唤起我童年时期对于号角和战马的梦。我一直没有告诉你,当时你临别敬礼的镜头烙在我心上有多深。
 
 我帮着你搜集资料,把抄来的范文一篇篇断句、注释。我那样竭力地做,怀着无上的骄傲。这件事对我而言有太大的意义。这是第一次,我和你共赴一件事。所以当你把录取通知转寄给我的时候,我竟忍不住哭了。德,没有人经历过我们的奋斗,没有人像我们这样相期相勉,没有人多年来在冬夜图书馆的寒灯下彼此伴读。因此,也就没有人了解成功带给我们的兴奋。
 
 我们又可以见面了,能见到真真实实的你是多么幸福。我们又可以去作长长的散步,又可以蹲在旧书摊上享受一个闲散的黄昏。我永不能忘记那次去泛舟。回程的时候,忽然起了大风。小船在湖里直打转,你奋力摇橹,累得一身都汗湿了。
 
 "我们的道路也许就是这样吧!"我望着平静而险恶的湖面说:"也许我使你的负担更重了。"
 
 "我不在意,我高兴去搏斗!"你说得那样急切,使我不敢正视你的目光,"只要你肯在我的船上,晓风,你是我最甜蜜的负荷。"
 
 那天我们的船顺利地拢了岸。德,我忘了告诉你,我愿意留在你的船上,我乐于把舵手的位置留给你。没有人能给我像你给我的安全感。
 
 只是,人海茫茫,那里是我们共济的小舟呢?这两年来,为着成为的计划,我们劳累到几乎虐待自己的地步。每次,你快乐的笑容总鼓励着我。
 
 那天晚上你送我回宿舍,当我们迈上那斜斜的山坡,你忽然驻足说:"我在地毯的那一端等你!我等着你,晓风,直到你对我完全满意。"
 
 我抬起头来,长长的道路伸延着,如何圣坛前柔软的红毯。我迟疑了一下,便踏向前去。
 
 现在回想起来,已不记得当时是否是个月夜了,只觉得你诚挚的言词闪烁着,在我心中亮起一天星月的清辉。
 
 "就快了!"那以后你常乐观地对我说:"我们马上就可以有一个小小的家。你是那屋子的主人,你喜欢吧?"
 
 我喜欢的,德。我喜欢一间小小的陋屋。到天黑时分我便去拉上长长的落地窗帘,捻亮柔和的灯光,一同享受简单的晚餐。但是,哪里是我们的家呢?哪儿是我们自己的宅院呢?
 
 你借来一辆半旧的脚踏车,四处去打听出租的房子,每次你疲惫不堪的回来,我就感到一种痛楚。
 
 "没有合意的,"你失望地说:"而且太贵,明天我再去看。"
 
 我没有想到有那么多困难,我从不知道成家有那么多琐碎的事,但至终我们总算找到一栋小小的屋子了。有着窄窄的前庭,以及矮矮的榕树。朋友笑它小得像个巢,但我已经十分满意了。无论如何,我们有了可以憩息的地方。当你把钥匙给我的时候,那重量使我的手臂几乎为之下沈。它让我想起一首可爱的英文诗:"我是一个持家者吗?哦,是的。但不止,我还得持护着一颗心。"我知道,你交给我的钥匙也不止此数。你心灵中的每一个空间我都持有一枚钥匙,我都有权径行出入。
 
 亚寄来一卷录音带,隔着半个地球,他的祝福依然厚厚地绕着我。那样多好心的朋友来帮我们整理。擦窗子的,补纸门的,扫地的,挂画儿的,插花瓶的,拥拥熙熙地挤满了一屋子。我老觉得我们的小屋快要炸了,快要被澎湃的爱情和友谊撑破了。你觉得吗?他们全都兴奋着,我怎能不兴奋呢?我们将有一个出色的婚礼,一定的。
 
 这些日子我总是累着。去试礼服,去订鲜花,去买首饰,去选窗帘的颜。我的人像一座喷泉,在阳光下溢着七彩的水珠儿。各种奇特复杂的情绪使我眩昏。有时候我也分不清自己是在快乐还是在茫然,是在忧愁还是在兴奋。我眷恋着旧日的生活,它们是那样可爱。我将不再住在宿舍里,享受阳台上的落日。我将不再偎在母亲的身旁,听她长夜话家常。而前面的日子又是怎样的呢?德,我忽然觉得自己好像要被送到另一个境域里去了。那里的道路是我未走过的,那里的生活是我过不惯的,我怎能不惴惴然呢?\textbf{如果说有什么可以安慰我的,那就是:我知道你必定和我一同前去。}
 
 \textbf{冬天就来了,我们的婚礼在即。我喜欢选择这季节,好和你厮守一个长长的严冬。}我们屋角里不是放着一个小火炉吗?当寒流来时,我愿其中常闪耀着炭火的红光。我喜欢我们的日子从黯淡凛冽的季节开始,这样,明年的春花才对我们具有更美的意义。
 
 我即将走入礼堂,德,当结婚进行曲奏响的时候,父亲将挽着我,送我走到坛前,我的步履将凌过如梦如幻的花香。那时,你将怎样的微笑迎接我呢?
 
 \textbf{我们已有过长长的等候,现在只剩下最后的一段了。等待是美的,正如奋斗是美的一样,而今,铺满花瓣的红毯伸向两端,美丽的希冀盘旋而飞舞。我将去即你,和你同去采撷无穷的幸福。}当金钟轻摇,蜡炬燃起,我乐于走过众人去立下永恒的誓愿。因为,哦,德,\textbf{因为我知道,是谁,在地毯的那一端等我。}
 
 \newpage
 \section{<你没有权利评判我,冯小凯>}
 
 \textbf{你没权利评判我,你看到的只是我人生的某个片段。}
 
 你人为地用蒙太奇把我的人生剪辑成了你可以批评的样式,凭什么?
 
 不久前同学聚会,多年未见,大家都激动万分。
 
 一位女同学最后出现,场面有点像《虎妈猫爸》里董洁弹着吉他、穿着抹胸礼服从屏风后面惊艳亮相一样,矫情是有了点,但并不影响她展现出了和我们大多数人不一样的依然赤子青春的状态。寒暄问好,谈笑风生。
 
 可等这位同学走后,大家开始议论纷纷。“她怎么变这样了”“我不喜欢现在的她”“好假好装”“听说她哦……”
 
 我无法揣摩每一个人的心态,我只觉得可笑。十几年,每个人都在经历自己不同的人生,谁也不知道彼此遇到了谁,发生了什么,仅仅一两个小时的相见就可以窥探到别人的过往虚伪复杂甚至不堪,真是天赋异禀。
 
 曾经,我在商场看到一个六七岁模样的小男孩“嗖”地一下从我身边跑过,吓了我一大跳,然后他继续横冲直撞穿梭在人群里,他的妈妈在后面追,好不容易追上之后,一句也没批评他,反而搂在怀里不住地说:“乖,瞧这一头汗……”当时包括我在内的很多旁观者都向这位妈妈投去了不屑的目光,似乎想告诉她这样的溺爱终有你后悔的一天。
 
 后来我在饭店门口等座位的时候恰巧和这位妈妈挨着,再次目睹她如何纵容孩子的“顽劣”。我终于忍不住开口了:“您为什么不管一下您的孩子呢?”
 
 她愣愣地看着我,突然落下泪来。她告诉我,这个孩子患有一种先天性的神经系统疾病,无法控制自己的行为,甚至都没办法集中注意力听大人讲一句话,这是他们几个月来第一次出门……
 
 我为我提出了那样的质问和曾在心里轻易地论断过他们感到羞耻和内疚。以后,当看到很多局外人对别人如何养育孩子指手画脚的时候,若并不深入地了解别人的生活状况,我一定闭嘴。
 
 以前迷美剧,常看到两个人说着说着就来一句“Don’t judge me!”在他们的文化里\textbf{,即使是最好的朋友,哪怕是亲人,都没有资格去评判别人的对错,无论当面还是背后}。
 
 \textbf{我喜欢这种以尊重和爱护为前提的距离感。遇到事的时候,家人和朋友都可以给我建议,告诉我他们曾经的经验,奉劝我选错会承受的代价,但一旦我选了、做了,结果由我自己来承担,如果我不幸选错了,可以抚慰我,教育我,但不要对我做是非的判断。}
 
 我们不该评价,因为评价不好,我们不知道别人人生的360度角,我们只看到一个缝隙而已。我们评判不好又妄加评判,伤害了别人于自己又有什么好处?我们不能忍住自己去评判别人,必然也无法笑对别人对自己的评判。
 
 当然,善意的提醒和规劝与论断别人截然不同,我们都知道这边界在哪,苦口婆心和冷眼旁观,当事人感受得到那温度。
 
 其实,很多人的人生就浪费在评价别人身上。
 
 当你知道,你并没有被赋予对任何一个人进行论断性评价的权利的时候,你会发现突然有了更多的精力和时间去专注自己的事情,人缘也会变得特别好吧。
 
 当你知道,任何一个人也没有被赋予对你进行论断性评价的权利的时候,你便不会再去介意“别人”的话,生活也会更轻松吧。
 
 我们每个人都不可能百分之百去体会对方的处境。
 
 \textbf{即使你经历过类似的情况,即使你阅人无数、看尽人生繁华,他当下的感受和选择一定是原生家庭、教育背景、过往经历或是几十年错综复杂的原因促成的,那些不为人知的辛酸、无法言说的苦衷,你如何知晓?}
 
 所以,只要他没有触及法律,没有违反公共道德,没有侵犯他人利益,对与错,只有时间可以判断,只有上帝可以审判。
 
 \textbf{所以,那些常常挂在嘴边的狠狠的形容词就都收了吧,别人的形容词传到自己耳朵里就都忘了吧,以欣赏的眼光接受每一个生命的样态,没那么难。}
 
 
 \paragraph{罗胖曰}:
 
 1. 最坏的提问方式是——
 “关于什么,你的意见呢?”
 
 这意味着把对方从具体的处境中抽离出来,以旁观者的身份表达观点。
 
 2. 最好的提问方式是——
 “在那个处境下,你为什么那样做?”
 
 这意味着把对方还原到具体的处境中,以当事人的身份解释他的选择。
 
 \newpage
 \section{<父亲,散文网>}
 那日逛街,看到一个孩子坐在父亲肩头嬉闹,便想到了我和我的父亲。
 
 小的时候,\textbf{父亲是座山,坐在他的肩头,我会因这种宠爱心生得意,还会因那里能看到新奇的世界而兴奋不已。}慢慢长大,父亲时常会骑着28自行车载我出门,怕后座不安全,让我坐在车的前梁。我在他双臂环绕的怀里,驶向前方,也在一路旅程中看到了后座看不到的风景。或许是父亲从小给我的这种视野过于开阔,让我习惯了如今远离家乡,在外闯荡。但时至今日,那些年在父亲面前成长的日子,依然如昨日般历历在目。\textit{时间就是那么矫情,从不肯在美好的事物上多停留一秒钟。当你身在幸福却不知体会感知,也未曾想起扭头去看一眼父亲的脸,时间滴答一声悄然走过,等你想起好像有什么东西就要溜走,猛一抬头,便看到了父亲那张满是皱纹的脸。}
 
 从什么时候父亲开始老了?高中,大学,还是上次回家过年?\textbf{走过的时光从不以分秒计算,当我意识到父亲开始渐老的时候,仿佛真的就那么一瞬间,他的脸上就生出深深的皱纹,像春日耕田的地垄沟儿,肤色也如黄土般,粗粗嶙嶙。我惊叹时间在他脸上刻下的痕迹,也因我自己未曾仔细关注父亲而心生愧疚。他怎么就说老就老了呢?我真的想不通。}
 
 回想起我和父亲的林林总总,从出生到现在,他从来没有打过我,也没骂过我。这在我淘气的童年里可算得上是一件幸福的事。我想,每一对农村父母,都会将自己的一生全部搭在孩子身上。我出身农民家庭,父亲是地地道道的农民,他肯定懂得“棍棒出孝子”的道理,只不过他的爱,如今看来,隐忍善良又大气,很多年后我才恍悟,这是性格使然。我也还算争气,作为家里的老三,唯一的男孩,在父亲这种看似并不严格的教育下,虽然淘气,但并不张扬跋扈,不可一世。
 
 \textbf{父亲的爱,浓烈但不善表达。}
 
 记得七岁那年,我正淘的没边,夏日大雨刚过,村里小河如我般活泼不可耐,涨得湍急。看到村里半大的哥哥姐姐们撸起裤管下河摸泥鳅,我也按捺不住,跟着下了河,却一不小心,滑进了桥下面的水窝里。水流冲着我的脑袋,压的我抬不起头,我在慌乱中挣扎,却越陷越深。后来被人救起,迷迷糊糊中又被送到了奶奶家。刚被奶奶裹进被窝,就看到一个男人风驰电掣地跑进屋来,一抬头,是父亲。我突然哇的一下子哭了出来。从陷入漩涡到父亲来,我都没有掉一滴眼泪,可是看到父亲,气喘吁吁地站在我的面前,惊慌失措地望着我,我也不知怎的了,仿佛胸腔中灌进的不是水,而是这世界上最大的委屈一样,全都倾泻而出。父亲见我哭,更慌了神,裹紧被子,抱起我就回家。回家的路很短,但在我的记忆里,却走的很漫长。他全身都在抖,还抱我抱得紧,手上的力道也勒得我生疼。 
 
 \textbf{那时父亲怀里的我,尚不能学会去顾及父亲的感受,直到后来长大了,发生了一件事,才让我多多少少体会到,当年父亲怀里的自己,把他的心,勒的有多痛}。
 
 事情发生在高中的一天,母亲突然给我打电话,说父亲帮舅舅家翻修房子,不小心从梯子上摔下,背过气去。我的心当时就懵的一声,跌了下去,挂了电话就往家里赶,一路上手足无措,脑子也混乱无比,心里的惊慌与恐惧吞噬了我,我突然便想起了七岁夏日那条路上的父亲,他是怀着怎样的心情一路跑来找我,又是以怎样的心情把我抱回了家。他的颤抖和大力抱紧我的臂膀,一定也是因为恐惧。后来回到家看到父亲并无大碍,我的心瞬间就涌上了一种心酸,我想,随着我长大成熟,父亲变老,我会一点点地去承担那种心痛,一点点地接近那种恐惧,本该因成长脱离父亲独自有所担当,却因为体会了父爱而更加依赖和亲近父亲,这种难以言说的感情,终将愈来愈烈。
 
 转眼便到了高中毕业,选学校和专业的时候,我征求父亲的意见。父亲告诉我,让我自己拿主意。填志愿的前一晚,父亲很晚才回家,见我没睡,突然同我说道:我问你洪叔了,石油专业这几年都还可以,电力也是个不错的选择,然后给我说了很多关于石油和电力方面的就业前景。说罢他便离开了。我不觉莞尔,父亲让我自己做主,又希望我能一帆风顺,还不愿意强加给我意见。他的内心,这些天肯定比我还纠结和煎熬。
 
 我的洪叔是石油管道局的一名工人,也算是家里唯一有正式工作的人。父亲一个没读过几年书的人,也不知道是怎么将洪叔那些话一字不漏地记下来并转达给我。录取通知书发下来,我去异地求学,本来自己去报道即可,父亲却担心这担心那,非要我母亲同我一起。我便想借此机会也让父亲出来看看,他却以没有时间为托辞,不肯来。想起填报志愿那会儿的情景,便知当时父亲肯定也是心念着我,但碍于面子和各种家庭琐事,抽不开身,便不过来。适逢大学毕业,我再次央求父亲来学校,看看儿子四年生活学习的地方,可他还是那句话,没有时间。我不甘心,便以东西一个人带不回家为理由,硬是把父亲约到了学校。父亲来的那天,我把所有的事情都放到一边,专门陪着他们逛了我的校园。我们去了寝室,又去了学校图书馆、教学楼,还给他和母亲在学校的校牌前照相留念。那张照片,现在还在我的手机里,父亲的面容,看上去拘谨又欣慰。
 
 写到这里,突然想起关于父亲的好多优秀品质,像光一样,照着我。
 
 父亲没读过几年书,但印象中却把自己和家人的名字写的都很好看,虽不是什么书法墨画,但每个字看上去都苍劲有力,古朴厚实。或许这样形容有些夸张,但父亲的字在我眼中,就是这样子的,并且我觉得还很难得。现在想想,已多年不见他拿笔写字,一双手长满了老茧,我知道那是握多了锄头,握多了家里的担子,岁月在他手上打磨出的痕迹。我曾多次拿起笔,模仿记忆里的字迹,看看,再看看,从一开始连字体的样子都不见,到后来有了样子,但仍不见他字体的影子。况且上学工作,多年以来,我自己的字写的也越来越轻飘飘,有时候不认真便一划而过,有时候追求潇洒一挥而就,殊不知,这从来就是个追随者所为。在写字上,也未曾有个主心骨。父亲虽然受教育少,文化程度低,又不经常练字,但他懂得字如其人,每一个字,都虔诚又认真的去书写,因为这就是一个人处世态度的微缩影。我写不出,大概是因为我至今连韵味都未能完全体会,更找不到那种切入体肤的感受。
 
 父亲不醉酒,在男人的角度来讲,当真以身作则。但在我的印象里,也曾经喝多过一次,仅此一次。那年我还小,父亲酒后推门而入,浑身刺鼻的酒气,还没走到堂屋,便在院子里吐的死去活来,母亲在一旁照顾他。后来听母亲讲,是因为父亲做生意赔了几万块钱,亏了空,自然情郁于中,便发散于外,借酒消愁。那时亏了几万块钱,对一个农民家庭来说,不亚于晴天霹雳。他醉酒后的第二天,同往常一样,勤勤恳恳,踏踏实实干活,从此之后,直至今日,未再醉酒一次。
 
 父亲还做得一手好菜。我时常想,要是条件可以,他兴许是个不赖的大厨。父亲刀工漂亮,火候掌握的好,抖炒勺都不在话下。每次父亲掌勺,母亲便在一旁打下手,这厨房温馨的一幕和父亲的手艺,成就了我们家热气腾腾香喷喷的饭菜。饭余,父亲习惯抽支烟,但母亲在家里定了个硬规矩:抽烟可以,去外面。对于母亲的这个规矩,父亲至今都是个好的执行者。有时我放假回家,在屋里抽支烟,被母亲察觉,她因疼爱儿子并不说什么,反倒是父亲,为了母亲,也要求我去外面。
 
 毕业后,去单位报道那天,父亲送我到火车站,我很开心。\textbf{离别并没有想象的那么悲伤,反而因为我要去参加工作,心里有一种莫名的兴奋和些许期待。可能因为长期外地求学,我与父亲大概也习惯了这种离别和相聚。但我没有意识到的是,其实每一次的离别,都和以往不同。我们是父母的棉袄也好,破褂也罢,终归是难以脱下。辞别父亲后,甚至直到进站时我都未曾回头望他一眼。那日和父亲一起送我的姐姐后来告诉我,父亲在车站外望了你好久。听得别人讲起我未曾注意的、背影后的父亲,突然心里很难受。大概年轻人的每一次离别都是那么匆匆忙忙,更不知道,我们走后,父亲独自回家的路,又有多么的漫长。}
 
 拿到第一份薪水,给父亲买了一把刮胡刀。刮胡刀是手动的,因为父亲不喜欢用电动的,说胡子硬,电动的刮不干净。几个月后因事回家,发现刮胡刀依旧躺在崭新的盒子里,原封未动。我诧异地问父亲,“爸,为啥不用?”他答:“儿子买的太高级,用不惯。”看着他那把用了不知多少次的旧刀架,再看看旁边那把崭新的,就像他和我。我不知道,父亲会将它保存到什么时候,或许还没用上这把刮胡刀,便已逐渐老去。
 
 参加工作后,我常年漂泊在外,父亲虽想我想得紧,却从不会主动给我打电话,即便到了忍不住的时候,也只是对我的母亲念念叨叨:三儿几天没打电话了。大概是父子连心,每每父亲念叨我,我的电话便打了过去。母亲接了电话,总是先揶揄我们。可能是因为成长带给我和父亲男人间的尴尬,亦或是离家久了不知如何相处,我总不习惯和父亲长时间谈天,我们之间的谈话,总不过那么几句简短的问候。倒是母亲,占了电话大多数时间,而母亲所说的,也往往是电话那头父亲在旁边左一句右一句的问候和叮嘱,这时的母亲便起到了传话机的作用:你爸说你那冷不冷?你爸说让你出门注意,你爸说新闻里云南地震了你那没事吧?如此如此。
 
 \textbf{去年回家,看见父亲的头发又白了不少,牙齿也掉了两颗。}但精神矍铄,年近六十的他愈加显得小孩子气。晚上看电视,他会和母亲抢遥控器,还有一句没一句的逗我们开心,也会和我的外甥玩的不亦乐乎。这些都是我成长岁月里亲身经历但不曾感受到的,如今旁观者的我,突然觉得我的父亲、我、我的后辈,就像一个圆形跑道上的跑者。我看着我后辈们肆无忌惮地往前冲,就像当年父亲前面的我一样。跑得快,离着我的父亲也越来越远,终于有一天,你意识不到,人生便出现了一段缓慢的弯道,跑过弯道,我突然就从后面看到了我的父亲,开始有些遥远和模糊,后来越来越近,于是他在我眼前也越来越清晰。我看着父亲,似乎一不留神,儿子就奔向了远方,再也不需要他的呵护了。
 
 岁月从不肯厚待谁,也不肯薄待谁,他只管公平公正地走过每一秒,刻录平平凡凡的每一个日子。在这些流年岁月里,我的父亲,像中国世世代代的农民一样,脸朝黄土背朝天的养大了三个孩子,并成功扮演着儿子人生中的向导。他对子女的爱,因文化匮乏且不善于表达,成为一种不便言说又不言而喻的秘密。我们成长的岁月就是去感知这种力量,因为你一定相信,父母在,心就踏实很多,但是他在我们身边,又将继续离我们远去。那些我和我的父亲一路走来的历程,多年以后,再回首,不禁潸然泪下。\textbf{因为直到我从成年走向中年时,才恐惧到他们的渐行渐远,而我对他们的爱,却越来越浓。此时的我,多么想回到终点,看一看来时的风景。}
 
 \textbf{愿时光更慢一些,让我有时间,和父母面对着面,坐下来好好聊一聊天。在这个世界里,一起老去。}
 
 
 \newpage
 \section{<拥有风信子,是妈妈的幸福-刘继荣>}
 天晚欲雪,好友邀我去火锅城,说满腹心事要借火锅涮一涮。为着不肯做母亲,她与老公已成水火之势,欲借我这个过来人做灭火器,令我安置好女儿后速速赴约。
 
 当初她也极力劝过我,做母亲投资太多风险太大,如果生个神童还好,当妈的里子面子全赚足了,万一生个木头木脑的呆瓜,连自己的快乐都得赔进去,实在是亏大了。那时我笑她像个人贩子,现在却觉得她句句都是金玉良言。
 
 幼儿园门前熙熙攘攘,我牵着女儿的手,老师踌躇着,似有话要说。半晌,她微微叹道:“这孩子含羞草似的,音乐课嘴闭成一枚坚果,舞蹈课总比别人慢半拍,就连游戏时,也是独自在角落张望。
 
 我似乎感冒了,全身发冷,头痛欲裂。女儿将脸藏在我的大衣里,不安地蹭来蹭去,我愈发烦躁。一出世就得到病危通知的女儿,在这群活泼可爱的宝宝中间,不仅身高不足,性格也甚是木讷。
 
 老师斟酌再三,又说了一件愈发让我尴尬的事,女儿这些天用餐控制不住食量,常常吃到胃痛还要求添饭。旁边有位家长擦肩而过,他好奇地回过头,望望女儿,脸上的表情似笑非笑。我在老师面前兀自强撑着微笑,心里却暴躁得想找谁大吵一架。
 
 头晕目眩地到了家,一摊泥般软在床上。女儿推开门,期期艾艾地要我教她什么,我极力克制着恼怒,闭上眼睛不去睬她。可不一会儿,我刚昏昏欲睡,门又发出刺耳的吱呀声,她的脑袋在门边闪闪缩缩,心力交瘁的我终于爆发了,狂怒地指着她喊叫:“滚出去,我不想看见你!”
 
 女儿惊骇地缩到墙角,过了好一会儿,才瑟瑟发抖地问:“妈妈,一个人杀了自己的手,她会死吗?”我气急败坏地将她藏在背后的手拉出来,头立即嗡嗡作响,那么多的血,那么深的伤口!连淘气都笨得险些杀了自己,老天啊,你到底给了我一个什么样的孩子!
 
 我们跌跌撞撞地往医院走,雪大起来,女儿没有哭也没有要我抱,一声不响地在我身后紧追慢赶,看来她也知道自己闯了大祸。
 
 到了医院,医生说伤口太深,为防止感染,缝合后要输液,而且可能会留下永久性疤痕。好心的医生责备着我的疏忽,女儿默默听着,将瘦小的脸深深埋在膝间,长久地不肯抬起来。打上点滴后,女儿在病床上睡了,方想起好友之约,急急回电说明原因,她幽幽地说:“看来不要孩子是对的,太难了。”
 
 一句话触痛我所有的暗伤,泪猛然间决堤。这些年丈夫远在外地,我独自在病弱幼女和繁琐工作间奔走,巨大的压力几乎辗我为尘,皱纹天罗地网般自心底罩到面上。当初我认为孩子是上天赠送的最好礼物,现在才知道,这礼物有那么多教人承受不起的附加品。
 
 握着电话,忍不住向好友倾诉自己的委屈与懊恼,说到下午那位家长好奇的表情时,我已是泣不成声,好友连连劝我,说千万不能让孩子听到这些话。
 
 我回头看看女儿,她向里睡着,眼睫毛扑簌簌地抖,像蝴蝶湿了的翅膀。到家已经很晚,一进门就听见电话铃响,女儿轻手轻脚去了卧室。女儿的老师说,她今晚一直在给我打电话,如果打不通她会内疚得连觉也睡不着的。
 
 原来,那位听到我们谈话的家长去找了她。\textbf{他说他的孩子和我女儿最要好,那孩子告诉爸爸,好朋友拼命吃那么多饭,不是傻,也不是贪吃,是因为她妈妈工作很辛苦,她要吃得饱饱的就不会老是生病,会快快长高长聪明,会给妈妈做饭,帮妈妈拖地,妈妈就不会再烦了。}
 
 说着说着,老师忽然哽咽了,\textbf{她低声道:“您的孩子还说,妈妈最爱吃苹果,她一定要学会削苹果。”}
 
 我的心痉挛着,电光火石间忽然明白,她第一次进来,是想让我教她削苹果,我却没有睬她,她把自己伤得那么重,只是试图学着为我削一只苹果!
 
 我来到她的房间,她居然换上了夏天才穿的公主裙,默默站在红地毯上,似一个小小雪人,仿佛太阳一出即会融化。一见我,她眼里闪过浓浓的歉疚,一下子,我的鼻子酸起来。她喃喃地说妈妈别哭,我给你跳舞,跳我刚刚学会的《风信子开了》。
 
 我发现她右脚的袜子有些异样,她说,袜子破了一个洞,昨天脱掉鞋子进舞蹈教室时,有小朋友笑她露出的大脚趾,她便自己拿针线来缝,缝好后却成了一个小包。
 
 我蹲下来,摸着那个疙瘩,硬硬地硌着手,也硌着我的心。她的脚被磨了一整天,我却不知道,她只有四岁半,怕妈妈会烦,自己苦苦琢磨着,竟然补上了这个破洞,做妈妈的却嫌她笨!
 
 她轻轻唱着,缓缓摆动手臂,合拢的双手如一枚含羞紧闭的花苞。在灯光底下,花苞怯怯地打开,风来了,雨来了,她的单眼皮的黑眼睛一直看着我。她举在头顶的左手,还裹着厚厚的绷带,花瓣一点一点展开,女儿如同一个小小的勇敢的伤兵,在这个大雪纷飞的夜晚,终于将自己开成了一朵比雪还洁白的风信子。
 
 风信子低声说:“妈妈,小朋友都笑我开得太慢了。还有人说我是白痴。”我一震,心被烫了似地猛一缩。
 
 \textit{她顿了一下,静静地说:“舞蹈老师告诉大家,我不是白痴,我是白色的风信子,很安静很怕羞,比紫色、蓝色和红色的风信子要开得慢一些,可等到开好了会最美。”}
 
 全世界的雪都在瞬间融化,我的脸上溢过暖暖的柔波,我俯下身子,抱住她柔软的小身体,抱住漫漫红尘里离我最近的温暖。
 
 她伏在我的胸前,我看见窗外路灯暖暖的光里,映着一个纤尘不染的琉璃世界。温柔的屋檐上,慈爱的树枝间,静默的巷子里,每一处,都盛放着白色的风信子。每一粒种子,都拼尽气力,自九天深处赶来,匆匆赶赴一场花的盛会,从天上到人间,只为让自己那一颗小小的心,开出一树一树的繁华。
 
 我的心里是从来没有过的安然与甜蜜,我想告诉全世界的人:请允许白色的风信子害羞吧,因为,风雪再大,受伤再深,她都会拼尽全力为你开一朵最美的花。
 
 明天,我将告诉我的好朋友,拥有任何一朵风信子都是一件幸运的事。
 
 
 \newpage
 \section{<作为读者的谦虚-蒋方舟>}
 我在北京,目睹过很多场次的“作者见面会”,即使是比较小众和生僻的作者,也有人数多到超出预计的读者早早抢占了坐席,看来“吃到了鸡蛋,不必见下蛋的母鸡”的说法,并没有深入人心,人们依然还是要去听讲座——重点是看看那个作者,看他和自己想象中的那个人,吻合程度有多少。然后就到了提问的环节,一些人抓住了这个机会,开始大段大段地阐述自己的看法,最后以“你认为我说的对不对?”来结束提问——其实,这不是抓住机会,而是过度关注自我,忽视作者,浪费了这个机会。
 
 我读过一篇文章,是“水晶先生”写自己拜会晚年张爱玲的经历,那时张爱玲深居简出,不见朋友,更不见读者或粉丝,水晶先生幸运地得到见面的机会,他却浪费了这个机会。
 
 那是一次尴尬的拜会,也是一篇尴尬的文章。全篇都是水晶先生滔滔不绝地讲自己如何看待张爱玲的作品、如何看章回体小说、如何批评沈从文与钱钟书,然后张爱玲说:“嗳。”“很赞同。” 唯有一处,水晶先生说《金瓶梅》不好,而张爱玲很诧异,说自己每次读到宋蕙莲以及李瓶儿临终两段,都要大哭一场。
 
 水晶先生接下来又开始为自己辩护,坚持认为《金瓶梅》写得粗糙、单调而淫秽……如果水晶先生能够从绵延不绝的自我关注中抽出一两秒,观察张爱玲的反应,他是否会发现她的表情是在哂笑呢?
 
 我在年少无知、阅读甚少的时候,也是这样一个的读者。别人看动漫,看言情小说,我不屑,我找米兰·昆德拉、尼采来看,一方面为了接受采访时候能够引用他们的话;另一方面,也是抱着挑剔和反驳的目的,读一两段就在旁边标注:“写得也不怎么样。”“真的么?”“我看不懂,是他表达得不清楚?”
 
 直到我上高中的一个下午,\textbf{读到赫尔曼·黑塞的《荒原狼》},\textbf{其中有一段话“因为我跟你一样。因为我也和你一样孤独,和你一样不能爱生活,不能爱人,不能爱我自己,我不能严肃认真地对待生活,对待别人和自己。世上总有几个这样的人,他们对生活要求很高,对自己的愚蠢和粗野又不甘心。”}
 
 这段话穿透了纸张,穿越了时间和空间,准确地指向我的内心,让我看到一个我未曾发现过的自己。\textit{我才意识到,读书的目的不是为了求异,而是为了求同,我的幼稚和自大轰然崩塌,回归到一个读者的谦虚。}
 
 什么是一个读者的谦虚?中国古代私塾的教学方式,叫做“素读”,意思是看书的时候不带自己观点,脑袋空白地看。不在书本周围砌起预备的知识围墙,不做价值判断,不添油加醋,不预设任何目的。如同弗吉利亚·伍尔夫所说,理想的阅读是“不要向作者发号施令,而要设法变成作者自己。做他的合伙者和同伴。”
 
 \textbf{阅读,如同走进一座陌生的建筑,或是走向一个陌生的人。然后,等待。等待他走向你,与你共享他的人生。}如同《金瓶梅》中清河县城的李瓶儿准确地找到旧金山的张爱玲。
 
 \textbf{我们阅读,在他人的经验中找到自己的影子,发现一群像自己、但比自己更优秀的人组成的世界,他们四周是荒野,头顶是星辰。他们帮助我们抵抗脆弱的友谊、不完美的爱情、抵抗孤独引发的脆弱等一切打击,能够更轻盈更辽阔地生活着。}
 
 越来越多的人告诉我,读书这件事,最终会变得像采购一样——不需要自己亲自去实施,而有人替你完成。比如现在有很多渊博的人做这项工作,他们把一本书拆解、打烂、萃取、重塑,然后用几分钟的视频节目或是广播,把书中“有价值的内容”讲给你,就像电影预告片,把打斗、爆破、激情戏全部剪辑在一起,让你觉得看过“精华”之后,不再有必要看正片。
 
 \textit{而我将永远拒绝让人替我阅读,因为阅读是极个人化的,是可以提供给我的最大乐趣之一}。\textbf{书的本质,是孤独的作者与破碎的社会之间的一种交流方式,作者发出声响,或许几百年后,在青灯孤照的图书馆,一个孤独而谦虚的读者报以应和的回响}。
 
 \newpage
 \section{<假若我再上一次大学-季羡林> 《季羡林谈人生》}
 
 “假若我再上一次大学”,多少年来我曾反复思考过这个问题。我曾一度得到两个截然相反的答案:一个是最好不要再上大学,“知识越多越反动”,我实在心有余悸。一个是仍然要上,而且偏偏还要学现在学的这一套。后一个想法最终占了上风,一直到现在。
 
 我为什么还要上大学而又偏偏要学现在这一套呢?没有什么堂皇的理由。我只不过觉得,我走过的这一条道路,对己,对人,都还有点好处而已。我搞的这一套东西,对普通人来说,简直像天书,似乎无补于国计民生。然而世界上所有的科技先进国家,都有梵文、巴利文以及佛教经典的研究,而且取得了辉煌的成绩。这一套冷僻的东西与先进的科学技术之间,真似乎有某种联系。其中消息耐人寻味。
 
 我们不是提出了弘扬祖国优秀文化,发扬爱国主义吗?这一套天书确实能同这两句口号挂上钩。我举一个具体的例子。日本梵文研究的泰斗中村元博士在给我的散文集日译本《中国知识人の精神史》写的序中说到,中国的南亚研究原来是相当落后的。可是近几年来,突然出现了一批中年专家,写出了一些水平较高的作品,让日本学者有“攻其不备”之感。这是几句非常有意思的话。
 
 实际上,中国梵学学者同日本同行们的关系是十分友好的。我们一没有“攻”,二没有争,只是坐在冷板凳上辛苦耕耘。有了一点成绩,日本学者看在眼里,想在心里,觉得过去对中国南亚研究的评价过时了。我觉得,这里面既包含着“弘扬”,也包含着“发扬”。怎么能说,我们这一套无补于国计民生呢?
 
 话说远了,还是回来谈我们的本题。
 
 我的大学生活是比较长的:在中国念了4年,在德国哥廷根大学又念了5年,才获得学位。我在上面所说的“这一套”就是在国外学到的。我在国内时,对“这一套”就有兴趣,但苦无机会。到了哥廷根大学,终于找到了机会,我简直如鱼得水,到现在已经坚持学习了将近六十年。如果马克思不急于召唤我,我还要坚持学下去的。
 
 如果想让我谈一谈在上大学期间我收获最大的是什么,那是并不困难的。在德国学习期间有两件事情是我毕生难忘的,这两件事都与我的博士论文有关联。
 
 我想有必要在这里先谈一谈德国的与博士论文有关的制度。当我在德国学习的时候,德国并没有规定学习的年限。只要你有钱,你可以无限期地学习下去。德国有一个词儿是别的国家没有的,这就是“永恒的大学生”。德国大学没有空洞的“毕业”这个概念。只有博士论文写成,口试通过,拿到博士学位,这才算是毕了业。
 
 写博士论文也有一个形式上简单而实则极严格的过程,一切决定于教授。在德国大学里,学术问题是教授说了算。德国大学没有入学考试。只要高中毕业,就可以进入任何大学。德国学生往往是先入几个大学,过了一段时间以后,自己认为某个大学、某个教授,对自己最适合,于是才安定下来。在一个大学,从某一位教授学习。先听教授的课,后参加他的研讨班。最后教授认为你“孺子可教”,才会给你一个博士论文题目。再经过几年的努力,搜集资料,写出论文提纲,经教授过目。论文写成的年限没有规定,至少也要三四年,长则漫无限制。拿到题目,十年八年写不出论文,也不是稀见的事。所有这一切都决定于教授,院长、校长无权过问。
 
 \textbf{写论文,他们强调一个“新”字,没有新见解,就不必写文章}。见解不论大小,唯新是图。论文题目不怕小,就怕不新。我个人觉得,这是非常重要的一点。只有这样,学术才能“日日新”,才能有进步。否则满篇陈言,东抄西抄,饾饤拼凑,尽是冷饭,虽洋洋数十甚至数百万言,除了浪费纸张、浪费读者的精力以外,还能有什么效益呢?
 
 我拿到博士论文题目的过程,基本上也是这样。我拿到了一个有关佛教混合梵语的题目,用了三年的时间,搜集资料,写成卡片,又到处搜寻有关图书,翻阅书籍和杂志,大约看了总有一百多种书刊。然后整理资料,使之条理化、系统化,写出提纲,最后写成文章。
 
 我个人心里琢磨:怎样才能向教授露一手儿呢?我觉得,那几千张卡片,虽然抄写时好像蜜蜂采蜜,极为辛苦;然而却是干巴巴的,没有什么文采,或者无法表现文采。于是我想在论文一开始就写上一篇“导言”,这既能炫学,又能表现文采,真是一举两得的绝妙主意。我照此办理。费了很长的时间,写成一篇相当长的“导言”。我自我感觉良好,心里美滋滋的,认为教授一定会大为欣赏,说不定还会夸上几句哩。我先把“导言”送给教授看,回家做着美妙的梦。
 
 我等呀,等呀,终于等到教授要见我,我怀着走上领奖台的心情,见到了教授。然而却使我大吃一惊。教授在我的“导言”前画上了一个前括号,在最后画上了一个后括号,笑着对我说:“这篇导言统统不要!\textbf{你这里面全是华而不实的空话,一点新东西也没有!别人要攻击你,到处都是暴露点,一点防御也没有}!”对我来说,这真如晴天霹雳,打得我一时说不上话来。但是,经过自己的反思,我深深地感觉到,教授这一棍打得好,我毕生受用不尽。
 
 第二件事情是,论文完成以后,口试接着通过,学位拿到了手。论文需要从头到尾认真核对,不但要核对从卡片上抄入论文的篇、章、字、句,而且要核对所有引用过的书籍、报刊和杂志。要知道,在三年以内,我从大学图书馆,甚至从柏林的普鲁士图书馆,借过大量的书籍和报刊,耗费了大量的时间。当时就感到十分烦腻。现在再在短期内,把这样多的书籍重新借上一遍,心里要多腻味就多腻味。然而老师的教导不能不遵行,只有硬着头皮,耐住性子,一本一本地借,一本一本地查,把论文中引用的大量出处重新核对一遍,不让它发生任何一点错误。
 
 后来我发现,德国学者写好一本书或者一篇文章,在读校样的时候,都是用这种办法来一一仔细核对。一个研究室里的人,往往都参加看校样的工作。每人一份校样,也可以协议分工。他们是以集体的力量,来保证不出错误。这个法子看起来极笨,然而除此以外,还能有“聪明的”办法吗?\textbf{德国书中的错误之少,是举世闻名的。有的极为复杂的书竟能一个错误都没有,连标点符号都包括在里面。读过校样的人都知道,能做到这一步,是非常非常不容易的。德国人为什么能做到呢?他们并非都是超人的天才,他们比别人高出一头的诀窍就在于他们的“笨”}。我想改几句中国古书上的话:德国人其智可及也,其笨(愚)不可及也。
 
 反观我们中国的学术界,情况则颇有不同。在这里有几种情况。中国学者博闻强记,世所艳称。背诵的本领更令人吃惊。过去有人能背诵四书五经,据说还能倒背。写文章时,用不着去查书,顺手写出,即成文章。但是记忆力会时不时出点问题的。中国近代一些大学者的著作,若加以细致核对,也往往有引书出错的情况。这是出上乘的错。等而下之,作者往往图省事,抄别人的文章时,也不去核对,于是写出的文章经不起核对。这是责任心不强,学术良心不够的表现。
 
 还有更坏的就是胡抄一气。只要书籍文章能够印出,哪管它什么读者!名利到手,一切不顾。我国的书评工作又远远跟不上。即使发现了问题,也往往“为贤者讳”,怕得罪人,一声不吭。在我们当前的学术界,这种情况能说是稀少吗?我希望我们的学术界能痛改这种极端恶劣的作风。
 
 我上了9年大学,在德国学习时,我自己认为收获最大的就是以上两点。也许有人会认为这卑之无甚高论。我不去争辩。我现在年届耄耋,如果年轻的学人不弃老朽,问我有什么话要对他们讲,我就讲这两点。
 
 \newpage
 \section{<有一种魅力,叫惜字如金-颜卤煮>}
 我特别欣赏的一种人:不说废话的人。
 
 生活里,有些人说话总是掺杂很多水分。“水“,不是指作假,而是指话语凝练程度极低。
 
 比如,喜欢加很多无关痛痒的词,尤其是宏大词汇。
 
 “规模”、“模式”、“规划“……
 
 比如,喜欢用转述口吻或夸张来表达事实层的东西,添油加醋。
 
 “我就说,当时我就特别XXXX呢,果然,XXXXX,就知道!”
 
 比如,一次次重复地描述一件事,使用一些词。
 
 “我觉得”“你知道吗?“”你能懂吗?“
 
 还有一种,说一堆绕来绕去,核心意思要听半天才能搞清楚。
 
 ……
 
 (故意拐弯抹角的话术不在此文范围内,此文只讨论表达习惯和能力)
 
 
 
 (二)
 
 为什么自己会欣赏不说废话的人呢。
 
 因为, 人的能力和性格,和话语之间,有一种准确的对应关系。
 
 从能力方面说, 一个人说话,本质是一个抵达目的的过程。
 
 当说话要经过很多弯路,能反应两个能力缺失:
 
 1、目的性弱:想不清楚到底想表达什么。
 
 2、逻辑性差:分解目标,辨识含义层次、高效组合言语的能力太差。
 
 当一个人说话精准明确时,能反映出以下能力:
 
 1、\textbf{抓得住重点}:蹩脚的说话者,总在做加法,越描越黑。精炼的说话者,大多是在做减法,只把最核心的东西拎出来。
 
 2、\textbf{深思熟虑}:习惯捞“干货”的说的人,早就在心里把旁枝末节剔除得差不多,他们习惯先想清楚再张口,脑子一般转得也很快。
 
 3、\textbf{情商不低}:人的注意力有限,如果铺垫太多,听话者是没有耐心听下去的。所以不说废话的人,大多深谙听者心理,知道如何在有限时间里吸引注意力,不耽误彼此时间。
 
 
 
 (三)
 
 在一个人的能力之下,更深一层,是性格。
 
 1、\textbf{克制力}:自己曾在《沉默,比说出口更让人无法忘记》里说过,说话做事懂得“留白”的人,其实很少。
 
 过度地描述和表达,其实说到底,是一种贪。
 
 和贪吃、贪财、贪性一样,\textbf{贪图表达},也是一种贪。人对于对想要达到的意义,总是贪图描述。
 
 但话语,是一个过犹不及的东西,它和一般努力不一样,说出的话是不可逆的,\textbf{很多东西讲太多,反倒毁了}。
 
 所以,当一个人原希望达到的状态被欲望压倒时,他的性格根基是柔弱的,克制力不足,意志薄弱。
 
 2、\textbf{务实}:人在世界上,跟说话对应的,是行动。不是说话多之人,做得一定少;只是相对而言,\textit{务实之人,一般不会花太多时间在无效率表达上}。
 
 以事情结果为导向的人,往往废话很少。\textbf{在评价事情时,也较客观具体,不在空洞的理念上来回玩弄}。
 
 这背后其实是一种世界观在“作祟”,有些人喜欢喜欢将世界从抽象往具体过渡、落实;有些人则喜欢将具体和琐碎,往抽象化上去归纳,评判。前者更为实干和唯物,后者较为学院和唯心。
 
 3、\textbf{内强}:演讲场合,不同人有不同的面相:有些人废话多,有些人则从容不紊……这跟一个人是否内强有很大的关系。
 
 一个人,心里越是发虚,就越像溺水者,试图抓住能想到的所有辞藻,来填补和掩饰准备不足或内心不安的尴尬。
 
 \textbf{而一个准备充分,经验丰富,或是依靠实干起来的人,会更有底气}。
 
 TA与所说的话,不是对抗的不信任关系,而是十足的受控关系:\textbf{我说的话,每个字眼都是亲身经历的,它们是事实}。
 
 这就是一个人内强的意涵:真实的实力。
 
 4、\textbf{果断}:\textit{人的所思、所说和所做常常是一根直线上的三个点,彼此互相映照}。
 
 多余的话,反映在思考上,就是在盘算更多的可能;反映在做事上,就是在给自己留后路。
 
 不够坚定。
 
 而一个割舍了废话的人,正是带着一种破釜沉舟的决心:没那么多借口,没那么多其他可能。
 
 这样的人在工作中:小到一件事,是什么,能不能做,怎么做,几点能做完,干脆利落。大到一番事业,开不开干,怎么干,一条路走到黑。
 
 他们就像一只豹,盯住一个猎物,就锚定了往死里追,目的意识极强,专注力很高,没有太多中间状态。
 
 
 
 (四)
 
 以上,在我看来,是男人很重要的几个特质。关于此,曾在文章《女人最抗拒不了哪种男人?》中说过。
 
 对男人来说,\textbf{做,永远比说来得重要};\textbf{实力,永远比浮夸来的重要};\textbf{理性,永远比感性重要}。
 
 人,可以有间隙,但如果过度被情绪和软弱俘获,就很难承担责任。
 
 事业和责任,驱动力来源于本能(创造价值),但执行的过程却又是反本能(软弱、放纵、退后、虚荣)的,注定是痛苦的。
 
 人,作为动物,天生有一种“坠落”的本能,向下寻找着黑暗、轻松、欢愉、世俗的东西。 所以,如果你心里有一个更大的picture,就该学会克制、务实、内强、果断。
 
 这也是为什么那些历史上有所建树的人,看起来都跟铁人一样,很难想象他们一路上对那些坠落本能的控制。
 
 人生,要有所实现,就必然要有所失去。
 
 不仅男人,女人何尝不如此。
 
 对女人来说,尊重和宠溺,从来都很难兼得。
 
 一个少女心的女人,就很难有魄力;
 
 一个被外人羡慕的金丝雀,往往没有自由;
 
 一个聪明的女人,常常就很难可爱;
 
 一个清醒理智的女人,也就没那么多情痴傻了。
 
 人在世界上的位置,是自己一点一滴抉择出来的,不见得每一步都是具有决定性意义,但冥冥,你就在朝着心里的目的靠近。
 
 无所谓对错,世俗的欢愉浮夸,自有作用,它们让生命更“轻”,但对有些人来说,那些“重”而密实的东西,才是其活着的真正价值。 
 
 \newpage
 \section{《 我不喜欢这世界、我只喜欢你 》-乔一}
 \subparagraph{感悟:}暖暖的细节,让我想起自己日记的方式的描写方式, 描写的细节,固定的段式,特别棒..
 
 \subsection*{句子}
 \begin{itemize}
 	\item 老板,需要特殊服务吗?
 	\item 姐夫,我们这样做对得起我姐吗?
 	\item 我才是最爱乔一的人!我是不会把她让给你的!
 	\item 我俩这样子像不像在偷情?--动作快点,我老婆五点下班。
 	\item 我这人运气一向不好,我这辈子最幸运的事大概就是遇见你,所以我特别特别珍惜,长这么大唯一坚持下来的事情就是爱你。
 	\item 但是你不要以为这样我就会原谅你凌晨三点才回家。
 	\item 听歌的时候感觉你就坐在我旁边
 	\item 不,在家很爱撒娇,经常看电影哭得眼泪鼻涕要我哄,跟个小孩儿一样。” 同事很困惑:“为什么?” “\textbf{因为只有在我面前,她可以不用坚强}。”
 	\item 天正好车里在放张悬的《宝贝》,里面有一句歌词:“我的小鬼小鬼,逗逗你的眉眼,让你喜欢这世界,” 我说:“你看这歌词写的不就是你吗,跟个小孩儿似的好像世界都是你的。” 我自说自话了半天,声音越来越小,越来越哽咽,心里委屈得要死,心想不理就不理吧大不了分手。 一路无话。车在我公司楼前停下,我正准备开门,身后的他突然拉住我,低头闷闷地说:“可是……我不喜欢这世界,我只喜欢你。”
 	\item 我也不知道。 他温柔地握住我的手,“但我知道,一想到能和你共度余生,我就对余生充满期待。”
 	\item 我一边抄一边安慰自己,圣经上说,施比受更有福,我不是作弊,我是在帮F同学积攒幸福的资本。
 	\item 有一次他感冒很久没好,我特心疼,想给他带药,但又不好意思。纯情的我想了个特别迂回的方式——回家用冷水洗了个头,成功把自己也弄感冒了
 	\item 现在想想,真是被自己的脑回路蠢哭了(婷婷生日行-10.04动车)
 	\item 说来奇怪,我小时候幸福感不如现在充沛,因为害怕失望,所以对任何事都不敢抱有希望,但我就是信任他,特别信任,从来没有人像他那样,可以给我带来那么多的安全感
 	\item 小时候我最羡慕的就是独生子女,观潮跟我从小打到大,而且他从来不让我。去年我表姐生了二胎,大儿子闷闷不乐,吃满月酒时观潮眉开眼笑地跟人家说,恭喜啊,以后做错事可以找她背黑锅——可见我小时候过得多么悲催。
 	\item 卧槽!!!你他妈太区别对待了!!!
 	\item 好,朕这去帮你欺负回来。
 	\item 他说那时他站在医院的楼道里,很认真地想,要是我没挺过去,他就把名字改成我的,替我在这个世界上继续活
 	\item 是啊,妹妹太要强,哥哥什么都帮不了,只能帮她哭
 	\item 冷战的原因说来好笑——他跟我表白,而我拒绝了他。其实也算不上表白,F君闷骚又傲娇,连表白都特别婉约。
 	\item \textbf{遇到自己喜欢的人,反而小心翼翼不敢更靠近他,我不太明白自己这是什么心态。后来我看了一部电影,男主角问他的老师:“为什么我们总爱上那些不在乎我们的人?”}
 	\item 恍然大悟,是的,我觉得自己不配得到他的爱。
 	\item 只好说实话:“读书的时候有个特别好的男生喜欢我,我觉得人不能太自私,他为我走了九十九步,我也应该为他走一步。” 他不客气地问:“没这男的你会死吗?” “死倒不至于,但肯定会遗憾,我长这么大第一次觉得有人值得我拼命去珍惜,我不想再失去他。
 	\item \textbf{喜欢一个人,就像喜欢富士山,你可以看到它,但是不能搬走它——你唯一能做的,就是自己走过去,去争取自己的爱人。}
 	\item 我从来不想要什么更好的人,我只想要眼前的人,你究竟什么时候才会懂?”
 	\item \textbf{海底月是天上月,眼前人是心上人。}
 	\item 郝五一看到我,愣了一下,然后哇地一声就哭了,哭得特响亮。看见她哭我也哭了,可能是害怕自己真的也会被记大过吧。如果是电影,这时候要是来个俯拍,画面就能看见一片红彤彤的海洋里,两个小蓝点面对面哇哇哭,跟神经病似的。连主席台上训话的校长都停下来,目瞪口呆地看着我们俩。 当然后来我们没被记过,只被罚打扫了一个月卫生。郝五一跟我说,那是她长这么大最丢人的一次经历,却因为我让记忆无比温暖,那时候她就觉得,我一定是她一辈子的朋友。
 	\item 那时青春年少的我们,单纯而充满热情,叛逆而天真善良,那真是最好的我们。
 	\item \textbf{喜欢上一个人总需要一点时间,而我又总是在这段时间里发现对方是个傻逼。}
 	\item 亲爱的,你做什么我都站在你这边,就算你要逃婚我也给你买跑鞋。
 	\item 善良,体贴,有趣,独立,性格好,有品位。
 	\item 直到那一刻我才明白,其实很早以前,早到我还在瑟缩着逃避着将他推远时候,他就一声不吭地,把我计划进了他的未来。
 	\item \textbf{成长最遗憾的部分在于,我们总在最无知的年华遇到最好的人,却不自知。}
 	\item 那是他走了之后,我第一次因为这件事流泪。我很能克制自己的感情,也很擅长忍耐,可是那一刻真的觉得自己无法再克制下去了。我发现自己有那么那么多话想对他讲,想告诉他,我现在很快乐,我没有特别自卑了,我的生活轻松愉悦,每天都努力让自己变得更好,我去参加社团学生会各种选修课,一点一点变优秀,我对未来充满期许。现在的我是最好的我,如果你在我身边就好了。 只是春光如此,却不得见你。
 	\item \textbf{回去的路上,那句话一直一直在脑海里挥不去。从图书馆回寝室平时只要十分钟,那天我走了很远的路,一个人把学校绕了一圈}
 	\item 最后一抹余辉留在地面,梧桐树在两侧被风吹得沙沙响,一切都平常,一切都很好,可是那个瞬间,我突然体会到了什么叫“能与人说的都不算孤独。
 	\item 第一步:闭嘴。不管女友说什么都不能顶嘴,小不忍则乱大谋,如果忍不住吵起来,后面会有更大的麻烦。 第二步:认错,别问为什么,必须主动认错。 第三步:拥抱+表白,要紧紧拥抱住对方,注视着对方的眼睛,用你这辈子最诚恳的语气说:“对不起,我爱你。”
 	\item 是的,消停的这段时间他是去采集资料了,为了游说我答应同居,他从各方面分析了利弊,写出具体执行方法,甚至还有案例借鉴,做了柱状图,区域图,mekko图…… “如……如果我还是不答应呢?” “我再做一个ppt。” 你真的够了!
 	\item 打电话找F救场,周末他在公司加班,我本想让他回家给我带双鞋过来,但又怕耽误时间就忍着没说。没过多久他来了,还提了个纸袋,我心想难道这家伙还买了礼物讨好室长?结果他拿出来我就愣了,居然是我的鞋,平底的!我说:“你回家了?”他说:“没,早上看你出门穿的高跟鞋,估计你会累,就帮你带了双鞋放车上。”他一边说着,一边很自然地蹲下来给我换鞋。室长当时愣住了,一脸“我靠我看到了什么!”的表情。
 	\item 我幼稚地以为不会爱上任何人,我不确定自己有爱的能力。但是他告诉我,爱是人类与生俱来的天赋,是根植于每个人的生命之中,无论周围的土壤再怎么贫瘠,它都不会消失,只要有人唤醒,它就一定在。
 	\item 可是现在我真的很怕有那一天,倒不是害怕病痛的折磨,而是害怕现在珍惜的一切会再也无法感受,无法再拥抱我爱的人,无法陪我的子女成长,更重要的是,我和他已经成为对方的一部分,点点滴滴都连在一起,离开的那个人不是最痛苦的,最痛苦的反而是留下来的那一个。我一点都舍不得他难过。想到这些,我就害怕得不行。他说我多愁善感,可他不知道,让我变成这样的人是他。\textbf{爱让人突然有了盔甲,也突然有了软肋}。
 	\item 走进超市,我脸上写着“精打细算”四个大字
 	\item 人家说男人最性感的时候是你对他拳打脚踢时他静静的看着你再把你搂进怀里,可是你从来没有过!
 	\item 这货做的西红柿鸡蛋不光放了酱油居然还有醋味儿!
 	\item 他一本正经的胡说八道:“我在给它们军训,指甲油同学体质太弱,晕了过去……”
 	\item “小乔同学何等人物,就算把门窗都关上,她也能爬上房把屋顶掀了。”
 	\item 好像真是这样,不管多累,心情多不好,只要一见到他就会高兴起来。用室长的话来说,我看F君的眼神就跟她们家狗看到火腿肠一样,就差摇尾巴了。一个人爱着另外一个人是藏不住的,不信你去看看《大话西游》里紫霞仙子看至尊宝的眼神,满满的全是爱。
 	\item F君得之我小时候爸妈没给我讲过睡前故事,心生怜惜,主动要求每晚给我讲童话哄我睡觉,难得摩羯男会做这么浪漫的事,
 	\item \textbf{上了辆双层巴士,坐在第一排,长安街上霓虹五光十色,秋末的夜晚凉风正好,车上没什么人,一切都刚刚好,我突然想,这个时候多适合接吻啊}
 	\item 遇到红灯,巴士稳稳停下。我在心里倒数,心想变成绿灯就吻他,眼睛偷偷瞄红灯旁边的数字。“5——4——3——”他突然凑上来,轻轻盖住了我的唇。脑子里一片空白,本能的接了这个吻。像有根羽毛轻轻划过心尖,酥酥麻麻的。
 	\item 天气好得想骂娘,不想读书想去浪……
 	\item 说来伤感,有些人就是这样,平时插科打诨从来不正经说话,可是安静下来的那一秒,你会突然发现,他心里什么都懂,他只是不说。
 	\item 青春就像一场飓风,轰轰烈烈席卷而去,我们在狼藉中上踩一脚,能踩出一地砖头瓦砾
 	\item 跟着你,去哪,做什么都好。
 	\item 不考虑现实因素,你最想做什么职业?Q:海盗。F:海军。Q:为什么?F:追你啊。
 	\item 我突然发现在爱情里人人平等,原来他也会不自信,会害怕,会软弱,会小心翼翼,会不知道该如何爱一个人。我想爱情可能不是谁带领谁,而是双方共同成长才能达到安心与自在,如果说,喜欢是渴望将好的一起分享,那么,爱是愿意把坏的共同承担。我知道这条路很长,好在一辈子很长,我想陪他慢慢走。
 	\item 我相信现在一切危机都是你我人生必经的——没钱,压力大,不得志……每个人都会有这一段。你不用焦虑,慢慢来,大不了我养你。
 	\item 车堵在路上久久没动,北京永远在堵车,这里空气不好,城市太大,人潮拥挤,我有一万个不喜欢这里的理由,可我爱的人在这里,我就在这里安了家。爱让我们褪去身上的青涩的棱角,穿越汹涌的人潮,用最温柔最炙热的爱拥抱彼此,我知道这个世界什么都善变,可是说真的,\textbf{眼前这个人,他让我相信永远}
 	\item \textbf{能够到达金字塔顶尖的动物有两种:雄鹰和蜗牛,雄鹰是少数,大多数都是蜗牛,所以我们没资格懒惰。}
 \end{itemize}
 
 
 \newpage
 \section{<背影 - 朱自清>}
 我与父亲不相见已二年余了,我最不能忘记的是他的背影。那年冬天,祖母死了,父亲的差使也交卸了,正是祸不单行的日子,我从北京到徐州,打算跟着父亲奔丧回家。到徐州见着父亲,看见满院狼藉的东西,又想起祖母,不禁簌簌地流下眼泪。父亲说,“事已如此,不必难过,好在天无绝人之路!”
 
 回家变卖典质,父亲还了亏空;又借钱办了丧事。这些日子,家中光景很是惨淡,一半为了丧事,一半为了父亲赋闲。丧事完毕,父亲要到南京谋事,我也要回北京念书,我们便同行。
 
 到南京时,有朋友约去游逛,勾留了一日;第二日上午便须渡江到浦口,下午上车北去。父亲因为事忙,本已说定不送我,叫旅馆里一个熟识的茶房陪我同去。他再三嘱咐茶房,甚是仔细。但他终于不放心,怕茶房不妥帖;颇踌躇了一会。其实我那年已二十岁,北京已来往过两三次,是没有甚么要紧的了。他踌躇了一会,终于决定还是自己送我去。我两三回劝他不必去;他只说,“不要紧,他们去不好!”
 
 我们过了江,进了车站。我买票,他忙着照看行李。行李太多了,得向脚夫行些小费,才可过去。他便又忙着和他们讲价钱。我那时真是聪明过分,总觉他说话不大漂亮,非自己插嘴不可。但他终于讲定了价钱;就送我上车。他给我拣定了靠车门的一张椅子;我将他给我做的紫毛大衣铺好坐位。他嘱我路上小心,夜里警醒些,不要受凉。又嘱托茶房好好照应我。我心里暗笑他的迂;他们只认得钱,托他们直是白托!而且我这样大年纪的人,难道还不能料理自己么?唉,我现在想想,那时真是太聪明了!
 
 我说道,“爸爸,你走吧。”他望车外看了看,说,“我买几个橘子去。你就在此地,不要走动。”我看那边月台的栅栏外有几个卖东西的等着顾客。走到那边月台,须穿过铁道,须跳下去又爬上去。父亲是一个胖子,走过去自然要费事些。我本来要去的,他不肯,只好让他去。我看见他戴着黑布小帽,穿着黑布大马褂,深青布棉袍,蹒跚地走到铁道边,慢慢探身下去,尚不大难。可是他穿过铁道,要爬上那边月台,就不容易了。他用两手攀着上面,两脚再向上缩;他肥胖的身子向左微倾,显出努力的样子。这时我看见他的背影,我的泪很快地流下来了。我赶紧拭干了泪,怕他看见,也怕别人看见。我再向外看时,他已抱了朱红的橘子望回走了。过铁道时,他先将橘子散放在地上,自己慢慢爬下,再抱起橘子走。到这边时,我赶紧去搀他。他和我走到车上,将橘子一股脑儿放在我的皮大衣上。于是扑扑衣上的泥土,心里很轻松似的,过一会说,“我走了;到那边来信!”我望着他走出去。他走了几步,回过头看见我,说,“进去吧,里边没人。”等他的背影混入来来往往的人里,再找不着了,我便进来坐下,我的眼泪又来了。
 
 近几年来,父亲和我都是东奔西走,家中光景是一日不如一日。他少年出外谋生,独力支持,做了许多大事。那知老境却如此颓唐!他触目伤怀,自然情不能自已。情郁于中,自然要发之于外;\textbf{家庭琐屑便往往触他之怒。他待我渐渐不同往日。但最近两年的不见,他终于忘却我的不好,只是惦记着我,惦记着我的儿子}。我北来后,他写了一信给我,信中说道,\textbf{“我身体平安,惟膀子疼痛利害,举箸提笔,诸多不便,大约大去之期不远矣。”我读到此处,在晶莹的泪光中,又看见那肥胖的,青布棉袍,黑布马褂的背影}。唉!我不知何时再能与他相见!
 
 \newpage
 \section{《 挪威的森林 》 -村上春树 }
 \begin{itemize}
 	\item 理解一件事需要有个过程。但只要时间,总会完全理解你的,而且比世上任何人都理解得彻底。
 	\item 文章这种不完整容器所能容纳的,只能是不完整的记忆和不完整的意念
 	\item “这些日子总是这样。一想表达什么,想出的只是对不上号的字眼。有时对不上号,还有时完全相反。可要改嘴的时候,头脑又混乱得找不出词来,甚至自己最初想说什么都糊涂了。好像身体被分成两个,相互做追逐游戏似的。而且中间有根很粗很粗的大柱子,围着它左一圈右一圈追个没完。而恰如其分的字眼总是由另一个我所拥有,这个我绝对追赶不上。”
 	\item 如今想来,那真是奇特的日日夜夜。在活得好端端的青春时代,居然凡事都以死为轴心旋转不休
 	\item 我们两人漫无目标地在东京街头走来转去。上坡,过河,穿铁道口,只管走个没完。没有明确的目的地,反正走路即可。仿佛举行一种拯救灵魂的宗教仪式般地
 	\item 当秋天过去,冷风吹过街头的时节,她开始不时地依在我的胳膊上。透过粗花呢厚厚的质地,我可以微微感觉出直子的呼吸。她时而挽起我的胳膊,时而把手插进我的大衣口袋里。特别冷的时候,就紧贴着我身旁籁籁发抖,但仅此而已。她的这些动作并无更深的含义。我双手插进大衣兜,一如往常地走动不止。我和直子穿的都是胶底鞋,几乎听不见两人的脚步声,只有踩上路面硕大的法国梧桐落叶的时候,才发出"嚓擦"的干燥声响。而一听到这种声响,我便可怜起直子来。她所希求的并非我的臂,而是某人的臂。她所希求的并非我的体温,而是某人的体温。而我只能是我,于是我觉得有些愧疚。
 	\item 不是说我不相信现代文学。我只是不愿意在阅读未经过时间洗礼的书籍方面浪费时间。人生短暂。
 	\item 这是再简单不过的买卖,就像拧开水龙头喝水一样。我们转眼间就可以发泄,而对方又求之不得。这就是所谓可能性。这种可能性就在眼前来回晃动,难道你能视而不见?自己具有这种能力,又有发挥这种能力的场所,你能默默通过不成
 	\item 惟有死者永远17
 	\item 我几乎下意识地搂过她的身体。她在我怀中浑身发抖,不出声地抽泣着。泪水和呼出的热气弄湿了我的衬衣,并且很快湿透了。直子的十指在我背上摸来摸去,仿佛在搜寻什么曾经在那里存在过的珍贵之物。我左手支撑直子的身体,右手抚摸着她直而柔软的头发,如此长久地等待直子止住哭泣。然而她哭个不停。
 	\item 也想过,或许我不该那样做。但此外别无他法。当时我在你身上感觉到的亲密而温馨的心情,是一种迄今我从未曾感受过的情感。请你回信,什么内容都可以--只要回信。
 	\item 我心里失落了什么,而又没有东西填补,只剩下一个纯粹的空洞被弃置不理。身体轻得异乎寻常,语音虚无缥缈。
 	\item 眼下我还没有见你的准备,不是不想见,是没完成见的准备。一旦准备完成,我马上写信给你。到那时候,我想我们也许会多少相互了解。如你说的那样,我们应该加深对对方的了解才是。
 	\item 这种莫可名状的心绪,我既不能将其排遣于外,又不能将其深藏于内。它像掠身而去的阵风一样没有轮廓,没有重量。我甚至连把它裹在身上都不可能。风景从我眼前缓缓移过,其语言却未能传人我的耳底。
 	\item 日落天黑,宿舍院里十分寂静,竟同废墟一般,国旗从旗杆降下,食堂窗口亮起灯光。由于学生人数减少,食堂的灯一般只亮一半。左半边是黑的,只有右半边亮。但还是微微荡漾着晚饭的味道,是奶油炖菜的气味儿。
 	\item 我凭依栏杆,细看那萤火虫。我和萤火虫双方都长久地一动未动。只有夜风从我们身边掠过。榉树在黑暗中磨擦着无数叶片,籁籁作响。
 	\item 萤火虫消失之后,那光的轨迹仍久久地印在我脑海中。那微弱浅淡的光点,仿佛迷失方向的魂灵,在漆黑厚重的夜幕中往来彷徨。 我几次朝夜幕中伸出手去,指尖毫无所触,那小小的光点总是同指尖保持一点不可触及的距离。
 	\item "哪里会有人喜欢孤独!不过是不乱交朋友罢了。那样只能落得失望。"
 	\item "也不是特别喜欢,什么都无所谓。"
 	\item "绅士就是:所做的,不是自己想做之事,而是自己应做之事。"
 	\item "人生中无需那种东西,需要的不是理想,而是行为规范!"
 	\item 你离去后,无论做什么我都觉得索然无味,很想同你见面好好谈一次。
 	\item 真想把我的时间分出些来,让你在里边好好睡上一觉
 	\item 没有《战争与和平》,没有《性的人》,没有《麦田里的守望者》。这就是小林书店,这烂摊子到底有什么可值得羡慕的
 	\item 也算不得吃苦头,是的。不过是说钱不是大把大把的。世上的人大都如此。
 	\item 就是--把别人不写的内容多少加进去一点。这一来,地图公司的负责人就会认为'那孩子会写文章',心里佩服得不得了,就又找工作给你。其实也用不着大动脑筋,一点点就足够
 	\item 然后坐在最后边的位置,观望外面几乎擦窗而过的一排排古旧房屋。电车紧贴着家家户户的房檐穿行。一户人家的晾衣台上一字排开十盆盆栽西红柿,一只大黑猫蹲在一头晒太阳。在院子里吹肥皂泡的小孩闪入眼帘,石田亚由美的歌声不知从何处传来耳畔。甚至有咖喱气味飘至鼻端。电车像根缝衣针一样在密密麻麻的住宅地带婉蜒前行。途中有几个人上来。三位老太婆亲密无间地头对着头,不厌其烦地谈着什么。
 	\item 站在小林书店门前时,我不由产生了一种似曾相识的亲切之情:哪条街上都有这样的书店。
 	\item 一面呷着啤酒,一面望着全神贯注做饭的绿子背影。她快捷而灵巧地挪动着身子,同时操作四五样菜。眼看在这边品尝菜的味道,转眼就在菜板上飞快地切什么东西,又从电冰箱里取出什么盛上,一回手把用完的锅涮好。从后边望去,那样子不禁使人想起印度打击乐的演奏者来:刚击响那边的吊钟,马上又敲这边的板,旋即拍打水牛骨。每一个动作都敏捷而准确,相互配合得恰到好处。我出神地望着。
 	\item 后面看,她的腰格外的苗条、格外的窈窕,简直像在使腰肢壮实起来的发育过程中,不知什么原因跳过了一个阶段:就是这样美不胜收的腰。因此,同一般女孩子穿窄牛仔裤时相比,她给人的印象要中性得多。烹调台上方窗口射进的明晃晃的阳光,为她身段的轮廓镀上了一层恍惚而隐约的光膜。
 	\item 太麻烦了。譬如说半夜断烟时那个难受滋味吧,等等。所以戒了。我不情愿被某种东西束缚住。
 	\item 我们爬上三楼的晾衣台看热闹,而且不知不觉地接了吻。这么说也许像是装傻,可过程确实如此。
 	\item "起码我是认认真真这样想的,也只能这样想,不过把它照实说出口罢了。我从不认为我的想法与别人有什么两样,也不去追求那种两样。坦率地说,我觉得她们统统是在自欺欺人或逢场作戏。因此有时候对什么都讨厌得要死。
 	\item 看看绿子的眼睛,绿子也看看我的眼睛。我搂过她的肩,吻住她的嘴。绿子只是肩头稍微抖动一下,旋即软绵绵地闭上眼睛。约有五六秒,我们悄无声息地对着嘴唇。初秋的阳光把她的眼睫毛投影在脸颊上,看上去微微发颤。
 	\item 我蓦然注意到一个事实:每个人无不显得很幸福。至于他们是真的幸福还是仅仅表面看上去如此,就无从得知了。但无论如何,在9月间这个令人心神荡漾的下午,每个人看来都自得其乐。而我则因此而感到平时所没有过的孤寂,觉得惟独我自己与这光景格格不入。
 	\item 走出电影院时已快凌晨4点,在凉意袭人的新宿街头一边胡思乱想,一边漫无目的地转悠着。
 	\item 反正我是在看书,谁与我对坐都不碍事。
 	\item 我把直子寄来的七页信纸拿在手里,沉浸在漫无边际的思绪中。只读罢开头几行,我便觉得周围的现实世界黯然失色。我闭上眼睛,花很长时间把自己的心收拢回来,然后深深吸了口气,继续读下去。
 	\item 我们在此静静地生活,避免相互伤害。
 	\item 好歹入睡时,已过半夜1点了
 	\item 再说我来这里已经7年,世上的事,早就一无所知了
 	\item 你这个人,说话方式还挺怪的。”她说,“是模仿《麦田里的守望者》里那个男孩吧?”
 	\item 有四个房间:客厅、卧室、厨房、盥洗室,简洁明快,给人的感觉不错。没有多余的装饰,没有不谐调的家具,但并不给人以凄清之感。
 	\item 不知为什么,在这房间里一躺,过去几乎未曾想起的事情居然纷至沓来地浮上脑海。有的令人心神荡漾,有的则带有一丝凄楚
 	\item 棱线上浮现着淡淡的夕晖,宛如镀上了一层光边。
 	\item 说是托马斯·曼的《魔山》。
 	\item 我们吃饭时,几个人进来,几个人出去。
 	\item 我猜想,直子和他们在一起时,恐怕也是这样讲话。说来奇怪,一瞬间,一股夹杂着嫉妒的寂寥感掠过我的心头。
 	\item 谈什么?平常事啊。一天中遇到的事,看的书,明天的天气,不外乎这些
 	\item 再说也没有必要提高嗓门,既用不着说服谁,又没有引人注目的必要。
 	\item 对某一特定领域怀有强烈兴趣的人集中在特定的场所,交换惟独同行间才懂得的信息。
 	\item 事情不过发生在半年前,我却觉得似乎过去了很久很久。或许因为我对此不知反复考虑了多少次的缘故。由于考虑的次数太多了,对时间的感觉便被拉长,而变得异乎寻常。
 	\item “一听这曲子,我就时常悲哀得不行。也不知为什么,我总是觉得似乎自己在茂密的森林中迷了路。”直子说,“一个人孤单单的,又冷,里面又黑,又没一个人出来救我。
 	\item 那深邃澄澈的眸子和羞涩似的嗫嚅着的小嘴唇倒是和以前一样,但整个看来,她的娇美已开始带有成熟女性的风韵。往日她那娇美中时隐时现的某种锐气——如同使人为之颤栗的刀刃般的锐气——已经远远遁去,转而荡漾着一种给人以亲切抚慰之感的特有的娴静。我为这样的娇美而怦然心动。同时又感到有些惊愕:不过半年时间,一个女人居然会有如此明显的变化
 	\item 他那人,不属于喜欢不喜欢的范畴,而且他本人所追求的也不是这个。在这个意义上,他是个非常直率的人、不弄虚作假的人、极其清心寡欲的人
 	\item 死的人就一直死了,可我们以后还要活下去。
 	\item “不时出现那种情况,亢奋、哭泣。不过不要紧,这样还好,因为可以把感情宣泄出去。可怕的是感情泄不出去。那一来,就会憋在心里,越憋越多,各种感情憋成一团,在体内闷死,那可就要坏事了。
 	\item \textbf{人若要在某件事上扯谎,就势必为此编造出一大堆相关的谎言。这就是说谎症}
 	\item 音量并不大,而且大概由于过度吸烟的关系,嗓音有些沙哑,但很有厚度,娓娓动人。我喝着啤酒,望着远山,耳听她的歌声,恍惚觉得太阳会再次从那里探出脸来。那心境实在太温馨、太平和了
 	\item 我们坐在草地上的干草上,抱在一起。我们的身体完全隐没在草丛之中,除了天空和白云,什么都看不见了。我把直子慢慢放倒在草上,紧紧搂住她.直子的身体柔软而温暖,双手摸索着我的身子。我和直子接了一个深情的吻。
 	\item 风凉浸浸的,玲子在衬衫外面套了件对襟羊毛衫,双手插进裤袋。她边走边望天,像狗似的抽鼻子嗅了嗅,说“有一股雨气味儿”。我也同样嗅了一下,却什么也没嗅到。不过天空里云层确实多起来,月亮也被掩到后面去了
 	\item 他们不懂得下苦功夫,忽略了对人格形成必不可少的这一主要因素
 	\item 不是让她贪多求快,而是让她停下来回味。
 	\item 或使人恼怒,或使人悲伤,或使人同情,或使人沮丧,或使人欣喜,随心所欲地刺激别人的感情。她这样做,无非是因为想尝试一下自己的才能,但却无谓地操纵了别人的感情
 	\item 任凭怎么解释,\textbf{世人也只能相信自己愿意相信的事情。越是拼命挣扎,我们的处境越是狼狈}
 	\item 反正就是要送东西,接下去反正就是不管死活地给她灌酒,要灌醉,一杯接一杯灌,反正。再接下去就只剩下动干戈了。简单吧?
 	\item 不习惯的人都这样。味道、噪音、沉闷的空气、病人的面孔、紧张、焦躁、失望、痛苦、疲劳——就是这些造成的
 	\item “照你说的,独自一人,什么也不说,让脑袋处于真空状态。”
 	\item 随着午后时间的推移,窗外的阳光的色调变得柔和而沉静,一派秋日气息。小鸟成群结伙地飞来,落在电线上,又一忽儿飞去。我和绿子两人并坐在屋角处,压低声音说个不止
 	\item 在温暖的被窝里想你是十分惬意的事。恍惚觉得你就在我的身边,弓着身子睡得很熟很熟。倘若这是真的,那该多美呀!我想。
 	\item 见不到你固然是痛苦的,但倘若没有你,我在东京的生活将更不堪忍受。正因为一清早我就在床上想你,我才下决心拧紧发条,自强不息地生活下去。如同你在那边自强不息一样,我在这里也必须自强不息。
 	\item 周日的下午是安静而平和的,也是孤独的。我一个人看看书、听听音乐。也有时逐一回忆你在京时星期天咱俩行走的路线。你穿的衣服也清楚得如在眼前。星期天的下午我确实能记起很多东西。
 	\item 不由心想:这样的星期日以后将重复几十次、几百次吧?“安静的、平和的、孤独的星期日”——我出声说道。星期天我是不上发条的。
 	\item 嗯。语言这东西还是多学一种有好处,再说这是我天生的拿手好戏。法语也是自学的,几乎达到无懈可击的地步。和玩一个道理,只要摸到一条规律,往下任凭多少都是一个模式
 	\item 固然,有时也对人生怀有恐怖感,这也是理所当然!只是,我并不将它作为前提条件来加以承认。我要百分之百地发挥自己的能力,不达到极限绝不罢休。想拿的就拿,不想拿的就不拿,就这样生存下去。不行的话,到不行的时候再行考虑。反过来想,不公平的社会同时也是大有用武之地的社会
 	\item 所以,\textbf{有时我环顾世人就气不打一处来--这些家伙为什么不知道努力呢?不努力何必还牢骚满腹呢?}
 	\item “那不是努力,只是劳动。”永泽断然说道,“我所说的努力与这截然不同。所谓努力,指的是主动而有目的的活动。”
 	\item 我想绿子的父亲恐怕从来就未曾想起过要开始学什么西班牙语,恐怕根本就未曾考虑过努力和劳动的区别在哪里。
 	\item “这家伙一旦决定不说,就绝对守口如瓶。”
 	\item 但谁也没搭腔,如同小石子掉进了无底洞。
 	\item 我同渡边的相近之处,就在于不希望别人理解自己。”永泽说,“这点与其他人不同,那些家伙无不蝇营狗苟地设法让周围人理解自己。但我不那样,渡边也不那样,而觉得不被人理解也无关紧要。\textbf{自己是自己,别人归别人}。”
 	\item \textbf{我不是那样的强者,也并不认为不被任何人理解也无所谓,希望相互理解的对象也是有的。只不过对除此以外的人,觉得在某种程度上即使不被理解也无可奈何,这是不可强求的事}
 	\item 能如此执着地爱上一个人,这本身恐怕就是件了不起的事。
 	\item 我写了封长信,边写边用大杯子喝咖啡,边听迈尔斯·戴维斯的唱片。窗外细雨霏霏,室内如同水旅馆似的凉意浸人
 	\item 今天在下雨,下雨的周日多少使我有些惶惶然。因为下雨不能洗衣服,自然也不能熨衣服。既不能散步,又不能在天台上东倒西歪。只好坐在桌前,一边用自动反复唱机周而复始地听《温柔的蓝》,一边百无聊赖地观望院子的雨中景致。
 	\item 窗外商店街上的路灯光,宛似一派月华,给房间镀上一层若明若暗的银
 	\item 秋意的加深是与你返回东京同时开始的,因此我许久都捉摸不透自己心里仿佛出现一个大洞的感觉是由于你不在造成的,还是时令的更迭所致。
 	\item “不要同情自己!”他说,“同情自己是卑劣懦夫的勾当。”
 	\item 窗外是一大片庭园,附近的猫们将其作为集会场所。我一得闲,就歪倒在檐廊中观望那些猫。具体多少只倒不甚清楚,反正数目相当之多,而且都在横躺竖卧地晒太阳。它们似乎不大欢迎我住在这所独房里,但我拿出几块吃剩下的干酪后,有几只便挪步上前,战战兢兢地吃了下去。说不定过几天就会同它们成为好朋友。其中有一只耳朵少了半边的花纹公猫,这家伙同我原来宿舍的管理主任相似得惊人,我真担心庭园里会马上有国旗升起。
 	\item 我过去就有这毛病——一旦对什么人了迷,周围的一切便视而不见。
 	\item \textbf{我觉得应该思考点什么,又不知思考什么、怎么思考才好。其实说老实话,我什么都懒得思考。我想那不得不思考的时刻恐怕不久就将来临,届时再慢慢思考吧。至少现在什么都不想思考。}
 	\item 我走进屋子,拉合窗帘。屋内到底还是荡漾着春日的馨香,而且天地间无所不在,但现在使我联想起来的却惟有腐臭。我在窗帘拉得严严实实的屋子里狠狠地诅咒春天,诅咒春天给我带来的创伤——它使我心灵深处隐隐作痛。生来至今,如此深恶痛绝地诅咒一种东西还是第一次。
 	\item 你想必很痛苦,但我也不轻松,不骗你。这也是你留下直子死去造成的!但我绝不抛弃她,因为我喜欢她,我比她顽强,并将变得愈发顽强,变得成熟,变成大人——此外我别无选择。这以前我本想如果可能的话,最好永远十七、十八才好,但现在我不那样想。我已不是十几岁的少年,我已感到自己肩上的责任。喂木月,我已不再是同你在一起时的我,我已经20岁了!我必须为我的继续生存付出相应的代价!
 	\item 刚交5月,我就不能不感到自己的心开始在阑珊的春日中摇颤。这种摇颤大体在薄暮时分袭来。在浮动着玉兰花淡淡幽香的苍茫暮色里,自己的心开始无端地膨胀、颤抖、摇摆、针刺般地痛。这时我便紧闭双目、咬紧牙关,等待这番袭击的过去,而这要花很长时间,之后还留下丝丝隐痛。
 	\item 我只写得意的事项、愉快的感受和美好的际遇,只写芳草的清香、春风的怡然和月光的皎洁,只写看过的电影、喜欢的歌谣和动心的读物。写罢反复阅读之间,我本身竟也得到了慰藉,心想自己所生活的世界是何等美妙绝伦
 	\item 这固然遗憾,但别无他法。以前我也对你说过,对待这件事最好的办法就是耐心。不放弃希望,把相互纠缠的线索一一理出头绪。\textbf{无论事态看上去多么令人悲观,也必定在某处有突破口可寻。倘若周围一团漆黑,那就只能静等眼睛习惯黑暗。}
 	\item 两人足有两个月没开口了。上完课,绿子来我邻座坐下,手拄下巴,半天没有吭声。窗外雨下个不停。这是梅雨时节特有的雨,没有一丝风,雨帘垂直落下,一切都被淋得湿漉漉的。其他同学全部离开教室后,绿子也还是以那副姿势默然不动。一会儿,她从棉布上衣袋里掏出万宝路衔在嘴上,把火柴递给我。我擦燃一根给她点上。绿子圆圆地噘起嘴唇,把烟缓缓地喷在我脸上。
 	\item 也许从早上就开始下雨的关系,商店里空空荡荡,没有几个人影。整个店内充溢着雨的气味,店员也因无所事事而显出无聊的神情。我们走到设在地下室的餐厅。细细看了一遍陈列的样品,两人都决定吃盒饭。虽是午饭时间,但餐厅里人并不挤
 	\item 雨中的天台一个人也没有。宠物用品柜台看不见售货员。小卖部和乘用物售票处也都落着卷闸门。我们撑着伞,在湿漉漉的木马、花木架、摊床之间散步。东京的闹市区中心居然有此等荒凉的场所,我有些意外。绿子说要看望远镜,我投进一枚硬币,她看的时候我为她撑伞。
 	\item \textbf{对我来说,你这人总像有些与众不同。和你在一起,我感觉再称心如意不过。我信赖你,喜爱你,不愿放弃你}。
 	\item “为什么?”绿子吼道,“你脑袋是不是不正常?又懂英语假定形,又能解数例,又会读马克思,这一点为什么就不明白?为什么还要问?为什么非得叫女孩子开口?还不是因为我喜欢你超过喜欢他么?我本来也很想爱上一个更英俊的男孩儿,但没办法,就是相中了你。”
 	\item “只是,要我时就只要我,抱我时就得只想我。明白我说的意思?”
 	\item 我把伞放在脚下,顶着雨把绿子紧紧搂在怀中。惟有车轮碾过高速公路的沉闷回响仿佛缥缈的雾霭笼罩着我们。雨无声无息、执着地下个不停,我们的头发已被彻底淋透,雨滴如同泪珠一般顺颊而下,她的棉布牛仔夹克和我的黄色尼龙风衣全被染成了深色。
 	\item "喜欢我喜欢到什么程度?”绿子问。 “整个世界森林里的老虎全都融化成黄油。”
 	\item 管爱的方式在某一过程中被扭曲得难以思议,但我对直子的爱却是毋庸置疑的,我在自己心田中为直子保留了相当一片未曾被人染指的园地。
 	\item 当然我很遗憾,遗憾你同直子未能迎来大团圆的结局。然而归根结底,又有哪个人能明白什么算是好结局呢!因此你无须顾忌谁,如若你认为可以获得幸福,那就及时抓住机会!以我的经验来看,\textbf{人的一生中这种机会只有两三回,一旦失之交臂,一辈子都将追悔莫及。}
 	\item \textbf{信上说那既非我的责任,也不是某人的责任,而是如同天要下雨,不是任何人所能制止的。}
 	\item  继之给绿子写了封短信:现在一言难尽,希望稍待时日,请谅。此后三天时间里,我挨家进电影院,从早看到晚,大凡东京上映的影片统统看了一遍。尔后收拾好旅行背囊,提出所有的银行存款,去新宿站乘上第一眼看到的特快列车。
 	\item 至于去了什么地方以及如何去的,我全然无法记起。风景、气氛和声响记得真真切切,而地点却忘得干干净净。连顺序也忘了。我乘上火车或公共汽车,或搭坐路上所遇卡车的助手席,一个城镇接一个城镇地穿行不止。如果有空地有车站有公园有河边有海岸,及其他凡是可以睡觉的场所,我不问哪里,铺上睡袋便睡。也有时央求睡在派出所里,有时睡在墓地旁。只要是不影响通行而又可以放心熟睡的地方,我便肆无忌惮地大睡特睡。我将风尘仆仆的身子裹在睡袋里,咕嘟咕嘟喝几口低档威士忌,马上昏睡过去。遇到热情好客的小镇,人们便为我端来饭菜,借给我蚊香;而若是人情淡薄的地方,人们便喊来警察把我逐出公园。对我来说,好也罢坏也罢怎么都无所谓。我所寻求的不过是在陌生的城镇睡个安稳觉而已。
 	\item 我并无特定目的地,只是逐一在城镇中穿行不止。世界广阔无边,到处充满怪异的现象和奇妙的人们。
 	\item 在她身上自己做的何等残酷!想到这点,我心里感到一阵冰冷,无可救药的冰冷。我几乎从未思考过她会作何想法,有何感受,以及心灵受何刺激。甚至至今都未好好想过她一下。其实她是个非常温柔的女孩儿,只是当时我将那种温柔视为理所当然的东西,丝毫未加珍惜。
 	\item 云如枯骨,细细白白,长空寥廓,似无任何遮拦。又是一个秋天,我想。风的气息,光的色调,草丛中点缀的小花,一个音节留下的回响,无不告知我秋天的到来。四季更迭,我与死者之间的距离亦随之渐渐拉开。木月照旧17,直子依然21,直至永远。
 	\item 许久没听她的吉他了,那声音一如既往地温暖着我的心。
 	\item 实话,\textbf{见面前我担心得不得了,生怕她一下子瘦得摇摇晃晃,憔悴不堪。}
 	\item 外面漆黑一团,如同给墨汁涂得没留一点空白。虫声听起来格外响。连房间里都充满扑鼻的夏草气息
 	\item 夕阳垂垂西坠,斜晖奄奄一息,树影长长地伸至我们脚前。我一边喝茶,一边望着纷然杂陈的奇妙庭园——棣棠、杜鹃、南天竹等在那里我行我素地横躺竖卧。
 	\item 我现在哪里?
 	\item 拿着听筒扬起脸,飞快地环视电话亭四周。我现在哪里?我不知道这里是哪里,全然摸不着头脑。这里究竟是哪里?目力所及,无不是不知走去哪里的无数男男女女。我是在哪里也不是的场所的正中央,不断地呼唤着绿子。
 \end{itemize}
 \newpage
 \section{《 了不起的盖茨比 》 -菲茨杰拉德  }
 \begin{itemize}
 	\item “每逢你想要批评任何人的时候,”他对我说,“你就记住,这个世界上所有的人,并不是个个都有过你拥有的那些优越条件。”
 	\item 久而久之,我就惯于对所有的人都保留判断,这个习惯既使得许多有怪僻的人肯跟我讲心里话,也使我成为不少爱唠叨的惹人厌烦的人的受害者
 	\item \textbf{保留判断是表示怀有无限的希望。}
 	\item 盖茨比本人到头来倒是无可厚非的、使我对人们短暂的悲哀和片刻的欢欣暂时丧失兴趣的,却是那些吞噬盖茨比心灵的东西,是在他的幻梦消逝后跟踪而来的恶浊的灰尘。
 	\item ‘我很高兴是个女孩。而且我希望她将来是个傻瓜――这就是女孩子在这种世界上最好的出路,当一个美丽的小傻瓜。”
 	\item
 	\item 
 	\item 
 	\item 
 	\item 
 	\item 
 	\item 
 	\item 
 	\item 
 	\item 
 	\item 
 	\item 
 	\item 
 	\item 
 	\item 
 	\item 
 	\item 
 	\item 
 	\item 
 	\item 
 	\item 
 	\item 
 	\item 
 	\item 
 	\item  
 \end{itemize}
 
 \newpage 
 \section{《 撒哈拉的故事 》 -三毛  }
 \begin{itemize}
 	\item 我盡力願意把我自己的時間,分給每一個關懷我的朋友,可惜的是,我一天也只能捉住二十四小時
 	\item 生活突然的忙碌熱鬧,使我精神上興奮而緊張,體力上透支再透支,而內心的寧靜卻已因為這些感人的真情流露起了很大的波瀾。
 	\item 過去長久的沙漠生活,已使我成了一個極度享受孤獨的悠閒鄉下人,而今趕場似的吃飯和約會,對我來說,就如同劉姥姥進了大觀園,昏頭轉向,意亂情迷。
 	\item 我願意不斷的做一個說故事的人。我不會講什麼大道理,因為我沒有學問,但是,\textbf{我願意在將來的日子裡,仍做不斷的努力,以我的手,寫我的口,以我的口,表達我的心聲。}
 	\item 如果我突然停頓了,那只表示我在培養自己、沉澱自己;在告訴自己:寫,是重要,而有時擱筆不寫,卻是更重要。
 	\item 在台北,我不覺得離你們近,在非洲我也不覺得離你們遠,只要彼此相知欣賞,天涯真是如比鄰啊!
 	\item \textbf{失去了你的個性和作風,我何必娶你呢!}
 	\item 這個啊,是春天下的第一場雨,下在高山上,被一根一根凍住了,山胞紮好了背到山下來一束一束賣了米酒喝,不容易買到哦!
 	\item 都不是,是你釣魚的那種尼龍線,中國人加工變成白白軟軟的了
 	\item 怪名堂真多,如果我們真開飯店,這個菜可賣個好價錢
 	\item 我正要大大宣揚中國人的所謂骨氣,又講不明白,再一接觸到荷西的面部表情,這個骨氣只好梗在喉嚨裡啦
 	\item 荷西含情脈脈的望了我一眼,婚後他第一次如情人一樣的望著我,使我受寵若驚,不巧那天辮子飛散,狀如女鬼。
 	\item 送走老闆,夜已深了,我趕快脫下長裙,換上牛仔褲,\textbf{頭髮用橡皮筋一綁,大力洗碗洗盆,重做灰姑娘狀使我身心自由}
 	\item 荷西將我一把抱起來,肥皂水灑了他一頭一鬍子,口裡大叫:「萬歲,萬歲,你是那隻猴子,那隻七十二變的,叫什麼,什麼——。」我拍了一下他的頭,「齊天大聖孫悟空。這次不要忘記了。」
 	\item 荷西有一個很大的優點,任何三毛所做的事情,在別人看來也許是瘋狂的行為,在他看來卻是理所當然的。所以跟他在一起也是很愉快的事
 	\item 我相信荷西,他過去說出來的事總是做到的。
 	\item 難道荷西先生今天不知道明天自己要結婚嗎?他不知道,我也不知道。」司機聽了看著我,露出好怕的樣子,將車子歪歪扭扭的開走了。我才發覺又講錯話了,他一定以為我等結婚等瘋了。
 	\item \textbf{漫漫的黃沙,無邊而龐大的天空下,只有我們兩個渺小的身影在走著,四周寂寥得很,沙漠,在這個時候真是美麗極了。}
 	\item 
 	\item 
 	\item 
 	\item 
 	\item 
 	\item 
 	\item 
 	\item 
 	\item 
 	\item
 \end{itemize}
 
 \newpage 
 \section{<古代美好的18种交情称谓>  }
 \subparagraph{忤chu臼jiu之交:} 指不分贵贱而交的朋友。
 \subparagraph{患难之交:} 指同在一起共过忧患的朋友。
 \subparagraph{忘年之交:} 指不计年岁长幼而重在德行才能而交的朋友。
 \subparagraph{金石之交:} 指友情深厚如金石般坚固的朋友。
 \subparagraph{胶漆之交:} 志趣相投、亲密无间的朋友。
 \subparagraph{再世之交:} 指与人父子两代都结为朋友。
 \subparagraph{八拜之交:} 旧称异性结拜的兄弟姐妹为八拜之交。
 \subparagraph{车笠之交:} 指不以贵贱而渝的朋友。
 \subparagraph{布衣之交:} 指平民百姓之间的交往。
 \subparagraph{肺腑之交:} 指无语不谈、推心置腹的朋友。
 \subparagraph{君子之交:} 指不分贵贱而交的朋友。
 \subparagraph{莫逆之交:} 指情趣一致、十分要好的朋友。
 \subparagraph{贫贱之交:} 指在贫穷低贱时结交的朋友。
 \subparagraph{平昔之交:} 指往日结交的朋友。
 \subparagraph{忘形之交:} 指彼此以心相许、不拘形迹的朋友。
 \subparagraph{刎颈之交:} 以同生死、共患难的朋友。
 \subparagraph{竹马之交:} 指幼时结交的朋友。
 \subparagraph{总角之交:} 指童年时结交的朋友。
 
 
 \newpage 
 \section{描述女人的美}
 \begin{itemize}
 	\item 苑内似有合欢影,眼前分明胭脂香。 红罗帐下怜娇娥,菱花镜前试新妆。
 	\item 百般难描
 	\item 薄粉敷面
 	\item 白璧无瑕
 	\item 班姬续史之姿,谢庭咏雪之态 
 	\item 步履轻盈,姗姗作响
 	\item 不施粉黛而颜色如朝霞映雪
 	\item 闭月羞花、冰肌莹彻
 	\item 细润如脂,粉光若腻
 	\item 巴东有巫山,窈窕神女颜
 	\item 出水芙蓉、楚楚动人
 	\item 沉鱼落雁、浮翠流丹
 	\item 淡扫娥眉、清喉娇啭
 	\item 淡雅脱俗、齿如含贝
 	\item 灿如春花、皎如秋雨(月)
 	\item 唇色朱樱一点
 	\item 丹唇列素齿,翠彩发峨眉
 	\item 丰姿绰约
 	\item 粉腮红润、秀眸惺忪
 	\item 粉腻酥融娇欲滴
 	\item 丰盈窈窕、芳香袭人
 	\item 风流尔雅
 \end{itemize}
 
 
 \newpage
 
 \section{<走过残酷而美好的时光> - 姗姗}
 周末,最好的朋友夏薇打电话给我。她在遥远的慕尼黑,正在一家很赞的酒馆里和一帮留学生快活地喝酒吃肉。热情的法国小妞喝high了会站起来跳舞。
 
 她对我说:“人生在世,快意二字。邱邱,真希望你也来这里‘放放风’。”
 
 我和夏薇,我们的人生截然不同。
 
 夏薇家境富裕,外表出色,成绩优秀,从小就是命运的宠儿。她并没有在优越的成长环境里恣意。大学毕业后,夏薇以优秀的成绩奔向了慕尼黑大学。
 
 而我和她相反,我没有漂亮的脸蛋,也没有骄人的成绩,更没有显赫的家世。
 
 我是个本分的女孩,本分得现实,只存在合理范围的期待,只许可与实现相符的愿望。
 
 当然,我知道,我的内心并不如行动一样本分。内心世界里的另一个我,很想跳出来,看一看远方。
 
 夏薇,就像一盏小小的明灯。跟她在一起会让我觉得,有一天,我也可以站到傲人的高度。
 
 事实上,我却品尝了太多失败的味道。
 
 柴静说:“失败感比口含硬币还苦。”
 
 \textbf{也许,有些人的成长就是《龟兔赛跑》故事里的那只慢乌龟,在人生的道路上走得颇为缓慢,一路被挫折、沮丧和失落感包围。虽然最终还是走到了终点,但是那毕竟是童话故事。而在现实中的人生旅途,我们不会遇见爱睡懒觉的对手。}
 
 想想也是,这才是童话与生活的区别。
 
 我的整个学生时代都是灰败的。我性格内向,沉默寡言,因为功课不好很难交到朋友,发成绩单时被老师责骂,也只会默默地掉眼泪。
 
 每天放学,我常常在一大群拥挤的接孩子的家长群里艰难穿行,饿着肚子,背着书包慢慢踱步。
 
 有的时候,迷糊如我,会忘记带钥匙。
 
 可我不敢给父母打电话,只好坐在家门口的葡萄藤下写作业。夏天,热风袭人,毛毛虫会偷摸掉下来。秋天,叶子抖落,伴着瓜果香。很多时候,夕阳西下,瑰丽的晚霞柔和地晕满天际,宽阔的天空上颜色不停转换,粉色之后是浅紫,紫色逐渐暗下去,就成了一望无际的深蓝。天空变暗,地上的蚂蚁们头顶着一块块食物,成群结队地往黑黑的洞里赶。
 
 当蚂蚁彻底消失后,妈妈便回来了。
 
 这样的次数一多,她便懒得骂人,只会剜我一眼。
 
 我太明白这种眼神的意义,从小它一直伴着我的成长,让我感到恐慌,它时刻提醒着父母对我有多不满意。
 
 大学毕业后,我便来到了北京。
 
 为什么?
 
 因为它,够大,够远,够繁华。
 
 最重要的是,太多的作品,文学、影视还有歌曲把这座城市渲染得极好,好像这座城市里没有失败的人。即使是失败,也是与众不同,它装着诗意,装着悲壮。
 
 人在北京,很容易交到朋友。因为,只要你连续几次参加同样的饭局,那么,你基本上就可以被纳入某个圈子。熟悉的人在东直门的簋街喝着啤酒,剥麻辣小龙虾,调侃到天明;不熟悉的人一起去吃火锅,火锅袅袅升起的热气遮住了人的脸和眼睛,大家在这样的遮挡下都变得自然起来。只是这样的热闹太不真实,是一群落单的人拼凑在一起,互相取暖,一起共勉,却终究是杯盘狼藉后作鸟兽散,萍水相逢,自是分别匆匆。
 
 漂泊的艰辛一言难尽,虽然满腹埋怨,却又不肯离开。
 
 一个人孤独久了,难免渴望转角遇到爱。刚来到北京,我特别渴望爱情,以为爱能抵挡一切,能遮住生活的平淡苍白。
 时光流逝,我才发现,爱情在北京这样的大都市,太虚弱无力。
 
 这座城市,人人都需要爱情,人人都渴望爱情。然而,穿梭在拥挤、高压的环境里,大家的恋爱也变得非常现实。恋爱是为了一起抵抗大都市的高压生活,恋爱也为了降低房租的成本。你不敢问“你爱不爱我”这样的傻问题,你知道这样的关系很脆弱,所以从来不敢去考验它。这样的爱,真让人泄气。
 
 我也去相过几次亲,几顿饭吃下来,也可以理直气壮地去天涯发帖写一写“这些年,我遇到的极品相亲男”。
 
 你屁股尚未坐稳,A男就告诉你:“这餐,我们得AA制。”
 
 你释放出善意的微笑,B男视如无睹,而是像查户口一样严肃地问:“你月薪多少啊?山东哪里的?我可说了,我妈不许我找农村的。”
 
 你大方埋单时,C男轻松地吁了一口气,突然露出大赦天下的表情:“你是不是处女?如果是的话,我们可以交往试试。”
 ……
 
 我不知道现在的人都怎么了。
 
 过去的年轻人有理想,单纯又善良。男孩靠一把吉他和一首诗就能追到心爱的姑娘。那个时候爱上一个人很简单,不过是:那一天下午阳光明媚,你穿了件白色衬衣,笑容很好看。
 
 这样的单纯美好,怎么都不见了?
 
 在京五年,我时常搬家。
 
 每次拖着破旧又沉重的行李,我就觉得心底很绝望。
 
 二十九岁,青春已经快用光了,我知道这偌大的城市,每天有多少年轻人搭乘县城的老公车坐上火车,兴奋地冲过来。
 在一座城市扎根,尤其是北京这样陌生的城市,过程是缓慢而痛苦的,痛苦很长,快乐很短。很多时候,我觉得坚持不住了,想要逃跑的时候,又很快被它偶尔展现的柔情蛊惑,走不成,又继续重复过往的失意。
 
 但,有的人即使被流放到荒野之地也可以过得像总统一样满足,这是为什么?因为,我们过着怎样的生活,始终与自己的心态有关。生活不可能没有波澜。你以为某段时光最残酷,在艰难走过后才发现,最残酷的往往也是最美丽的;你以为前方是绝路,走过才能看到希望就在转角。
 
 度过了一段很长时间的恐慌期,我终于如愿地与这座城市发生了关系——五环外的一套小居室。虽然,它的首付几乎让我倾家荡产。
 
 师太亦舒说:“很小很小的时候,我们都曾立志,要做一个怎么样怎么样的人,我们都曾天真地以为,只要发愤努力,好好做人,愿望就可以达到。要到很久很久以后才发觉,原来,等待着整治我们的,是命运模子。不管我们愿不愿意,便套将上来挤压,终于,我们忍着疼痛在夹缝中奇怪地存活下来。这时,同我们原来的样子,已有着很大的出入。”
 
 人生,你离开或不离开,都会后悔;你接受或不接受,都会后悔。你逃避,你远离,跳来跳去无非还是在生活里躲猫猫。
 
 只是,每沮丧一次,就会发现,原来过去也不是那么面目可憎,甚至,有的旧时光让你觉得很怀念。人生不进则退,成长岂止是喜、怒、哀、乐几个选项那么简单,迷茫和痛苦更真实而残忍。
 
 人生的路,那么多,那么长,无论你选择哪一条,生活都不会十分完美。重要的,不是你选了哪条路,而是,无论你选择哪一条,要坚持,要一直好好地走下去。
 
 
 \newpage
 \section{<如果常流泪,就不能看见星光> - 三毛}
 
 亲爱的朋友:
 
 \textit{翻阅了将近一整夜的书信,却找不出一两封可以公开回信的题材}。书信专稿原本应该多彩多姿、各色各样才叫美丽活泼,可是手边的来信,归类起来却是如此的相同——千篇一律的抱怨和苦痛,\textbf{好似没有几个 人对自己拥有的生活现况感觉欣赏与赞叹,也少有几个人除了看见自己之外还看见其他的人和事。}
 
 我将回不出的书信放在桌上,走到窗口去站了一会儿,想到书信中一个自找苦痛的生命中,看见高楼下深夜的灯火,心中禁不住要问——难道在这片灯火下的人群真的那么不快乐吗?
 
 \textbf{好似书信中的每一个人都在羡慕他人,每一个人都以为自己的遭遇是人间最不幸的,每一个人都只强烈的抱怨自己的命运甚而怪责社会与家庭,而极少在文字处理中对自己之所以形成今日的局面有所检讨和反省。}
 
 反正自己永远是对的,总而言之,社会和生命是对不起人的。
 
 存着这种心态生活的人,是没有法子通信的,这很难,真的很难,要改也很难,如果自己不改,他人也是没法进言的。
 
 \textit{其实,任何一份生命都有它生长期的创痛与成长的过程,这些过程仿佛是种子,在日后的生活中都会彰显出来,于是我们的生命便在这许多的历练中越见成熟,生命的成熟过程 其实避免不了挣扎和伤感,而生命之美,却也是人间世人加以赋形和圆全的,这十分主观,见仁见智,各有所得。可是,如果只是一味的抱怨,这份在我看来极有价值的存活,便显不出来了。}
 
 有人问过我,人生最重要的是什么?脱口而出的回答是——智慧,后来想了想,觉得不太周全,难道除了智慧之外,快乐不重要吗?真诚不重要吗?金钱不重要吗?爱不重要吗?自由不重要吗?勇气呢?健康呢?家庭呢?友谊呢和了解呢?难道这些都不重要?
 
 我又告诉自己,这一切,其实都已被智慧所涵盖,在智慧的大前提下,其他的东西应该自然而然随之而来的。
 
 “三十六计走为上策”是每一个中国人都知道的计谋之一。如果我们对目前生命的局面不能满意,而且已经尽力而为了,仍然不成,那么为什么不由这一个局面中跨出来,再去开发一个局面呢?许多人说:“我不能。”这句话没有道理。你能,如果你下决心去做,你能的,问题是没有决心就真的不能了,当然,在有计划的开始一个新的局面时,知己知彼却是不可忽视的要素。没有能力去摘月亮的时候,我们便去摘果子吧。不喜欢橘子可以去摘葡萄,不喜欢葡萄还可以去种菜呢。
 
 这封信其实也是写给自己的,也是写给许许多多来信中对上司不满,跟丈夫不和、向社会反抗、同父母争执、与同学处不来……这种种人生普通现象抱怨的朋友们。\textbf{让我们彼此共勉,期许自我的生命接近完美的发现,尽可能减少缺陷的心情,在心灵上脱离一层又一层的束缚,使得生命达到某种程度的自由,而这种自由不是白白便能得来的,如果我们不提升——或说返璞归真,不痛下决心去调查局面,一切都是枉然。}
 
 \textit{《圣经》上说智慧,佛经上也说智慧,我多么愿意自己是一个追求真光的勇者,不怨怪客观环境的一切,尽力将生命的舵交给智慧之星的引导,航向无边无涯的广阔人生。}
 
 亲爱的朋友,包涵吧!尊重吧!这里面包括了对自己的那一份看重。\textbf{偶尔抱怨一次人生可能性是某种情感的宣泄,也无不可,但是习惯性的抱怨而不谋求改变,但是不聪明的人了。}
 
 西班牙有一句谚语:\textbf{“如果常常流泪,就不能看见星光 。”}我很喜欢这句话,所以即使要哭,也只在下午小哭一下,夜间要去看星,是没有时间哭的。再说,我还要去采果子呢。许多来信,在这里做一个总回,同样性质的信,便不再另回了。敬祝  安康!	 
 
 
 \newpage
 \section{爱情需要注意的一些心得}
 \begin{enumerate}
 	\item \textbf{底牌亮的太快太早}-从开始就恨不得掏心窝子的爱人多半是要被抛弃的。
 	
 	别人是尘封的老酒,越喝越有味,你是开听的可乐,到后面气儿都没了。
 	\item 有底线-这句话不是说要没底线,而是\textbf{你和一个人在一起本来就应该是相互扶持安慰共同享受生活的,对方把你逼到开始预设底线了。还谈什么呢?}
 	
 	\item 人大多是恃宠而骄的,\textbf{别把对方当孩子养}-好端端的伴侣供成了神。
 	
 	\item 没有各自空间-\textbf{自己不给自己空间,自然也给不了对方空间}。
 	
 	\item 算计-爱多爱少,\textbf{付出多付出少,算到这一步也就该算还有多少日子可以走了}。
 	
 	\item 双重标准-Ta 生气你哄,你生气还得你哄,自求多福,\textbf{不平等条约签订是因为爱情么}?
 	
 	\item 口无遮拦-\textbf{别拿什么爱你才会跟你发脾气这样的理由妄图诠释自己的自私与SB行为}。
 	
 	\item 话说太透-\textbf{摆在台面说的话,规矩是完善了,情分却少了}。
 	
 	\item 三观不合-其实人与人三观是有差异的,真正不合的人决定开始就是错误的,最为可气的是甚至某一方完全看不起另一方的观念,表现形式就是不管什么情况,你怎么表现都只会听到Ta痛心疾首的说-“你不要太幼稚了,好吗”
 	
 	\item 秀恩爱,死得快- 其实我不反感晒个情侣照的,受不了的是喜欢在众人面前以呼喝对方为荣,以对方听话为恩爱的表现方式,你这么不给人面子你是作死你造不?
 	
 	\item 不信任- 其实我想吐槽的不是某个人不信任对方还怎么谈恋爱,诚然过于敏感人士需要自省。\textbf{但是正常情况下你都不能让你的伴侣信任你,你是得有多糟糕?}道理不多讲,好好想想不被信任的原因再去对着Ta吼“你为毛不信任我?”
 	
 	\item 永远永远永远别认为自己在付出!!-一旦觉得自己在付出你就注定输了整盘棋,\textbf{心态不好势必行为也不能正能量到哪里去}。
 	
 	\item 不甘心-我们没那么优秀,对方也没那么差劲。\textbf{不要总盯着爱人的缺点,忍让不是包容,理解和宽容才能让彼此都感觉舒服幸福}。
 	
 	\item 距离-离伴侣的好友保持在你的伴侣觉得你们两根本不熟的位置就好,既不会有姐妹 淘私下 聚会时太多关于你的八卦,也不会引起伴侣不适,男女皆适用。
 	
 	\item 莫相求-求来的不是爱,是神,\textbf{真想要维护和挽回就去学会吸引对方}。
 \end{enumerate}
 
 \newpage
 \section{<一棵大树> - 谢尔.西弗斯汀}
 从前有一棵树,她很爱一个男孩。每天,男孩都会到树下来,把树的落叶拾起来,做成一个树冠,装成森林之王。 
 
 有时候,他爬上树去,抓住树枝荡秋千,或者吃树上结的果子。有时,他们还在一块玩捉迷藏。要是他累了,就在树荫里休息,所以,男孩也很爱这棵大树。 
 
 树感到很幸福。 
 
 日子一天天过去,男孩长大了。树常常变得孤独,因为男孩不来玩了。 
 
 有一天,男孩又来到树下。树说:“来呀,孩子,爬到我的树干上来,在树枝上荡秋千,来吃果子,到我的树荫下来玩,来快活快活。” 
 
 “我长大了,不想再这么玩。”男孩说:“我要娱乐,要钱买东西,我需要钱。你能给我钱吗?” 
 
 “很抱歉,”树说,“我没钱。我只有树叶和果子,你采些果干去卖吧,卖到城里去,就有钱了,这样你就会高兴的。” 
 
 男孩爬上去,采下果子来,把果子拿走了。 
 
 树感到很幸福。 
 
 此后,男孩很久很久没有来。树又感到悲伤了。 
 
 终于有一天,那男孩又来到树下,他已经长大了。树高兴地颤抖起来,她说:“来啊,男孩,爬到我的树千上来荡秋千,来快活快活。” 
 
 “我忙得没空玩这个。”男孩说,“我要成家立业,我要间屋取暖。你能给我间屋吗?” 
 
 “我没有屋,”树说:“森林是我的屋。我想,你可以把我的树枝砍下来做间屋,这样你会满意的。” 
 
 于是,男孩砍下了树枝,背去造屋。 
 
 树心里很高兴。 
 
 但男孩又有好久好久没有来了。有一天,他又回到了树下,树是那样的兴奋,连话都说不出来了,过了一会,她才轻轻地说:“来啊,男孩,来玩。” 
 
 “我又老又伤心,没心思玩。”男孩说:“我想要条船,远远地离开这儿。你给我条船好吗?” 
 
 “把我的树干锯下来做船吧。”树说:“这样你就能离开这里,你就会高兴了。” 
 
 男孩就把树干砍下来背走,他真的做了条船,离开了这里。 
 
 树很欣慰,但心底里却是难过。 
 
 又过了好久,男孩重又回到了树下。树轻轻地说:“我真抱歉,孩子,我什么也没有剩下,什么也不能给你了。” 
 
 她说:“我没有果子了。” 
 
 他说:“我的牙咬不动果子了。” 
 
 她说:“我没有树枝了,你没法荡秋千。” 
 
 他说:“我老了,荡不动秋千了。” 
 
 …… 
 
 树低语说:“我很抱歉。我很想再给你一些东西,但什么也没剩下。我只是个老树墩,我真抱歉。” 
 
 男孩说:“现在我不要很多,只需要一个安静地方坐一会儿,歇一会儿,我太累了。” 
 
 树说:“好吧,”说着,她尽力直起她的最后一截身体,“好吧,一个老树墩正好能坐下歇歇脚,来吧,孩子,坐下,坐下休息吧。” 
 
 于是男孩坐在了树墩上。
 
 \newpage
 \section{<别人讨厌你,并不是因为你不会说话> - 网摘}
 有一天,下班早,正好赶上一个小学放学。孩子们成群结队地涌出校门,正要奔向等候多时的家长,一个老师突然拦住了最前面的同学,要大家一起高喊“老师再见”才让他们离开。
 
 原来,有些规矩几十年都不会变。在队伍的角落里有一个小男孩,站在队伍里格格不入,因为他的紧皱的眉头和阴郁凝重的表情,大概是因为不喜欢这样的仪式。
 
 和老师道别之后,是小朋友们互相道别,大部分小朋友恋恋不舍地拥抱之后,投入了父母的怀抱。唯独这个小男孩,默默地绕过人群,走到妈妈身旁,也没有人理他。他妈妈附身轻轻地问他:“你怎么不和小朋友道别呢?”小男孩不说话,扎进妈妈怀里,妈妈急忙跟身旁另一个孩子的家长解释,“我们家孩子就是不爱说话。”
 
 身边的很多妈妈都特别担心自己的孩子不会说话,所以从小就循循善诱地告诫他们,一定要多和老师同学交流。所以,从小,我们就学了能言善道,因为这样可以被喜欢。如果一个孩子不合群,被排挤,大部分家长也会像这个妈妈一样做出一种解读:我的孩子没有朋友,是因为他不会说话。
 
 然而,事实上,我却看到很多不善言谈的孩子身边簇拥着亲密的伙伴,他们不会说好听的话,但每次有了好吃的都会默默地给小伙伴留一份,酷酷的样子让人爱极了。
 
 前阵子,公司IT部的一个男同事哭丧着脸来找我诉苦,说自己一把年纪了还没个像样的女朋友,父母一直催促他早点结婚。这个八零后的大男孩人很好,也很随和,钱虽然赚的不多,但也不是硬伤。之前有过几个相处的女孩,但最后也都无疾而终。我其实有点不理解,这个世界上再穷再丑的人都有人爱,真的看不出他有什么找不到女朋友的理由。他总说是因为自己不会说话,不会主动追女孩,即使在朋友和媒人的帮助下,有了一点进展,见过几次面,姑娘那边又大多没了音讯。
 
 他来找我,其实不是为了抱怨,而是为了一个女孩可可。可可是我的邻居,我知道她单身了很多年,一直盼望着一个爱她的人。可可是个安静的女孩,我觉得那些侃侃而谈的男孩并不适合她。所以,我把这个男同事介绍给了可可。见了几次面,他就很喜欢可可,但是可可却有点退缩。他还是觉得因为嘴笨,不会哄人,也不会讨女孩欢心。我不相信,起码可可不是。
 
 那天晚上,我专程到可可家去找她聊天,想知道真正的原因。她跟我说,有一次,他开车带可可出去吃晚餐,因为正值下班的高峰,拥堵的车流和横冲直闯的行人让他变得很焦躁,他不停地按喇叭,嘴里不停地咒骂那些不守规矩的行人。好不容易到了餐厅,却发现已经排了很长的队。这一次,他更加不耐烦了, 险些跟领位的服务员吵起来。
 
 我终于明白为什么可可会拒绝他,每次约会我迟到,可可都是安静地坐在那里翻着一本书,丝毫没有在等我的样子,每次我因为一件小事变得急躁的时候,她都劝我放轻松。
 
 我知道,他们不可能在一起了。
 
 可可说到等位的时候,我忽然想到上大学时一个很受欢迎的男同学。那时,并不流行沉默的美男子,大部分女生还是喜欢会讲笑话的男孩。但他却是个例外,沉默寡言,却总是有人示爱。直到有一次我们一起做校刊,我才明白,这个世界没有毫无缘由的爱。每次熬夜赶稿,他都是最后一个离开,把零乱的教室整理好。每次聚餐,他也没都提前好几天把位子订好,那些不能订餐的地方,他都早早地去占位,从来不怕等人的尴尬。还有一次,我看见他把打包精美的稻香村礼盒拿给外地的女朋友,应该是让她带回去孝敬父母的礼物。我猜想,他从不说要为你做什么,而是直接就做了。
 
 \textbf{语言,是一种表达方式,但却不是唯一的表达方式。比起语言,行为更能真实地反映一个人的品性。每次走在街上,看到男孩特意走到女孩的左边,只是为了挡在她和车流之间,我都觉得很温暖。再动听的话可能都比不上那一刻带来的安全感。可是还有人傻傻地以为人生遭遇的拒绝只是因为不会说话。}
 
 后来我又发现,原来这个男同事在公司里的人缘很差。整个公司除了个别人,比如我,几乎没有人愿意跟他聊天。他总是给自己一个理由,自己遭到的排挤,得不到升迁,都是因为自己内向。
 
 可是,明明有那么多不善交往的同事得到了老板的青睐,也有那么多爱讲冷笑话的同事得到了大家的喜爱,他们从不苛责别人尽善尽美,也不会提出无理的要求,而他呢?每天下班第一个走,周末的加班从不愿分担,却总以为是他不会说话才被大家拒之门外。
 
 内向,原本是一种独特的气质,如今却成了一种缺陷,人们害怕承认自己的内向,反而努力地去伪装可以谈笑风生。有些感觉,无法分享,这并没有什么可耻或者可悲。\textbf{不善言谈或许会阻碍你结交很多朋友,但那些懂的人会始终留在你的身边。}
 
 但如果你固执地认为自己是某一类人,你就会用这类人的标准界定自己的人生。而当你身上的某种特质被认为是缺陷时,你就会习惯性地躲避在这所谓的缺陷里,责怪它摧毁了你的生活。
 
 \textbf{一个人不被喜欢,需要一个理由。于是,你找到了那些“缺陷”,给了自己一个合理的解释,便心安理得。}
 
 \textbf{然而,别人讨厌你,并非因为你不善言谈,而是因为你的修养不够,还有你从来不为他人着想。}
 
 
 \newpage
 \section{<你说话直?对不起我很介意!> - 值得反思的一篇网摘}
 有些话,说出来就是一场祸。
 
 有些话,一开口就能点着火。
 
 \textbf{别把你的情商低和说话直混为一谈。}
 
 否则我容忍傻逼的方式,兴许就是抬手一巴掌。
 
 女生们一定有过这样的体验:
 
 周末三两人去逛街,好不容易相中一条裙子,问大家意见如何。结果,其中一个不算熟的朋友突然来了一句:我这人说话直,你别介意啊。
 
 你心里咯噔一下,没来得及拒绝,对方就弹珠炮弹一样的把后面的话抛了出来:这条裙子不显瘦,你那么肥,应该换条黑色的。另外你有点矮,皮肤也比较黑,其实穿裤子是最好的,裙子的话最好别试。
 
 话一出,就连站在一边的店老板脸色都僵硬了。
 
 \textbf{“哎呀,不会不会,这条裙子很适合你啦~”余下的朋友给你打打圆场,结果“说话直”的她又来了一句:唉,我这些话你可能不爱听,但我说的都是实话。而此刻的你怒火汹涌,恨得牙龈发痒,真恨不得大庭广众下直接扇她一个耳光。}
 
 
 
 那我打人也挺疼的,你可别喊疼啊。
 
 \textbf{生活中什么都缺,最不缺的就是把这种情商低当成说话直的贱人。}
 
 你跟他去吃饭,有点饿,点多了点东西。他就说你这么胖了还吃这么多。
 
 你去做个头发,回来之后他就冷嘲热讽。说本来就难看了现在更丑。
 
 你不管做什么,他都会用一句“说话直”来打击你。
 
 \textbf{他习惯了和所有人对着干。}
 
 \textbf{坐下来聊天,一旦有人发表了一点自己的观点,他就会条件反射般地反驳。}
 
 \textbf{他有本事把一次好端端的下午茶变成一场辩论会,到最后你是在不想跟他辩驳了,他就露出那种“果然如此”的神情,心里头还自鸣得意。}
 
 \textbf{这种人,动不动就说“不,我是这样认为的…”,动不动就强行灌输自己的观点,顺带鄙视一下别人。末了还来一句:我这人说话直,你别介意。}
 
 呵呵。
 
 什么时候说话直等同于一项自我辩解的武器了?
 
 拜托你,请不要用“说话直”这三个字来开脱自己因为情商低而对别人造成伤害,你的存在简直玷污了我们的眼睛。
 
 
 大学时代里遇见不少这样的人。
 
 去个图书馆,问室友要不要一起去,他说去图书馆干嘛?别假积极了,你那样子也不是什么学习的料,装给谁看?等着挂科算了。
 
 和同班的女生刚好有事,约出去吃个饭,他又开始笑话:算了吧,看看你那样子,这辈子相亲能找到老婆就算不错了,还约女生出去,呵呵,去哪里吃啊?打个包呗。
 
 他的口头禅就是“我这个人就是说话直,没啥坏心眼。”
 
 你性子那么直,怎么不见得你看辅导员不爽的时候冲上去骂她几句?
 
 你性子急,怎么不见你路见不平拔刀相助?
 
 玩笑话归玩笑话,但倘若你把“说话直”“性子直”作为旗号来掩饰自己情商低的事实,那对不起,我没有你这样的朋友。
 
 
 
 生活中很多原本可以相谈甚欢而止步于“说话直”的场面。
 
 太多了。
 
 朋友圈有女生发:工作太辛苦啦,不知何时是个头。说话直的人直接评论:你们女生找个好人家嫁了不就行了,谈什么工作。着实是话题终结者,估计那头的女生被气得可以。
 
 我女朋友,好看不?——切,真丑。
 
 这我从老家给你们带来的特产,吃一下看好吃不?——靠,真难吃
 
 唉,看来我这次考证又过不了啦。——你的智商低,能考过才怪。
 
 你看看这照片,我爱豆帅吗?——丑,追星的都是傻逼。
 
 朋友,你可知道,你说话不是直,你这是脑袋缺根筋啊!
 
 \textbf{我们生活中常说的说话直是什么意思?说话直,首先你的出发点要是好的,是为了别人而提出来的,其次,你要客观公正的指出一件事的两面性,而不是随意的反驳别人,嘲笑别人。}
 
 你失恋了,看到别人在热恋中,你就恨不得告诉她男人没有一个好东西,因为你不说出来心里不爽。你看到别人买东西你就忍不住要敲击她。你忍不住要去反驳别人的观点。你在大庭广众之下不给别人面子。你把自己过分直接的没礼貌,当做别人开不起玩笑。
 以上,统统都是傻叉
 
 
 周末约朋友出来喝东西,他抱怨了他们单位最近新来的一个小女生。刚刚大学毕业不久,因为学历比较高的缘故,他总觉得自己在单位里屈才了,所以对什么东西都看不顺眼。
 
 朋友说,她是多读了几年书,硕士毕业,但她在工作上的问题简直处理的一团糟!
 
 大家去吃饭,相互意思意思敬一杯酒,她在下面冷嘲热讽:“你们可真会来事,我就不行,反正我喝不了酒,这种虚情假意我看还是算了吧。”虽然说的很小声,但大家的气氛一瞬间还是有点尴尬。
 
 她总是说:“我说话直,因为我没什么弯弯肠子,凭实力说话”\textbf{你凭实力说话,到头来你还是要有那个实力才行。工作上的事都处理不好,哪里来的实力?不过是情商低罢了。最近又听朋友说,因为她说话总是四处得罪人,所以被开除了。}
 
 \textbf{你情商低,不懂得尊重别人,最后祸害的,不过是你自己。}
 
 有人会问,生活中常常遇到这种以“说话直”为借口,四处用言语伤人的人,应该怎么?
 
 也许只有一个办法,不用给他任何脸面,直接拒绝他。
 
 “有一句话,不知道你愿不愿意听…”
 
 “不愿意。”
 
 “我说话比较直,你不要介意哦。”
 
 “对不起,我很介意。”
 
 “你怎么这么开不起玩笑?”
 
 “不好意思,就是开不起玩笑。”
 
 不要不好意思拒绝。倘若你一时心软,接下来遭殃的大概就是你自己。凭什么要惯着他?一巴掌摁住他,别让他说话。
 
 到头来,远离傻×才能珍爱自己。
 
 \newpage
 \section{我爱你的多种表达}
 \begin{itemize}
 	\item 周星驰: 我养你呀
 	\item 苏轼: 不思量,自难忘
 	\item 黄伟文: 余生请你指教
 	\item 王家卫: 那一刻,。我很暖
 	\item 夏目漱石:今夜月色真美
 	\item 张学友: 很想带你去吹吹风
 	\item 玛格丽特: 我在床上,饭在锅里
 	\item 范仲淹:酒入愁肠,化作相思泪
 	\item 李白: 郎骑竹马来,绕床弄青梅
 	\item 张爱玲: 你还不来,我怎敢老去。
 	\item 钱武萧王:陌上花开,可缓缓归矣
 	\item 方文山: 天青色在等你,而我在等你
 	\item 元稹: 曾经沧海难为水,除却巫山不是云
 	\item 张国荣: 就让我陪你唱一辈子戏,不行么?
 	\item 王小波: 你好哇,李银河,见到你真高兴
 	\item 李之仪: 只愿君心似我心,定不负相思意
 	\item 柳永: 衣带渐宽终不悔,为伊消得人憔悴
 	\item 林夕: 你是我这一生等了半世未拆的礼物
 	\item 李商隐: 直道相思了无益,未尝惆怅是轻狂
 	\item 冯唐: 春水初生,春林初盛,春风十里,不如你
 	\item 鲁迅: 我爱子君,仗着她逃出这寂静和空虚
 	\item 卓别林: 我可以选择让你看见,也可以选择坚持不让你看见
 	\item 顾城: 草在结它的种子,风在摇它的种子,我们站着,不说话,就十分美好
 	\item 沈从文: 我行过许多地方的桥,看过许多次数的云,喝过许多地方的酒,却只爱过一个正当最好年龄的人。
 	\item 黄桦: 如果你是我的阳光,我给你整个的蓝天。
 	\item 徐志摩: “一生至少该有一次,为了某个人而忘了自己,不求有结果 ,不求同行,不求曾经拥有,甚至不求你爱我,只求在我最美的年华里,遇到你 。”
 \end{itemize}
 
 \newpage
 \section{《 简爱.Jane.Eyre 》 - 夏洛蒂·勃朗特}
 简·爱是个孤女,她出生于一个穷牧师家庭。不久父母相继去世。
 
 幼小的简·爱寄养在舅父母家里。舅父里德先生去世后,简·爱过了10年倍受尽歧视和虐待的生活。舅母把她视作眼中钉,并把她和自己的孩子隔离开来,从此,她与舅母的对抗更加公开和坚决了,简被送进了罗沃德孤儿院。
 孤儿院教规严厉,生活艰苦,院长是个冷酷的伪君子。简·爱在孤儿院继续受到精神和肉体上的摧残。由于恶劣的生活条件,孤儿院经常有孩子病死,她最好的朋友海伦患肺结核去世。海伦去世后也使孤儿院有了大的改善。简·爱在新的环境下接受了六年的教育,并在这所学校任教两年。由于谭波尔儿小姐的离开,简·爱厌倦了孤儿院里的生活,登广告谋求家庭教师的职业。
 
 桑菲尔德庄园的女管家聘用了她。庄园的男主人罗切斯特经常在外旅行,她的学生是一个不到10岁的女孩阿黛拉·瓦朗,罗切斯特是她的保护人。
 
 一天黄昏,简·爱外出散步,邂逅刚从国外归来的主人,这是他们第一次见面。以后她发现她的主人是个性格忧郁、喜怒无常的人,对她的态度也是时好时坏。整幢房子沉郁空旷,有时还会听到一种令人毛骨悚然的奇怪笑声。一天,简·爱在睡梦中被这种笑声
 惊醒,发现罗切斯特的房间着了火,简·爱叫醒他并帮助他扑灭了火。
 
 罗切斯特回来后经常举行家宴。在一次家宴上向一位名叫英格拉姆的漂亮小姐大献殷勤,简·爱被召进客厅,却受到布兰奇母女的冷遇,她忍受屈辱,离开客厅。此时,她已经爱上了罗切斯特。其实罗切斯特也已爱上简·爱,他只是想试探简·爱对自己的爱情。当他向简·爱求婚时,她答应了他。
 
 在婚礼前夜,简·爱在朦胧中看到一个面目可憎的女人,在镜前披戴她的婚纱。
 
 第二天,当婚礼在教堂悄然进行时,突然有人出证:罗切斯特先生15年前已经结婚。他的妻子原来就是那个被关在三楼密室里的疯女人。法律阻碍了他们的爱情,使两人陷入深深的痛苦之中。
 
 在一个凄风苦雨之夜,简·爱离开了罗切斯特。
 
 在寻找新的生活出路的途中,简·爱风餐露宿,沿途乞讨,历尽磨难,最后在泽地房被牧师圣·约翰收留,并在当地一所小学校任教。不久,简·爱得知叔父去世并给她留下一笔遗产,同时还发现圣·约翰是她的表兄,简·爱决定将财产平分。圣·约翰是个狂热的教徒,打算去印度传教。他请求简·爱嫁给他并和他同去印度,但理由只是简·爱适合做一位传教士的妻子。
 
 简·爱拒绝了他,并决定再看看罗切斯特。她回到桑菲尔德庄园,那座宅子已成废墟,疯女人放火后坠楼身亡,罗切斯特也受伤致残。
 
 简·爱找到他并大受震动,最终和他结了婚,得到了自己理想的幸福生活。
 
 \subparagraph{美句}
 \begin{itemize}
 	\item  假如你避免不了,就得去忍受。不能忍受生命中注定要忍受的事情,就是软弱和愚蠢的表现。
 	\item  你难道认为,我会留下来甘愿做一个对你来说无足轻重的人?你以为我是一架机器?——一架没有感情的机器?能够容忍别人把仅有的一口面包从我嘴里抢走,把一滴生命之水从我杯子里泼掉?就因为我一贫如洗,默默无闻,长相平庸,个子瘦小,就没有灵魂,也没有心了吗?——你想错了。我的心灵跟你一样丰富,我的心胸跟你一样充实!要是上帝赋予我一点姿色和充足的财富,我会使你同我现在一样难分难舍!我不是根据习俗,常规,甚至也不是血肉之躯同你说话,而是用灵魂同你的灵魂在对话,就好像我们两个人穿过坟墓,站在上帝脚下,彼此平等——本来就如此!
 	\item  我越是孤独,越是没有朋友,越是没有支持,我就得越尊重我自己。
 	\item  生命太短暂了,没时间恨一个人那么久。Life is too short to continue hating anyone for a long time.
 	\item  即使整个世界恨你,并且相信你很坏,只要你自己问心无愧,知道你是清白的,你就不会没有朋友。
 	\item  我无法控制自己的眼睛,忍不住要去看他,就像口干舌燥的人明知水里有毒却还要喝一样。我本来无意去爱他,我也曾努力的掐掉爱的萌芽,但当我又见到他时,心底的爱又复活了。
 	\item  谁说现在是冬天呢?当你在我身旁时,我感到百花齐放,鸟唱蝉鸣。
 	\item  生命太短暂了,不应该用来记恨。人生在世,谁都会有错误,但我们很快会死去。我们的罪过将会随我们的身体一起消失,只留下精神的火花。这就是我从来不想报复,从来不认为生活不公平的原因。我平静的生活,等待末日的降临。
 	\item  假如刮一阵风或滴几滴雨就阻止我去做这些轻而易举的事情,这样的懒惰还能为我给自己规划的未来作什么准备呢?
 	\item  暴力不是消除仇恨的最好办法——同样,报复也绝对医治不了伤害。
 	\item  耐心忍受只有自己感到的痛苦,远比草率行动,产生恶果要好。
 	\item  在你未来的人生道路上,你常常会发现不由自主地被当作知己,去倾听你熟人的隐秘。你的高明之处不在于谈论你自己,而在于倾听别人谈论自己。
 	\item  第一次报复人,我尝到了滋味,像喝酒似的。刚一喝,芬芳甘醇,过后却满嘴苦涩。
 	\item  我会带着不倦的温柔体贴,在你身边走动,尽管你不会对我报之以微笑;我会永不厌腻的盯着你的眼睛,尽管那双眼睛已不再射出一缕确认我的光芒。
 	\item  要是你无法避免,那你的职责就是忍受,如果你命里注定需要忍受,那么说自己不能忍受 就是软弱就是犯傻。
 	\item  没有判断力的感情的确淡而无味,但未经感情处理的判断力又太苦涩、太粗糙,让人无法下咽
 	\item  真正的世界无限广阔,一个充满希望与忧烦,刺激与兴奋的天地等待着那些有胆识的人,去冒各种风险,追求人生的真谛。
 \end{itemize}
 
 \newpage
 \section{一些可以培养的习惯}
 \begin{enumerate}
 	\item 记录时间花销\\
 	当你记录一段时间之后,才能真正认识到自己的时间到底哪去了,才能真正认识到一天中真正花在有意义事情的时间少得可怜,一不小心就会浪费很多时间。而你在你的梦想上得倾注足够的时间才行。		
 	\item 记账\\
 	很多理财课程的第一步就是记账,当你明白你的钱具体花费在哪些,哪些应该花,哪些不应该,就开始了财务自由的第一步。
 	\item 定期运动\\
 	一旦当你养成这个习惯之后,你会发现这些习惯会影响你的整个生活,让你的世界完全不一样。有运动习惯的人,整个人的精气神都会不一样,并且能帮助你改变很多坏的生活习惯,例如吸烟、拖延
 	\item 早起\\
 	有效的利用早起的时间,并把早起多出来的大块时间好好利用起来,早起才会对你产生影响和改变。早起的前提是早睡,充足的睡眠时间是早起的重要保证
 	\item 定期总结\\
 	工作时所遇到的问题,包括工作中学到的经验,定期花时间去分析、总结、思考,你工作一年时间可能比别人几年的经验还要强,此方法在几乎所有的行业和岗位中都适用。
 	\item 每年订一个计划挑战自己\\
 	每年设定几个必须完成的目标,然后安排时间计划全力完成。开始了就坚持下去,虽会有成功,有失败,但每一个习惯的坚持都对自己有改变和影响。
 	\item 学习一项新的技能\\
 	生命不息、学习不止。学习一项新的技能,就开始了一种新的体验世界的方式,同时也会给固定的生活注入乐趣与生机。
 	\item 有意识的锻炼自己的注意力\\
 	过于忙碌的人都有一个特征:注意力不够而导致认知能力和判断力的下降。而注意力的强弱会直接决定你的工作质量,学习能力以及生活幸福度。
 	\item 阅读\\
 	安排定期的阅读时间和计划,通过对文章的理解,领悟,鉴赏,探索的思维过程,日积月累你会发现,书中深邃的思想,丰富的内容,高尚的品格..会让一个人获得极大的能量
 \end{enumerate}
 
 \newpage 
 \section{一些诗}
 \subsection{腹有诗书气自华 - 《 和董传留别 》   -苏轼   }
 
 \textbf{粗缯大布裹生涯,腹有诗书气自华。}
 
 厌伴老儒烹趏叶,强随举子踏槐花。
 
 囊空不办寻春马,眼乱行看择婿车。
 
 得意犹堪夸世俗,诏黄新湿字如鸦。
 
 \subsection{人生几回伤往事 - 《西塞山怀古》   -刘禹锡} 
 
 王睿楼船下益州,金陵王气黯然收。
 
 千寻铁锁沉江底,一片降幡出石头。
 
 \textbf{人世几回伤往事,山形依旧枕寒流。}
 
 今逢四海为家日,故垒萧萧芦荻秋。
 
 \subsection{情人眼里出西施 -《 集杭州俗语诗 》  -   黄增  }
 
 色不明人人自迷,情人眼里出西施。
 
 有缘千里来相会,三笑徒然当一痴。
 
 \subsection{夕阳无限好,只是近黄昏  - 李商隐}
 
 向晚意不适,驱车登古原。
 
 夕阳无限好,只是近黄昏。
 
 \subsection{春宵一刻值千金 - 苏轼}
 
 春宵一刻值千金,花有清香月有阴。
 
 歌管楼台声细细,秋千院落夜沉沉。
 
 \newpage 
 \section{诸葛教子书}
 \paragraph{宁静的力量:}
 
 \textbf{夫君子之行,静以修身,俭以养德。}
 
 宁静才能够修养身心,静思反省。处在平和状态时,孩子不仅能主动完成很多活动,而且非常专注。
 
 俭以养德:忠告孩子要节俭,审视理财,量入为出,才能避免成为物质的奴隶。
 \paragraph{淡泊的力量:}
 
 \textbf{非淡泊无以明志,非宁静无以致远。}
 
 忠告孩子不要过分讲求名利,静心反思,才能了解自己的志向。	
 \paragraph{学习的力量:}
 
 \textbf{夫学须静也,才须学也。}
 
 才能是学习的结果,若心境专注,则事半功倍。
 \paragraph{立志的力量:}
 
 \textbf{非志无以成学。}
 
 告诫孩子要立志,要有决心和毅力,避免半途而废。
 \paragraph{速度的力量:}
 
 \textbf{淫漫则不能励精}
 
 凡事拖延就会懈怠。
 
 \paragraph{性格的力量:}
 
 \textbf{险躁则不能治性。}
 
 太急躁不仅很难成功,还会危害身心健康。
 
 \paragraph{时间的力量:}
 
 \textbf{丰与时驰,意与日去,遂成枯落}
 
 时光飞逝,意志力会随时间消磨,因此要管理自己的时间,善用每分每秒。
 
 如果虚度年华,消磨时光,最终会像枯枝落叶般衰老下去。
 \paragraph{精简的力量:}
 
 \textbf{大道至简}
 
 精简沟通比长篇大论更有效。讲道理要言之有物,而非不停唠叨。	
 \newpage 
 \section{《 走在人生边上 》- 杨绛 }
 \section{前言的鬼神存在论}
 我试图摆脱一切成见,按照合理的规律,合乎逻辑的推理,依靠实际生活经验,自己思考。我要从平时不在意的地方,发现问题,解答问题;能证实的予以肯定,不能证实的予以肯定,不能证实的存疑。这样一步一步自问自答,看能探索多远。
 好在我是一个平平常常的人,无党无派,也不是教徒,没什么条条框框干碍我思想的自由。
 
 \newpage
 \section{语录摘要}
 \paragraph{1.鲁迅}
 
 \begin{itemize}
 	\item 希望是无所谓有无所谓无的,这正如地上的路,其实地上本没有路,走的人多了也就成了路。
 \end{itemize}
 
 \paragraph{2.万能的大熊}
 \begin{itemize}
 	\item 有一天我突然想,我还在上学的时候他们就在社会里挣扎奋斗了。他们在社会上奋斗积累了十几二十年,我们新人来了,他们有的我都想要,我这是不公平,我这是抢劫。 
 	
 	因为我要得太急,因为我忍不住寂寞
 	
 	\textbf{20多岁的男人,没有钱,没有事业,却有蓬勃的欲望。}
 \end{itemize}
 
 \paragraph{3.村上春树}
 \begin{itemize}
 	\item 终点线只是一个记号而已 ,其实并没有什么意义,关键是这一路你是如何跑的..人生也是如此
 	
 	\item 突然有一天,我出于喜欢开始写小说,又有一天,我出于喜欢开始在马路上跑步。不拘什么,按照喜欢的方式做喜欢的事,我就是这样生活的。—— 村上春树
 	
 	\item 哪里会有人喜欢孤独,不过是不喜欢失望罢了。<挪威的森林>
 	
 	\item 正值青春年华的我们,总会一次次不知觉望向远方,对远方的道路充满憧憬,尽管忽隐忽现,充满迷茫。有时候身边就像被浓雾紧紧包围,那种迷茫和无助只有自己能懂。\textbf{尽管有点孤独,尽管带着迷茫和无奈,但我依然勇敢地面对,因为,这就是我的青春,不是别人的,只属于我的。}——村上春树《挪威的森林》
 	
 	\item <舞舞舞>,<萤火虫>,<国境以南太阳以西>
 	
 	\item 有的东西不过很久是不可能理解的,有的东西等到理解了又为时已晚。大多时候,我们不得不在尚未清楚认识自己的心的情况下选择行动,因而感到迷惘和困惑。——村上春树《世界尽头与冷酷仙境》
 	
 	\item 没有专注力的人生,就仿佛大睁着双眼却什么也看不见。 —— 村上春树《眠》
 	
 	\item 不管全世界所有人怎么说,我都认为自己的感受才是正确的。无论别人怎么看,我绝不打乱自己的节奏。喜欢的事自然可以坚持,不喜欢怎么也长久不了。	——村上春树 《当我谈跑步时我谈些什么》
 	
 	\item 少年时追求激情,成熟后却迷恋平静,在我们寻找,伤害,背离之后,还能一如既往的相信爱情,这是一种勇气。每个人都有属于自己的一片森林,迷失的人迷失了,相逢的人会再相逢。 ——村上春树《不是相爱便是相逢》
 	
 	\item 不要因为寂寞随便牵手,然后依赖上,人自由自在多好,纵使漂泊,那种经历也好过牢狱般的生活,所以我刻意不让自己对网络太依赖,对失去的人也保持淡然的态度,数千个擦肩而过中,你给谁机会谁就和你有缘分,纵没有甲,也会有乙。——村上春树《挪威的森林》
 \end{itemize}   
 
 \paragraph{4.路遥}
 \begin{itemize}
 	\item 人之所以痛苦,在于追求错误的东西。如果你不给自己烦恼,别人也永远不可能给你烦恼。 因为你自己的内心,你放不下。 好好的管教你自己,不要管别人。<平凡的世界>
 \end{itemize}
 
 
 \paragraph{5.亦舒}
 \begin{itemize}
 	\item  无论怎么样,一个人借故堕落总是不值得原谅的,越是没有人爱,越是要爱自己。<星之碎片>
 \end{itemize}
 
 \paragraph{6.高晓松}
 \begin{itemize}
 	\item  生活不止眼前的苟且,还有诗和远方的田野
 \end{itemize}
 
 \paragraph{7.三毛}
 \begin{itemize}
 	\item 如果有来生,要做一棵树,站成永恒,没有悲欢的姿势。一半在尘土里安详,一半在风里飞扬,一半洒落阴凉,一半沐浴阳光。非常沉默非常骄傲,从不依靠从不寻找.   《说给自己听》
 	\item 如果有来生,要做一棵树,站成永恒,没有悲欢的姿势。一半在尘土里安详,一半在风里飞扬,一半洒落阴凉,一半沐浴阳光。如果有来生,要做一只鸟,飞越永恒,没有迷途的苦恼。东方有火红的希望,南方有温暖的巢床,向西逐退残阳,向北唤醒芬芳。
 	\item 非常沉默,非常骄傲,从不依靠,从不寻找。如果有来生,有没有人爱,我也要努力做一个可爱的人。不埋怨谁,不嘲笑谁,也不羡慕谁。阳光下灿烂,风雨中奔跑。做自己的梦,走自己的路。——三毛
 	
 	\item 世上的事,只要肯用心去学,没有一件是太晚的。 <送你一匹马>
 	
 	\item 人,不经过长夜的痛哭,是不能了解人生的,我们将这些苦痛当作一种功课和学习,直到有一日真正的感觉成长了时,甚而会感谢这种苦痛给我们的教导。—— 三毛
 	
 	\item 生命短促,没有时间可以再浪费,一切随心自由才是应该努力去追求的,别人如何想我便是那么的无足轻重了。——三毛《梦里花落知多少》
 	
 	\item 我迎着朝阳站在大海的面前,对自己说,如果时光不能倒流,就让这一切,随风而去吧——三毛《梦里花落知多少》
 	
 	\item 远方有多远?请你,请你告诉我, 到天涯海角,算不算远? 问一问你的心,只要它答应, 没有地方,是到不了的那么远。 ---《远方》三毛
 	
 	\item 每想你一次,天上飘落一粒沙,从此形成了撒哈拉。
 	
 	每想你一次,天上就掉下一滴水,于是形成了太平洋。——三毛
 	
 	\item 有些人走了就再也没有回来过,所以等待和犹豫是这个世界上最无情的杀手。 —— 三毛
 	
 	\item 一个人至少拥有一个梦想,有一个理由去坚强。心若没有栖息的地方,到哪里都是在流浪。
 	
 	\item 如果你给我的,和你给别人的是一样的,那我就不要了。
 	
 	\item 我来不及认真地年轻,待明白过来时,只能选择认真地老去。
 	
 	\item 或许,我们终究会有那么一天:牵着别人的手,遗忘曾经的他。
 	
 	\item 岁月极美,在于它必然的流逝。 春花、秋月、夏日、冬雪。
 	
 	\item 我笑,便面如春花,定是能感动人的,任他是谁。
 	
 	\item 梦想,可以天花乱坠,理想,是我们一步一个脚印踩出来的坎坷道路。
 	
 	\item 从容不迫的举止,比起咄咄逼人的态度,更能令人心折。
 	
 	\item 走得突然,我们来不及告别。这样也好,因为我们永远不告别。
 	
 	\item 大悲,而后生存,胜于不死不活的跟那些小哀小愁日日讨价还价。——三毛
 	
 	\item 当我们面对一个害怕的人,一桩恐惧的事,一份使人不安的心境时,唯一克服这种感觉的态度,便是面对它。—— 三毛
 	
 	\item 人,真是奇怪,没有外人来证明你,就往往看不出自己的价值。 —— 三毛
 \end{itemize}
 
 \paragraph{8.张爱玲}
 \begin{itemize}
 	\item  多年不见的老朋友,一旦相见,因为都是极熟而又极生疏的人,说话好像深了不是,浅了又不是,彼此都还在暗中摸索,是一种异样的心情,然而也不减少它的愉快。
 	\item  因为爱过,所以慈悲;因为懂得,所以宽容。
 	\item  如果你认识从前的我,那么你就会原谅现在的我。
 	\item  人总是在接近幸福时倍感幸福,在幸福进行时却患得患失。
 	\item  见了他,她变得很低很低,低到尘埃里。但她心里是欢喜的,从尘埃里开出花来。
 	\item  我要你知道,在这个世界上总有一个人是等着你的,不管在什么时候,不管在什么地方,反正你知道,总有这么个人。
 	\item  你问我爱你值不值得,其实你应该知道,爱就是不问值得不值得。
 	\item  喜欢一个人,会卑微到尘埃里,然后开出花来
 	\item  热闹、拥挤,然而陌生,隔阂,人与人之间的沟通充满幻觉、烟幕。 这个世上“好人”很多,“真人”很少。
 	\item  有些人一直没有机会见,等有机会见了,却又犹豫了,相见不如不见。 有些事一直没有机会做,等有机会了,却不想再做了。 有些话埋葬在心中好久,没机会说,等有机会说的时候,却说不出口了。有些爱一直没有机会爱,等有机会爱了,已经不爱了。
 	\item  抄袭是隆重的赞美
 	\item  生在这世上,没有一样感情不是千疮百孔的。
 	\item  说好永远的,不知怎么就散了。最后自己想来想去,竟然也搞不清楚当初是什么原因把彼此分开的。然后,你忽然醒悟,感情原来是这么脆弱的,经得起风雨,却经不起平凡..
 	\item  若只是喜欢,何必夸张成爱。若只是多心,何必虚张成情。若只是微凉,何必虚夸成殇。 若只是多情,何苦句句是恋。若只是心痛,何必说成心碎。若只是神伤,何必虚说成怨。
 	\item  一个人、如果没空,那是因为他不想有空; 一个人、如果走不开,那是因为不想走开;一个人、对你借口太多,那是因为不想在乎。
 	\item  我爱你,关你什么事? 千怪万怪也怪不到你身上去。
 	\item  没伞的挨着有伞的人走,靠的再近也躲不过雨,反淋得更湿。倒不如躲得远远的,就是无伞也有雨过天晴的时候。即使不靠近,也能拥有属于自己的阳光天地。
 	\item  \textbf{相爱着的人又是往往的爱闹意见,反而是莫不相干的人能够互相容忍。}
 	\item  死生契约,与子成说,执子之手,与子偕老,实在是最悲哀的一首诗,死与生与离别,都是大事,不由我们支配的。比起外界力量,我们人是多么小,多么小!可是我们偏要说:“我要永远和你在一起,一生一世也不分开”好像我们能做的了主似的。
 	\item 
 \end{itemize}		
 
 \paragraph{9.卡夫卡}
 \begin{itemize}
 	\item \textbf{人的主罪有二,其余皆由此而来:急躁和懒散}。由于急躁,他们被逐出了天堂;由于懒散,他们再也回不去。——卡夫卡
 	
 	\item 你还年轻。不相信明天的青年就是对自己的背叛。人要生活,就一定要有信仰。信仰什么?相信一切事物和一切时刻的合理的内在联系,相信生活作为整体将永远延续下去,相信最近的东西和最远的东西。——卡夫卡
 	\item  <城堡>:说的是一个人办居住证这样一件很简单的事,卡夫卡用了23万字的篇幅来写这个事,但到最后还是没办成。他把人的这种荒诞处境,可以说推到了极致。在上个世纪80年代,我们单位有一位搞基建的同事,他说那时候盖一栋房子,要盖72个章才能够动工。恰恰如《城堡》一样,看得见却总也走不到,求爷爷告奶奶也不行,现实就是这么个玩意儿
 	\item  正如人们常说,\textbf{做的如何}是一个水平问题,但是,\textbf{是否真诚去做},却是一个态度问题。生活中没有侥幸,生活将以铁一般的逻辑,粉碎任何人发自内心的背叛和梳理倾向
 	\item  \textbf{努力想得到什么东西,其实只要沉着镇静、实事求是,就可以轻易地、神不知鬼不觉地达到目的}。而如果过于使劲,闹得太凶,太幼稚,太没有经验,就哭啊,抓啊,拉啊,像一个小孩扯桌布,结果却是一无所获,只不过把桌上的好东西都扯到地上,永远也得不到了
 	\item  光勤劳是不够的,蚂蚁也非常勤劳。你在勤劳些什么呢?有两种过错是基本的,其他一切过错都由此而生:急躁和懒惰。
 	\item  人的主罪有二,其余皆由此而来:急躁和懒散。由于急躁,他们被逐出了天堂;由于懒散,他们再也回不去。
 	\item  人只因承担责任才是自由的。这是生活的真谛。
 	\item  您不知道,沉默包含了多少力量。咄咄逼人的进攻只是一种假象,一种诡计。人们常常用它在自己和世界面前掩饰弱点。真正持久的力量存在忍受中,只有软骨头才急躁粗暴,他们因此而丧失了人的尊严。
 	\item  你的意志是自由的。这就是说:当它想要穿越沙漠时,它是自由的,因为它可以选择穿越的道路,所以它是自由的,由于它可以选择走路的方式,所以它是自由的。可是它也是不自由的,因为你必须穿越这片沙漠,不自由,因为无论哪条路,由于其谜般的特点,必然令你触及这片沙漠的每一寸土地。<误入世界>
 \end{itemize}	
 
 \paragraph{10.钱钟书}
 \begin{itemize}
 	\item 似乎我们总是很容易忽略当下的生活,忽略许多美好的时光。而当所有的时光在被辜负被浪费后,才能从记忆里将某一段拎出,拍拍上面沉积的灰尘,感叹它是最好的。
 \end{itemize}
 
 \paragraph{11.雨果}
 \begin{itemize}
 	\item 当一切入睡,我常兴奋地独醒,仰望繁星密布熠熠燃烧的穹顶,我静坐着倾听夜深的和谐;时辰的鼓翼没打断我的凝思,我激动地注视这永恒的节日——光辉灿烂的天空把夜赠给世界。
 \end{itemize}
 
 \paragraph{12.席慕容}
 \begin{itemize}
 	\item 在一回首间,才忽然发现,原来,我一生的种种努力,不过只为了周遭的人对我满意而已。为了搏得他人的称许与微笑,我战战兢兢地将自己套入所有的模式所有的桎梏。走到途中才忽然发现,我只剩下一副模糊的面目,和一条不能回头的路。
 	\item 我只是个戏子,在别人的故事里,流着自己的泪。
 	\item 青春是一本太仓促的书,我们含着泪,一读再读。
 	\item 在这人世间,有些路是非要单独一个人去面对,单独一个人去跋涉的,路再长再远,夜再黑再暗,也得独自默默地走下去。
 	\item 青春,如同一场盛大而华丽的戏,我们有着不同的假面,扮演着不同的角色,演绎着不同的经历,却有着相同的悲哀。
 	\item 有情不必终老,暗香浮动恰好。
 	\item 我终于相信,每一条走上来的路,都有它不得不那样跋涉的理由。每一条要走下去的路,都有它不得不那样选择的方向。
 	\item 如果一开始就是一种错误,那么为什么,它会错的那样美丽。
 	\item 所有的悲欢都已化为灰烬,任世间哪一条路,我都不能与你同行。
 	\item 原来岁月并不是真的逝去,它只是从我们的眼前消失,却转过来躲在我们的心里,然后再慢慢地来改变我们的容貌。
 	\item 茉莉好像
 	没有什么季节,在日里在夜里,时时开着小朵的、清香的蓓蕾。\\
 	\textbf{想你,好像也没有什么区别,在日里在夜里,在每一个 恍惚的刹那间。}
 \end{itemize}
 
 \paragraph{13.周国平}
 \begin{itemize}
 	\item 生命不同季节的体验都是值得珍惜的,它们是完整的人生体验的组成部分。一个人在任何年龄段都可以有人生的收获,岁月的流逝诚然令人悲伤,但更可悲的是自欺式的年龄错位。
 	\item 人生是一个从一而终的女人,你不妨尽自己的力量打扮她,引导她,但是,不管她终于成个什么样子,你好歹得爱她。
 	\item 爱情既是在异性世界中的探险,带来发现的惊喜,也是在某一异性身边的定居,带来家园的安宁。但探险不是猎奇,定居也不是占有。毋宁说,好的爱情是双方以自由为最高赠礼的洒脱,以及决不滥用这一份自由的珍惜。——周国平《人与永恒》
 	\item 一个人要获得实在的幸福,就必须既不太聪明,也不太傻。人们把这种介于聪明和傻之间的状态叫做生活的智慧。——周国平《论幸福》
 	\item 读书到底是为了什么,如果我们排除做学问很实际的目的,读书就是我在吸取营养,把自己丰富起来。我自己感觉,读书最愉快的是什么时候,是你突然发现“我也有这个思想”。最快乐的时候是把你本来已经有的,你却不知道的东西唤醒了。 ——周国平
 	\item 每一个人都可能突然遭遇没有明天的一天。可是,世人往往为不可靠的明天复明天付出全部心力,却把一个个今天都当作手段牺牲掉了。——周国平
 	
 	\item 我们终于发现,忍受不可忍受的灾难是人类的命运。接着我们又发现,只要咬牙忍受,世上并无不可忍受的灾难。——周国平《妞妞》
 \end{itemize}
 
 \paragraph{14.林徽因}
 \begin{itemize}
 	\item  答案很长,我准备用一生的时间来回答,你准备要听了吗?
 	\item  你是一树一树的花开,是燕在梁间呢喃。你是爱,是暖,是希望,你是人间的四月天。
 	\item  风轻云淡,岁月安好。
 	\item  我情愿化成一片落叶, 让风吹雨打到处飘零; 或流云一朵,在澄蓝天, 和大地再没有些牵连。
 	\item  你给了我生命中不能承受之重,我将用我的一生来报答你。
 	\item  从没有人说过八月什么话,夏天过去了,也不到秋天。
 	\item  现在连秋云黄叶又已失落去辽远里,剩下灰色的长空一片透彻的寂寞,你忍听冷风独语?
 \end{itemize}
 
 \paragraph{15.萧红}
 \begin{itemize}
 	\item 生前何必久睡,死后自会长眠。
 	\item 满天星光,满屋月亮,人生何如,为什么这么悲凉。若赶上一个下雨的夜,就特别凄凉,寡妇可以落泪,鳏夫就要起来彷徨。
 	\item 逆来顺受,你说我的生命可惜,我自己却不在乎。你看着很危险,我却自以为得意。不得意怎样?人生是苦多乐少。
 	\item 花开了,就像睡醒了似的。鸟飞了,就像在天上逛似的。虫子叫了,就像虫子在说话似的。要做什么,就做什么。
 	\item 人生为了什么,才有这么凄凉的夜。
 	\item \textbf{我不能决定怎么生,怎么死。但我可以决定怎样爱,怎样活。}
 	\item 我仍搅着杯子,也许漂流久了的心情,就和离了岸的海水一般,若非遇到大风是不会翻起的。
 \end{itemize}	
 
 \paragraph{16.沈从文}
 \begin{itemize}
 	\item 有些人是可以用时间轻易抹去的,犹如尘土。
 	\item 这个人也许永远不回来了,也许明天回来。
 	\item 在青山绿水之间,我想牵着你的手,走过这座桥,桥上是绿叶红花,桥下是流水人家,桥的那头是青丝,桥的这头是白发。
 	\item 我就这样一面看水一面想你。
 	\item 凡事都有偶然的凑巧,结果却有如宿命的必然。
 	\item 我用手去触摸你的眼睛。太冷了。倘若你的眼睛这样冷,有个人的心会结成冰。
 	\item 我行过许多地方的桥,看过许多次数的云,喝过许多种类的酒,却只爱过一个正当最好年龄的人。
 \end{itemize}	
 \paragraph{*.昳名}
 \begin{itemize}
 	\item  人是不能闲的,闲久了,努力一下就以为在拼命.
 	\item  为什么有些人明明看起来很友善,却总是独来独往?\textbf{ 待人友善是修养,独来独往是性格}。
 	\item  等待也好,迷茫也罢,都不要把自己留在原地。只要你下定决心,每一天都可以是新的开始。
 	\item  对有些人来说,生活就是不断破墙而出的过程,而对另外一些人,生活是在为自己建起一座座的围墙。
 	\item  People say that you don't know what you've got until it's gone. Truth is, you knew what you had, but you just thought you'd never lose it. 
 	
 	人们说,直到失去了你才会知道拥有过什么。事实上,你一直知道你拥有什么,只是你以为你永远不会失去它。
 	\item  人像一粒种子偶然地飘落到这个世界,没有任何本质可言,只有存在着。要想确立自己的本质,必须通过自己的行动来证明。\textbf{人不是别的东西,而仅仅是他自己行动的结果}。 —— 萨特 
 	
 	\item 朋友或是情人,能走过三个月的已不容易,能坚持六个月的值得珍惜,能相守一年的堪称奇迹,能熬过两年的才叫知己,超过三年的值得记忆,五年后还在的,应该请进生命里。十年后依然在的,那就不是朋友了,已经是亲人,是生命的一部分了!——莫言
 	
 	\item 那时候,未来遥远而没有形状,梦想还不知道该叫什麽名字。
 	
 	我常常一个人,走很长的路,在起风的时候觉得自己像一片落叶。
 	
 	仰望星空,我很想知道:
 	
 	真的有人正从世界的某个地方朝我走来吗?
 	——几米
 	
 	\item 人生最痛苦的事,莫过于不断努力而梦想永远无法实现,而我们的人生正是如此。令人欣慰的是,我听见时间长廊另一端有个声音说,\textbf{“也许今天无法实现,明天也不能。重要的是,它在你心里。重要的是,你一直在努力。”}———《马丁·路德·金自传》
 	
 	\item 从来都是别离时,才知爱有多深。——纪伯伦
 	
 	\item 有些东西,并不是越浓越好,要恰到好处。深深的话我们浅浅地说,长长的路我们慢慢地走。——毕淑敏《恰到好处的幸福》
 	
 	\item 有一些人,这辈子都不会在一起,但是有一种感觉却可以藏在心里,守一辈子。——张小娴
 	
 	\item 真正有气质的淑女,从不炫耀她所拥有的一切,她不告诉人她读过什么书,去过什么地方,有多少件衣服,买过什么珠宝,因为她没有自卑感。——亦舒《圆舞》
 	
 	\item 不管全世界所有人怎么说,
 	
 	我都认为自己的感受才是正确的。
 	
 	无论别人怎么看,我绝不打乱自己的节奏。
 	
 	\textbf{喜欢的事自然可以坚持,不喜欢的怎么也长久不了。}
 	
 	——村上春树《当我谈跑步时我谈些什么》
 	
 	\item 一青一黄是一年,一黑一白是一天。桃花岁岁花相似,岁岁年年人不同,河水泉源千年在,青春过后,温暖过往回忆,岁月却不会回头。
 	
 	\item 情不知所起,一往而深——汤显祖《牡丹亭》
 	
 	\item 繁华尽处,寻一无人山谷,建一木制小屋,铺一青石小路,与你晨钟暮鼓,安之若素 。
 	
 	\item 择一城终老,遇一人白首;挽一帘幽梦,许一世倾城;写一字决别,言一梦长眠。我倾尽一生,囚你无期。 择一人深爱,等一人终老;痴一人情深,留一世繁华。;断一根琴弦,歌一曲离别。我背弃一切,共度朝夕。—— 冯骥才 《 择一城 终老,遇一人 白首》
 	
 	\item “也许天长地久可以做如是解:\textbf{你一生中只有那么一刻,你全心投入去爱过一个人,那一刻就是永恒。你一生中如果有那么一段路,有一个人与你互相扶持,共御风雨,那么,那一段也就胜过终生了}。”——蒋方舟《我承认我不曾历经沧桑》
 	
 	\item 你拥有青春的时候,就要感受它。不要虚掷你的黄金时代,不要去倾听枯燥乏味的东西,不要设法挽留无望的失败,不要把你的生命献给无知、平庸和低俗。这些都是我们时代病态的目标,虚假的理想。活着!把你宝贵的内在生命活出来。什么都别错过。——王尔德
 	
 	\item \textbf{每个人都是月亮,总有一个阴暗面,从来不让人看见}。--马克吐温
 	
 	\item 懂与不懂,不重要,重要的是路在脚下。走下去,走到明白的那一天,这过程的一步步,当有一天回头时,可以看到一种求真的美好。——耳根《求魔》
 	
 	\item 你现在的生活也许不是你想要的,但绝对是你自找的。 世界上100\%的抱怨都可以用这句话来回答。
 	
 	\item 大部分的恐惧与懒惰有关,这句话我深以为然。我们常常会害怕改变,其实都是因为自己太懒了,懒得去适应新的环境,懒得去学习新的知识,涉足新的领域,但如果总是这样的话如何能让自己成熟起来呢?| M•斯科特•派克《少有人走的路》
 	
 	\item  优雅的回复别人说你黑: 我不想做肤浅的人,我不想白活一辈子,我妈肚子里有墨水,老子之前白的,后来帅炸了,炸黑了; 我为了暗中保护你,我晒你家太阳了? 黑夜给了我一双黑色的眼睛,可是我不小心点了全选, 眼前的黑不是黑,你说的白是什么白
 	
 	\item 有那么一瞬间,因为一个人的一句话,就像被泼了一盆凉水一样,唰的一下,从头冷到脚,语言这东西,在表达爱意的时候是那么无力,在表达伤害的时候却又如此锋利。
 	
 	\item 如果只看合乎自己口味的书,那你永远只能知道你已经知道的事情。 ——蔡康永
 	
 	\item 在等待的过程中还是想着如何让自己变优秀吧,你优秀了自然有对的人与你并肩。最了不起的不是拥有最好一切的人,而是把一切都变好的人。
 	
 	\item 世间不存在失败,只不过是换个方向走走。
 	
 	\item 有时候,不小心知道了一些事,才发现自己所在乎的事是那么可笑。不要以为你放不下的人同样会放不下你,鱼没有水会死,水没有鱼却会更清澈,谁不虚伪,谁不善变,到最后谁都不是谁的谁,又何必高估了自己,把一些人,一些事看得那那么重要。
 	
 	\item 心情再差,也不要写在脸上,因为没有人喜欢看; 日子再穷,也不要挂在嘴边,因为没有人无故给你钱; 工作再累,也不要抱怨,因为没有人无条件替你干;生活再苦,也不要失去信念,因为美好将在明天; 品性再坏,也要孝顺父母,因为你也有老的那天。
 	
 	\item 年轻的时候以为不读书不足以了解人生,直到后来才发现如果不了解人生,是读不懂书的。读书的意义大概就是用生活所感去读书,用读书所得去生活吧。——杨绛
 	
 	\item \textbf{人们不应该期望从别人或外部世界获得太多,一个人对另一个人而言并没有那么重要——说到底,人只能靠自己}。——叔本华《人生的智慧》
 	
 	\item 张牙舞爪的人,往往是很脆弱的。因为真正强大的人,是自信的,自信就会温和,温和就会坚定。——《明朝那些事儿》
 	
 	\item 书读的越多而不加思考,你就会觉得你知道得很多;而当你读书而思考得越多的时候,你就会越清楚地看到,你知道得很少。—— 伏尔泰
 	
 	\item 你必须先宽恕才能理解。在你能宽恕之前,你挡住了自己获得理解的可能性。——玛里琳•鲁宾逊
 	
 	\item 能体谅别人,并且接纳别人的痛苦,那才是真正的温柔。比和颜悦色更可贵的,是你温柔的心。——柏邦妮
 	
 	\item 多读几本好书、多交几个好朋友、甚至养成几个好习惯。二十岁是投资和储蓄的时候,要养足未来的资本。魅力的三十,不会从天而降,如果之前的29年不够努力,那么过了30岁也不会有奇迹出现。——杨澜
 	
 	\item 性格的作用比智力大得多,头脑的作用不如心情,天资不如由判断力所节制着的自制、耐心和规律。开始在内心生活得更严肃的人,也会在外表上开始生活得更朴素。悔恨自己的错误,而且力求不再重蹈覆辙,这才是真正的悔悟。优于别人,并不高贵,真正的高贵应该是优于过去的自己。——海明威
 	
 	\item 一个人彻悟的程度,恰等于他所受痛苦的深度。——林语堂
 	
 	\item 永远不要把人拿来比较,每个人都与众不同,重要的是要找到最适合自己的差异性。——马克·李维《偷影子的人》
 	
 	\item 我几乎从来不生气,因为我认为没必要,有问题就去解决,不要让别人的错误影响自己。这是我大多时候感到快乐的秘诀。但是,我不生气,不代表我没脾气。我不计较,不代表我脾气好。如果你非要触摸我的底线,我可以告诉你,我并非善良。---- 陈丹青
 	
 	\item “ 有一个夜晚我烧毁了所有的记忆,从此我的梦就透明了;有一个早晨我扔掉了所有的昨天,从此我的脚步就轻盈了 。” -- 泰戈尔
 	
 	\item 佛曰,人生有八苦:生,老,病,死,爱别离,怨长久,求不得,放不下。
 	
 	\item 时间,是爱情唯一的货币。没有性价比,没有打折季,谁也不比谁便宜。在一起,就是我们的毕生积蓄。——张嘉佳
 	
 	\item 茶不过两种姿态,浮、沉;饮茶人不过两种姿势,拿起、放下。人生如茶,沉时坦然,浮时淡然,拿得起也需要放得下。
 	
 	\item 我对自己的要求很低:我活在世上,无非想要明白些道理,遇见些有趣的事。倘能如我愿,我的一生就算成功。——王小波《沉默的大多数》
 	
 	\item 大雨可以延迟我们到达的时间,但不能阻止我们前进。——卢梭
 	
 	\item 这世上有三样东西是别人抢不走的:一是吃进胃里的食物,二是藏在心中的梦想,三是读进大脑的书。
 	
 	\item 我们控制不好情绪,一是我们修行不够,二是我们接触了让我们产生负面情绪的人与事。当我们的情绪受到影响时,适当的远离负面的环境,让自己平息下来。冥想一些快乐的人与事或者一次远行,心胸开阔起来,再回头看之前的是是非非,一切也就释然了。很多矛盾的起因都是一些小事,不懂得退就会越搅越大。
 	
 	\item 你必须只有内心丰富,才能摆脱这些生活表面的相似。—— 王朔《致女儿书》
 	
 	\item 当你见识到男人最好一面时,请不要爱上,因为那不是全部。只有当你见到过他愤怒、沮丧、暴躁和孤独,才可以确定感情。许多人,总以一个侧面来爱上人,所以她未来会失望,觉得对方变了。但那反而是事实的全部。见到一个人最差的样子,依旧能爱上,才是真爱。---苏岑
 	
 	\item “ 盛喜中,勿许人物。盛怒中,勿答人书。喜时之言,多失信。怒时之言,多失体。 ” —— 弘一
 	
 	\item 我们所要做的事,应该一想到就做;因为人的想法是会变化的,有多少舌头,多少手,多少意外,就会有多少犹豫,多少迟延。——莎士比亚
 	
 	\item 那时漫天飞雪,想拍给你看;那时听到好歌,想唱给你听;那时激动的情绪,希望不用说都有人懂。喜欢就是看到所有美好的东西,都想和你分享。后来走廊被黄昏染色,冬天被大雪唤醒,思念被歌曲收藏,却找不到分享的人。告别就是看到所有美好的东西,也不会再和你说了。——卢思浩
 	
 	\item 圈子决定人生,接近什么样的人,就会走什么样的路,所谓物以类聚,人以群分。牌友只会催你打牌,酒友只会催你干杯,而靠谱的人却会感染你如何取得进步。
 	
 	\item 人们日常所犯最大的错误,是对陌生人太客气,而对亲密的人太苛刻,把这个坏习惯改过来,天下太平。--亦舒
 	
 	\item 置身事外,谁都可以心平气和,身处其中,谁还可以淡定从容?所以请不要轻易评论任何人,因为你不在其中。
 	
 	\item 一个人,有时真是蛮享受的。不用因迁就而遗失自我,亦不必为维持自我而得罪任何人,一个人有一个的好。—— 亦舒
 	
 	\item \textbf{无聊是非常有必要的,一个人在空白时间所做的事,决定了这个人和其他人根本的不同。} —— 宁远《远远的村庄》
 	
 	\item 也许生存世间的人们都只是在等待一种偶遇,一种适时的相遇,时间对了,你们便会遇上。 ——宫崎骏《龙猫》
 	
 	\item 当时的他是最好的他,后来的我是最好的我。 可是最好的我们之间,隔了一整个青春。 怎么奔跑也跨不过的青春,只好伸出手道别。——八月长安《最好的我们》
 	
 	\item 在哪里存在,就在哪里绽放。不要因为难过,就忘了散发芳香。—— 渡边和子《就在你所在的地方生根开花》
 	
 	\item 一个人最大的缺点,不是自私、多情、野蛮、任性,而是偏执地爱一个不爱自己的人。	——张小娴 《不如你送我一场春雨》
 	
 	\item 有时候,在恋情都已经结束了很久以后,我们才渐渐醒悟原来那时候我们爱那个人的方式,错了。然后,我们就忽然很想把那个人找回来,重新再爱一遍。——蔡康永《给未知恋人的爱情短信》
 	
 	\item 我不去想是否能够成功,既然选择了远方,便只顾风雨兼程。我不去想能否赢得爱情,既然钟情于玫瑰,就勇敢地吐露真诚。我不去想身后会不会袭来寒风冷雨,既然目标是地平线,留给世界的只能是背影。——汪国真 
 	
 	\item 你要永远记住,对于一对恩爱的夫妻,最重要的不是幸福,而是稳定。 ——马尔克斯《霍乱时期的爱情》
 	
 	\item 一个人,最好的时光是二十岁到三十岁。与其虚度年华,不如去追求真正要的梦想。幸好,我从来没有成为过“橡皮人”。你有没有回想过,少年时的梦想是什么? °
 	
 	\item 负面情绪过多的人,常常也会沾惹令人讨厌的事件上门,使得自己的情绪更加地负面。建议大家常常锻炼身体,别让有毒的情绪在身体里停留太久。同时多去郊外、大自然踏青,这样心胸变得比较开阔,就不会执着于一件事情或一个人,而沉浸在负面的故事情节中无法自拔。——张德芬
 	
 	\item 只要记住你的名字,不管你在世界的哪个地方,我一定会,去见你。—— 《你的名字》
 	
 	\item 成功的花, 人们只惊羡她现时的明艳! 然而当初她的芽儿, 浸透了奋斗的泪泉, 洒遍了牺牲的血雨。 ——冰心《繁星·春水》
 	
 	\item 识不足,则心多虑,多虑则伪学,以不知为知,表面学富五车,实乃败絮其中;威不足,则情多怒,多怒则易暴,目侧使心曲,气戾令人远;信不足,则口多言,然言多必失,夸夸其谈者虚,口吐莲花者浅。为人之要则,当抬头时观云,低首时看路,于学中思,于苦里行,善养浩然之气、谦卑之性,方能大写人生。
 	
 	\item 趁阳光正好,趁微风不噪。趁繁花还未开至荼蘼,趁现在还年轻。还可以走很长很长的路,还能诉说很深很深的思念。去寻找那些曾出现在梦境中的路径、山峦与田野吧。 ——毕淑敏
 	
 	\item 人生聚散无常,起落不定,但是走过去了,一切便已从容。无论是悲伤还是喜乐,翻阅过的光阴都不可能重来。曾经执著的事如今或许早已不值一提,曾经深爱的人或许已经成了陌路。这些看似浅显的道理,非要亲历过才能深悟。—— 林徽因
 	
 	\item 你不管做什么事,如果做得太好了,一不警惕,就会在无意中卖弄起来。那样的话,你就不再那么好了。	——塞林格《麦田守望者》
 	
 	\item 寂静在喧嚣里低头不语,沉默在黑夜里与目光结交,于是,我们看错了世界,却说世界欺骗了我们。 —— 泰戈尔
 	
 	\item 虽然至今为止的道路绝非一片坦途,但想到正因为活着才有机会感受到痛楚,我就成功克服了种种困难。——东野圭吾《解忧杂货店》
 	
 	\item \textbf{最深沉的感情往往是以最冷漠的方式表现出来的,最轻浮的感情往往是 以最强烈的方式表现出来的}。 —— 汪国真《年轻的季节》
 	
 	\item 看一个家族兴败,看三个地方:第一,\textbf{子孙睡到几点},假如睡到太阳都已经升得很高的时候才起来,那代表这个家族会慢慢懈怠下来;第二 ,\textbf{看子孙有没有做家务},因为勤劳、劳动的习惯影响一个人一辈子;第三,\textbf{看后代子孙有没有在读圣贤的经典},“人不学,不知义,不知 道”。—— 曾国藩
 	
 	\item 在生活的艺术中,\textbf{节省是一种实用哲学,因为极简朴的生活可以提高生活品质}。	——多米尼克·洛罗《简单的艺术》
 	
 	\item 现在不是去想缺少什么的时候,该想一想凭现有的东西你能做什么。	——海明威《老人与海》
 	
 	\item 一个人可以被毁灭,但不能被打败。——海明威
 	
 	\item 我不想谋生,我想生活。—— 王尔德
 	
 	\item 你学过的每一样东西,你遭受的每一次苦难,都会在你一生中的某个时候派上用场。—— 菲茨杰拉德
 	
 	\item 一个人的性格决定他的际遇。如果你喜欢保持你的性格,那么,你就无权拒绝你的际遇。	——罗曼·罗兰
 	
 	\item 小铁开心的时候会笑,摇着尾巴冲我们微笑。小铁心里既没有背叛也没有憎恨,只有爱。它永远保持着一颗纯洁的心,这颗心也感染着我。正处在青春期的女儿不愿意和我说话,我感到很寂寞。小铁仿佛看透了我的心思,轻轻地靠过来,依偎在我身上。	——(日)山口花《嘿,和我聊会儿天吧》
 	
 	\item 人是需要理想、需要幻想的,需要美,以美的意境、美的情操来陶冶自己。	——霍达《穆斯林的葬礼》
 	
 	\item 当你拼命想完成一件事的时候,你就不再是别人的对手,或者说得更确切一些,别人就不再是你的对手了,不管是谁,只要下了这个决心,他就会立刻觉得增添了无穷的力量,而他的视野也随之开阔了。—— 大仲马
 	
 	\item \textbf{今天做不成的,明天也不会做好。一天也不能虚度,要下决心把可能的事情,一把抓住而紧紧抱住,有决心就不会任其逃去,而且必然要贯彻实行。}	——歌德《浮士德》
 \end{itemize}
\end{document}
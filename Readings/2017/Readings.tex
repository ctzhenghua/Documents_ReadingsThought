\documentclass[UTF8,a4paper,8pt]{ctexart} 

\usepackage{graphicx}%学习插入图
\usepackage{verbatim}%学习注释多行
\usepackage{booktabs}%表格
\usepackage{geometry}%图片
\usepackage{amsmath} 
\usepackage{amssymb}
\usepackage{listings}%代码
\usepackage{xcolor}  %颜色
\usepackage{enumitem}%列表格式
\usepackage{hyperref}
\CTEXsetup[format+={\flushleft}]{section}

\geometry{left=1.6cm,right=1.8cm,top=2cm,bottom=1.7cm} %设置文章宽度

\pagestyle{plain} 		  %设置页面布局
\author{郑华}
\title{2017读书笔记}
%代码效果定义
\definecolor{codegreen}{rgb}{0,0.6,0}
\definecolor{codegray}{rgb}{0.5,0.5,0.5}
\definecolor{codepurple}{rgb}{0.58,0,0.82}
\definecolor{backcolour}{rgb}{0.95,0.95,0.92}

\lstdefinestyle{mystyle}{
	backgroundcolor=\color{backcolour},   
	commentstyle=\color{codegreen},
	keywordstyle=\color{magenta},
	numberstyle=\tiny\color{codegray},
	stringstyle=\color{codepurple},
	basicstyle=\footnotesize,
	breakatwhitespace=false,         
	breaklines=true,                 
	captionpos=b,                    
	keepspaces=true,                 
	%numbers=left,                    
	%numbersep=5pt,                  
	showspaces=false,                
	showstringspaces=false,
	showtabs=false,                  
	tabsize=2
}
\lstset{style=mystyle, escapeinside=``}

\begin{document}          %正文排版开始
	\maketitle
	\tableofcontents
	
	\newpage
	\section*{读书计划} 
		\subparagraph{诺贝尔系列}
		\begin{itemize}
			\item 2015年诺贝尔文学奖:白俄罗斯作家   斯维特兰娜·阿列克谢耶维奇 主要代表作品有《我是女兵,也是女人》,《我还是想你,妈妈》,《锌皮娃娃兵》,《我不知道该说什么,关于死亡还是爱情》,世界文坛最高水准的感人作品,讲述了这个时代的苦难和勇气。
			
			颁奖词:她的复调作品是对我们时代的磨难与勇气的纪念
			\item 2014年诺贝尔文学奖:法国作家   帕特里克-莫迪亚 获奖作品《暗店街》、《八月的星期天》
			
			获奖理由:唤醒了对最难以捕捉的人类命运的记忆和揭露了对人类生活的占领。
			\item 2013年诺贝尔获得者:爱丽丝•门罗。她的短篇小说集有《我青年时期的朋友》、《你以为你是谁?》(1978,亦得总督奖)、《爱的进程》(1986,第三次得总督奖)、《公开的秘密》、《一个善良女子的爱》、《憎恨、友谊、求爱、爱恋、婚姻》、《逃离》等,2006年出版《石城远望》是她最新的一部作品集。
			
			获奖理由:当代短篇小说大师
			\item 2012年获得者:中国作家莫言,其代表作有《蛙》《生死疲劳》《丰乳肥臀》《檀香刑》等
			
			获奖理由:“用魔幻现实主义的写作手法,将民间故事、历史事件与当代背景融为一体”。
			\item 010年获奖者:马里奥•巴尔加斯•略萨(1936—),秘鲁与西班牙双重国籍的作家及诗人。代表作:《绿房子》、《城市与狗》、《中国套盒》
			
			获奖理由:他“对权力结构的制图般的描绘和对个人反抗的精致描写”。
			\item 009年《呼吸钟摆》作者:赫塔•缪勒(Herta Müller,1953-),罗马尼亚裔的德国女性小说家、诗人、散文家。
			
			获奖理由:“专注于诗歌以及散文的率真,描写了失业人群的生活图景”。
			\item 2008年获奖作品《战争》作者勒•克莱齐奥
			\item 2007年获奖作品《金色笔记》多丽丝•莱辛(Doris Lessing,1919―),英国作家。
			
			获奖理由:“她用怀疑、热情、构想的力量来审视一个分裂的文明,其作品如同一部女性经验的史诗。
			\item 2006年获奖作品《我的名字叫红》奥尔汉•帕穆克(1952~),土耳其作家。
			
			获奖理由:他的作品“在寻找故乡的忧郁灵魂时,发现了文化碰撞和融合中的新象征”。
			\item 2000年作品《灵山》获诺贝尔文学奖。高行健(1940~)法籍华人。剧作家、小说家。
			
			获奖理由:“其作品的普遍价值,刻骨铭心的洞察力和语言的丰富机智,为中文小说和艺术戏剧开辟了新的道路。”
			\item 1999年作品《铁皮鼓哈里》获诺贝尔文学奖。君特•格拉斯(1927~)德国作家。
			
			获奖理由:“其嬉戏之中蕴含悲剧色彩的寓言描摹出了人类淡忘的历史面目。”
			
			\item 1994年《个人的体验》和《万延元年的足球队》获诺贝尔文学奖。大江健三郎(1935~)日本小说家。
			
			获奖理由:“通过诗意的想象力,创造出一个把现实与神话紧密凝缩在一起的想象世界,描绘现代的芸芸众生相,给人们带来了冲击。”
			\item 1991年作品《七月的人民》获诺贝尔文学奖。内丁•戈迪默(女)(1923~)南非作家。
			
			获奖理由:“以强烈而直接的笔触,描写周围复杂的人际与社会关系,其史诗般壮丽的作品,对人类大有裨益”
			\item 1990年作品《太阳石》获诺贝尔文学奖。奥克塔维奥•帕斯(1914~)墨西哥诗人。
			
			获奖理由:“他的作品充满激情,视野开阔,渗透着感悟的智慧并体现了完美的人道主义”
			\item 1988年作品《街魂》获诺贝尔文学奖。纳吉布•马哈富兹(1911~)埃及作家。
			
			获奖理由:“他通过大量刻画入微的作品—洞察一切的现实主义,唤起人们树立雄心—形成了全人类所欣赏的阿拉伯语言艺术”
			\item 1984年作品《紫罗兰》获诺贝尔文学奖。雅罗斯拉夫•塞弗尔特(1901~1986)捷克诗人。
			
			获奖理由:“他的诗富于独创性、新颖、栩栩如生,表现了人的不屈不挠精神和多才多艺的渴求解放的形象”
			\item 1983年作品《蝇王•金字塔》获诺贝尔文学奖。威廉•戈尔丁(1911~1994)英国作家。
			
			获奖理由:“具有清晰的现实主义叙述技巧以及虚构故事的多样性与普遍性,阐述了今日世界人类的状况”
			\item 1982年作品《霍乱时期的爱情》获诺贝尔文学奖。加夫列尔•加西亚•马尔克斯(1928~)哥伦比亚记者、作家。
			
			获奖理由:“由于其长篇小说以结构丰富的想象世界,其中糅混着魔幻于现实,反映出一整个大陆的生命矛盾”
			\item 1981年作品《迷茫》获诺贝尔文学奖。埃利亚斯•卡内蒂(1905~)英国德语作家。
			
			获奖理由:“作品具有宽广的视野、丰富的思想和艺术力量。”
			\item 1980年作品《拆散的笔记簿》获诺贝尔文学奖。切斯拉夫•米沃什,波兰诗人。
			
			获奖理由:“不妥协的敏锐洞察力,描述了人在激烈冲突的世界中的暴露状态”
			\item 1978年作品《魔术师•原野王》获诺贝尔文学奖。艾萨克•巴什维斯•辛格(1904~1991)美国作家。
			
			获奖理由:“他的充满激情的叙事艺术,这种既扎根于波兰人的文化传统,又反映了人类的普遍处境”
			\item 1976年作品《赫索格》获诺贝尔文学奖。获奖者索尔•贝娄(1915~)美国作家。
			
			获奖理由:“由于他的作品对人性的了解,以及对当代文化的敏锐透视”
			\item 1974年作品《露珠里的世界》获诺贝尔文学奖。哈里•埃德蒙•马丁逊(1904~1978)瑞典诗人。
			
			获奖理由:“他的作品透过一滴露珠反映出整个世界”
			\item 1974年作品《乌洛夫的故事》获诺贝尔文学奖。埃温特•约翰逊(1900~1976)瑞典作家。
			
			获奖理由:“以自由为目的,而致力于历史的、现代的广阔观点之叙述艺术”
			\item 1973年作品《风暴眼》获诺贝尔文学奖。帕特里克•怀特(1912~1990)澳大利亚小说家、剧作家。
			
			获奖理由:“由于他史诗与心理叙述艺术,并将一个崭新的大陆带进文学中”
			\item 1972年作品《女士及众生相》获诺贝尔文学奖。亨利希•伯尔(1917~1985)德国作家。
			
			获奖理由:“为了表扬他的作品,这些作品兼具有对时代广阔的透视和塑造人物的细腻技巧,并有助于德国文学的振兴。”
			\item 1970年作品《癌病房》获诺贝尔文学奖。亚历山大•索尔仁尼琴(1918~)苏联作家。
			
			获奖理由:“由于他作品中的道德力量,籍著它,他继承了俄国文学不可或缺的传统”
			\item 1968年作品《雪国•千只鹤•古都》获诺贝尔文学奖。川端康成(1899~1972)日本小说家。
			
			获奖理由:“由于他高超的叙事性作品以非凡的敏锐表现了日本人精神特质”
			\item 1967年作品《玉米人》获诺贝尔文学奖。安赫尔•阿斯图里亚斯(1899~1974)危地马拉诗人、小说家。
			
			获奖理由:“因为他的作品落实于自己的民族色彩和印第安传统,而显得鲜明生动”
			\item 1966年作品《逃亡》获诺贝尔文学奖。奈莉•萨克斯(女)(1891~1970)瑞典诗人。
			
			获奖理由:“因为她杰出的抒情与戏剧作品,以感人的力量阐述俩以色列的命运”
			\item 1966年作品《行为之书》获诺贝尔文学奖。萨缪尔•约瑟夫•阿格农(1888~1970)以色列作家。
			
			获奖理由:“他的叙述技巧深刻而独特,并从犹太民族的生命汲取主题” 
			\item 1965年作品《静静的顿河》获诺贝尔文学奖。史米哈伊尔•亚历山大罗维奇•肖洛霍夫(1905~1984)苏联作家。
			
			获奖理由:“由于这位作家在那部关于顿河流域农村之诗作品中所流露的活力与艺术热忱——他籍这两者在那部小说里描绘了俄罗斯民族生活之某一历史层面”
			\item 1964年作品《苍蝇》获诺贝尔文学奖。让•保尔•萨特(1905~1980)法国哲学家、作家。
			
			获奖理由:“因为他那思想丰富、充满自由气息和探求真理精神的作品对我们时代发生了深远影响”
			\item 1962年作品《人鼠之间》约翰•斯坦贝克(1902~1968)美国作家。
			
			获奖理由“通过现实主义的、寓于想象的创作,表现出富于同情的幽默和对社会的敏感观察”
			\item 1961年作品《桥•小姐》获诺贝尔文学奖。伊沃•安德里奇(1892~1975)南斯拉夫小说家。
			
			获奖理由:“由于他作品中史诗般的力量——他籍著它在祖国的历史中追寻主题,并描绘人的命运”
			\item 1957年作品《局外人•鼠疫》获诺贝尔文学奖。阿尔贝•加缪(1913~1960)法国作家。
			
			获奖理由:“由于他重要的著作,在这著作中他以明察而热切的眼光照亮了我们这时代人类良心的种种问题”
			\item 1955年作品《渔家女》获诺贝尔文学奖。赫尔多尔•奇里扬•拉克斯内斯斯(1902~)冰岛作家。
			
			获奖理由:“为了他在作品中所流露的生动、史诗般的力量,使冰岛原已十分优秀的叙述文学技巧更加瑰丽多姿”
			\item 1954年作品《老人与海》获诺贝尔文学奖。欧内斯特•海明威(1899~1961)美国作家。
			
			获奖理由:“因为他精通于叙事艺术,突出地表现在其近著《老人与海》之中;同时也因为他对当代文体风格之影响”
			\item 1953年作品《不需要的战争》获诺贝尔文学奖。温斯顿•丘吉尔(1874~1965)英国政治家、历史学家、传记作家。
			
			获奖理由:“由于他在描述历史与传记方面的造诣,同时由于他那捍卫崇高的人的价值的光辉演说。”
			\item 1947年作品《田园交响曲》获诺贝尔文学奖。德烈•纪德(1869~1951)法国作家、评论家。
			
			获奖理由:“为了他广泛的与有艺术质地的著作,在这些著作中,他以无所畏惧的对真理的热爱,并以敏锐的心理学洞察力,呈现了人性的种种问题与处境”
			\item 1946年作品《荒原狼》获诺贝尔文学奖。曼•黑塞(1877~1962)德国作家。
			
			获奖理由:“他那些灵思盎然的作品——它们一方面具有高度的创意和深刻的洞见,一方面象征古典的人道理想与高尚的风格”
			\item 1944年作品《漫长的旅行》获诺贝尔文学奖。约翰内斯•威廉•扬森(1873~1950)丹麦小说家、诗人。
			
			获奖理由:“由于籍著丰富有力的诗意想象,将胸襟广博的求知心和大胆的、清新的创造性风格结合起来
			\item 1939年作品《少女西丽亚》获诺贝尔文学奖。弗兰斯•埃米尔•西兰帕(1888~1964)芬兰作家。
			
			获奖理由:“由于他在描绘两样互相影响的东西——他祖国的本质,以及该国农民的生活时——所表现的深刻了解与细腻艺术”
			\item 1938年作品《大地》获诺贝尔文学奖。赛珍珠(珀尔•塞登斯特里克•布克)(女)(1892~1973)美国作家。
			
			获奖理由:“她对于中国农民生活的丰富和真正史诗气概的描述,以及她自传性的杰作”
			\item 1937 年作品《蒂伯—家》获诺贝尔文学奖。罗杰•马丁•杜•加尔(1881~1958)法国小说家。
			
			获奖理由:“由于在他的长篇小说《蒂伯一家》中表现出来的艺术魅力和真实性。这是对人类生活面貌的基本反映。”
			\item 1932年作品《有产者》获诺贝尔文学奖。约翰•高尔斯华绥(1867~1933)英国小说家、剧作家。
			
			获奖理由:“为其描述的卓越艺术——这种艺术在《福尔赛世家》中达到高峰”
			\item 1931年作品《荒原和爱情》获诺贝尔文学奖。埃利克•阿克塞尔•卡尔费尔德(1864~1931)瑞典诗人。
			
			获奖理由:“由于他在诗作的艺术价值上,从没有人怀疑过”
			\item 1930年作品《巴比特》获诺贝尔文学奖。辛克莱•刘易斯(1885~1951)美国作家。
			
			获奖理由:“由于他充沛有力、切身和动人的叙述艺术,和他以机智幽默去开创新风格的才华”
			\item 1929年作品《魔山》获诺贝尔文学奖。保尔•托马斯•曼(1875~1955)德国作家。
			
			获奖理由:“由于他那在当代文学中具有日益巩固的经典地位的伟大小说《布登勃洛克一家》。”
			\item 1928 年作品《新娘—主人—十字架》获诺贝尔文学奖。西格里德•温塞特(女)(1882~1949)挪威作家。
			
			获奖理由:“主要是由于她对中世纪北国生活之有力描绘”
			\item 1926年作品《邪恶之路》获诺贝尔文学奖。格拉齐亚•黛莱达(女)(1871~1936)意大利作家。
			
			获奖理由:“为了表扬她由理想主义所激发的作品,以浑柔的透彻描绘了她所生长的岛屿上的生活;在洞察人类一般问题上,表现的深度与怜悯”
			\item 1925年作品《圣女贞德》获诺贝尔文学奖。乔治•萧伯纳(1856~1950)爱尔兰戏剧家。
		
			获奖理由:“由于他那些充满理想主义及人情味的作品——它们那种激动性讽刺,常涵蕴着一种高度的诗意美”
			\item 1924年作品《福地》获诺贝尔文学奖。弗拉迪斯拉夫•莱蒙特(1868~1925)波兰作家。
		
			获奖理由:“我们颁奖给他,是因为他的民族史诗《农夫们》写得很出色”
			\item 1917年作品《天国》获诺贝尔文学奖。亨利克•彭托皮丹,丹麦小说家。
		
			获奖理由:“由于他对当前丹麦生活的忠实描绘”
			\item 1917年作品《磨坊血案》获诺贝尔文学奖。卡尔•耶勒鲁普,丹麦作家。
		
			获奖理由:“因为他多样而丰富的诗作——它们蕴含了高超的理想”
			\item 1915年作品《约翰—克利斯朵夫》获诺贝尔文学奖。罗曼•罗兰(1866~1944)法国作家、音乐评论家。
		
			获奖理由:“文学作品中的高尚理想和他在描绘各种不同类型人物时所具有的同情和对真理的热爱”
			\item 1913年作品《吉檀枷利—饥饿石头》获诺贝尔文学奖。罗宾德拉纳特•泰戈尔(1861~1941)印度诗人、社会活动家主。
		
			获奖理由:“由于他那至为敏锐、清新与优美的诗;这诗出之于高超的技巧,并由于他自己用英文表达出来,使他那充满诗意的思想业已成为西方文学的一部分”
			\item 1911年作品《花的智慧》获诺贝尔文学奖。莫里斯•梅特林克(1862~1949)比利时剧作家、诗人、散文家。
		
			获奖理由:“由于他在文学上多方面的表现,尤其是戏剧作品,不但想象丰富,充满诗意的奇想,有时虽以神话的面貌出现,还是处处充满了深刻的启示。这种启示奇妙地打动了读者的心弦,并且激发了他们的想象”
			\item 1909年作品《骑鹅旅行记》获诺贝尔文学奖。西尔玛•拉格洛夫(女)(1858~1940)瑞典作家。
		
			获奖理由:“由于她作品中特有的高贵的理想主义、丰富的想象力、平易而优美的风格”
			\item 1907年作品《老虎!老虎!》获诺贝尔文学奖。约瑟夫•鲁德亚德•吉卜林(1865~1936)英国小说家、诗人。
		
			获奖理由:“这位世界名作家的作品以观察入微、想象独特、气概雄浑、叙述卓越见长”
			\item 1905年作品《第三个女人》获诺贝尔文学奖。亨利克•显克维支(1846~1916)波兰小说家。
			
			获奖理由:“由于他在历史小说写作上的卓越成就”
			\item 1902年作品《罗马风云》获诺贝尔文学奖。特奥多尔•蒙森(1817~1903)德国历史学家。
			
			获奖理由:“今世最伟大的纂史巨匠,此点于其巨著《罗马史》中表露无疑”
			\item 1901年作品《孤独与深思》获诺贝尔文学奖。苏利•普吕多姆(1839~1907)法国诗人。
			
			获奖理由:“是高尚的理想、完美的艺术和罕有的心灵与智慧的实证” 
		\end{itemize}
	
		\subparagraph{名声系列}
		\begin{itemize}
			\item 麦田里的守望者
			\item 走到人生边上
			\item 了不起的盖茨比
			\item 从0到1
			\item 瓦尔登湖
			\item 解忧杂货铺
			\item 红楼梦
			\item 百年孤独
			\item 重读- 岛上书店
		\end{itemize}
		
	\section{《 了不起的盖茨比  》  }  
		一个钟情于黛西的人,一个绯闻和身世充满谜题的人,和对黛西一直不灭的梦,还有黛西最后的背叛。
		
		故事虽如此..一句句描写的认识却深入脑海。
	
	\section{《 雾都孤儿 》  }  	
		Oliver
		
		Bill
		
		Bigi
	
	\section{《 曾国藩家书 》   }	
		\verb|Zeng Guofan |
		\subsection{劝学}
			读书须有恒心。
			
			学问之道无穷,而总以有恒为主,兄往年极无恒,近年略好,而犹未纯熟。自七月初一起,至今则无一日间断,每日临帖百字,抄书百字,看书少须满二十页,多则不论。
			
			虽极忙,亦须了本日功课,不以昨日耽搁,而今日补做,不以明日有事,而今日预做。诸弟若能有恒如此,则虽四弟中等之资,亦当有所成就。
			
			切勿以家中有事,而间断看书之事,又勿以考试将近,而间断看书之课。虽走路之日,到店亦可看,考试之日,出场亦可看也。兄日夜悬望,独此有恒二字告诸弟,伏愿诸弟刻刻留心。
			
			
		\subsection{修身}
		\subsection{交友}

	\section{《 麦田里的守望者 》   }	
	
	
	\section{《 岛上书店 》   }	
		\subsection{往事}
			
		\subsection{现任}
		
		
		\subsection{句子}
			\begin{itemize}
				\item 如果没有好玩的轶事可以将给朋友们听,那些糟糕的约会还有什么意义呢? 
				\item 她三十一岁了,觉得自己到现在应该已经遇到某个人了,然而,...
				\item 大多数人如果能给更多事情一个机会话的话,他们的问题都能解决
				\item 任何一种花招,我都几乎没有共鸣。
				\item 找到一个跟你阅读兴趣相同的人又何其难啊
				\item 他在哈维眼里是多么微不足道,而哈维对他又是多么重要啊。
				\item 我喜欢看到它,让它可以提醒我什么时候我不想干了,什么时候就可以不干。
				\item 努力不让自己感觉像个被透了钱的老太太
				\item A.J 显得苍老许多,虽然觉得变老是一定的。
				\item 他内心残留的乐观的一面想去相信会遇到更好的东西
				\item 夜里,书店打样后,他又开始跑步,长跑中有很多难题,但是最大的难题这一是把钥匙放到哪里 
				\item 依我看,是人到中年变得更多愁善感了  
				\item 读小说需要在适合它的人生阶段却读
				\item 我们在20岁有共鸣的东西到了40岁的时候不一定能产生共鸣,反之依然,书本如此,人生依次。
				
				\verb|思悟:|有的人在有的时候可以产生感觉,而且有一种要永远在一起的感觉,而换个时间,就不一定有了。
				
				\item 失窃是种可被接受的事
				\item 我不愿意一夜不睡,也不想再像这本小说不得刺激我猛流眼泪那样流眼泪了
				\item 后来我发现,疼痛是件好事,他以前边跑边想事情,而疼痛让他可以不去做那种徒劳无益的事
				\item 此人似乎命中注定要见证A.j生活中所有的重要时刻
				\item 
				\item 
				\item 
				\item 
				\item 
				\item 
				\item 
				\item 
				\item 
			\end{itemize}
	
	\section{<想好就动手 - 林语堂>}	
		许多青年,常常在想定了一件事情以后,却还是犹豫不去进行;有许多人,天天在干着和他兴趣不合的工作,他们说起来总是说命运不好,等着机会,去干适当的事情。可是他们只是嘴里说,却不去干,如果一个人有了这样的惰性,那和自杀有什么两样呢?
		
		
		一般青年们,大多数留意一种成功的原素。这原素就是日积月累的经验,他们把事情看得过分容易,不肯集中所有精神,去不断努力。
		
		
		经验好比是一个雪球,它在人生之路上越滚越大,越滚越厚。任何人都应该把他的精力,集中在一事业上,随时工作,随时学习。你花费的工夫越大,得的经验也越多,而做起事来,也觉得格外来得方便。
		
		
		青年们你既然抱定了宗旨,为什么又不立刻去进行呢?你为什么不立刻去做你要做的工作呢?你既然打定了主意,就不应该再事犹豫了。你应该把你的精力,全部贯注到你所打定主意的工作中去。你如果准备做律师,你就专心致意于法律的研究,经过相当的时日,你的法律知识,自然会逐渐高深,你去出庭替人当辩护,一定也很出色的成功一位著名的律师。你在平时不要自己让步,不要以为我不比别人差,就算满足了。你必须随时研究,处处求进步。对于法律以外的学识,你尽可不问不闻,你不要去摸摸仪器,去动动画笔,碰碰刀斧,你的目的只有一个,就是律师是你唯一的事业。你必须成为见义勇为,辩才很好的大律师而不是要成为一个样样都得些皮毛的三脚猫!
		
		
		浪费时间的糊涂虫,专门消耗精力于放荡生活的愚笨者,快些醒来吧!
		
		
		你们这样过着放荡奢侈的生活,来糟蹋自己的精力和时间,实在就是社会中的蠢物啊!
		
		
		东碰西撞,左翻右倒没有一定主意的人,他们是永远不会有成就的,也永远不会有进步的。他不但停止了自己,而且还常常阻碍别人。他看见别人在做,就自夸自己也可以做;依他的话;似乎世界上的事物,他没有一件不会,也没有一件不精。而实际上他却什么也不会什么也不能动手!这种人要想成功一件事业,真比登天还难;因为他们整天只想抽出时间来快活,却不知道自己应该怎样去修养自己。
		
		
		“时间一去不复来,”这是一句最好的警语,你当初到社会上服务时,你总是带着满心的精力,你应该把全副精神灌注到你的事业中。无论你的事业是务农,做工,经商..你不要把时间在无形中溜走。
		
		
		哥德说:“你适宜站在那里,你就应该站在那里。”三心两意的人,读了这句话,可发生怎样的感想呢?这是警告你,不要东碰西撞了,依据你的个性,快决定你的事业,想好了立刻就动手。
		
		
		有些人工作虽然很努力,但是因为没有把精力集中,所以把精力一点点地消耗在无形的损失中。像漏了的水闸,他们不能把水挡住,而水却在漏洞中渗流出去了。

		
		一个精干的青年经理,在某一个时期,同时接着别的两个公司的聘请书,因为钦佩他的才干,聘请他去担任协理职务,可是他却都回拒了。有些关切他的朋友,问他为什么不愿接受别家公司的聘请,他们以为他有能力胜任兼理的工作。
		
		
		但是,他说:“我因为不愿把自己的精力分散,使各方面都受到损失。”
		
		
		是的,一点不错,一个人如果有了使精力可以渗漏出去的缝隙,这个人的成功,不知道要受到若干程度的损失。所谓缝隙,就是“心神不定”,他是一般成功希望的唯一仇敌。
		
		
		遇到一个困难,就愁眉不展,不知怎样才好。偶然逢到一点挫折,立刻就心灰意懒;碰着一些阻碍,就怀疑起来,想改做这样,改做那样,这分精力的漏洞,正不知使我们受到多大的损失;失去了,我们所积储着辛辛苦苦的资本,而结果,将是一无所成。
		
		
		无论什么人假使一开始就能善用他的精力,决定主意以后,立刻依着决定的目标循序渐进,不使它分散,那么,他们也是有成功的希望的。然而,他们偏不愿那样干,偏要东学一点,西做一下,因此把一生空费了,什么事也不能成功。
		
		
		任凭你怎样的聪明,任凭你怎样有天才,也任凭你以为有怎样了不起的天大本领,如果做事不把精力集中,不肯依据原定的目标去倾注全力,没有忍耐力,那么他的本领,天才,聪明都将毫无用处。
		
		
		你知道为什么一个有经验的园艺家,要把一颗植物的芽枝都剪除的原故呢?
		
		
		老实告诉你,为要使树木能生长得快,果实结得特别肥大,就非要这样不可。因为一棵的芽枝,是可以分散吸收根部的养料,使它的精力不能集中在树干和果实上。若不是这样做,他在收获上的损失,正不知要超越枝条的损失多少倍。
		
		
		那些有见识的花匠,也常把花蕾剪除去,只留着一个或两个在枝条上,难道他所剪除的花蕾,他们不会开美丽的花朵吗?不,不是这样的。他们的目的无非想使滋养料,都集中在一二个留下来的蓓蕾上,使将来开放的时候,格外美丽格外娇艳。
		
		
		我们的生长,又何尝不和花木一样呢?如果我们能集中精力在某一项事业中,那么,这件事业,一定可以获得十分美满的结果。
		
		
		肯集中精力,埋头苦干的人,他们的前途,真不知有多大的光明。你不要妄想,以为一个人同时可以成就多种事业的,无论怎样卓越的人,也决不能做到!
		
		
		你要在一件伟大的事业上获得成功,你立刻把所有微小,平凡,没有把握,和一切不适合自己的希望,都完全铲除,你更需用剪刀,大胆地把要分散你的精力的一切累赘,完全剪去。即使你所干的事情,已经获得相当的成效,但是你也得忍痛牺牲,否则,将来的损失,正不知比现在还要大上若干倍。
		
		
		世界上成千成万的失败者,并不是他们因为没有才能,也不是因为他们没有决心;而是由于他们不肯集中精力,由于他们不在决定主意以后,立刻动手去干,他们东碰西撞;这件试试,那件做做;他们既然学音乐,又想懂得点体育知识;同时又想干干地产事业,又希望做个实业家。这样还不够,又去研究一下法律,又写些文章去投稿,并且还学做文艺著作者;同时还准备当教师,而且还想成为诗人。他们尽量把精力分散开去,他们并不觉悟,假使能打定了主意,把精力集中在某一件事业,所获得的结果,会使他惊奇万分。
		
		
		一个人有了一种专门技巧,比有了十种本领的成功,不知要大得多少倍。
		
		
		因为他只注意着一种技巧,他在任何地方,都对于这方面下刻苦工夫。但是他如果注意于多方面,那么,他的结果还有什么成就可说呢?他有了十种本领,他将忙不过来了,他的精力,也不知如何支配了。事实上,一个人的精力是有限的,不可能面面俱到,而后来,他只能这样也敷衍一下,那样也将就一点,后来的成就,试间还有什么希望?
		
		
		现在的社会,是一个优胜劣败的时代,竞争的情形,一天比一天激烈,一天比一天紧张,一切事业,只有专门的人才把握得住,样样都会的,三脚猫一定给时代所淘汰。
		
		
		我现在确确实实的警告大家:
		
		
		一个现代的青年如果想在事业上获得胜利的成功只有:
		
		\begin{enumerate}
			\item 在一件事业上用工夫,把一切精力集中在一件事业中。
			\item 他必须在一种事业上,下最大决心埋头苦干。
			\item 他必须立志做一个专门人才,他必须在这种事业上随时求进步。
			\item 经验是累积成的学识,也是渐进的,你不能因为不可能立即成就而改变主旨。
		\end{enumerate}
		
		没有一种特长,什么都会一点的人,只配过平凡庸俗的生活,永远不会博得人们的崇敬和赞美,而专心致力于一种事业的,他在决定了事业以后,对于一切不相干无关的引诱,都能用自己的意志去拒绝,而向着他的目标前进。 	
	
	\section{三国演义}
		\subsection{第一章:桃园三结义,斩黄巾立首功}
			滚滚长江东逝水,浪花淘尽英雄。是非成败转头空。
			
			青山依旧在,几度夕阳红。
			
			一壶浊酒喜相逢,古今多少事,都付笑谈中。

			张角学之南山仙人术,起兵造反与黄巾。
			
			三国对刘备的描述,贤者:
			\begin{itemize}
				\item  那人不甚好读书;
				\item  性宽和,寡言语,喜怒不形于色;
				\item  素有大志,专结交天下豪杰;
			\end{itemize}
			张世平、苏双 助之;
			
			暂漏头角:
			\begin{itemize}
				\item 英雄露颖在今朝,一试矛兮一试刀。初出便将威力展,三分好把姓名标。(邓茂,程远志)
				\item 运筹决算有神功,二虎还须逊一龙。初出便能垂伟绩,自应分鼎在孤穷。
			\end{itemize}
			
			天下将乱,非命世之才不能济。能安置者,其在君乎?
			
			子治世之能臣,乱世之奸雄也。
			
			张角打败董卓,玄德救之,因白人无礼待之。
			
			安得快人如翼德,尽诛世上负心人。
\end{document}		